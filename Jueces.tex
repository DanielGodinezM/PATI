%        File: Jueces.tex
%     Created: Wed Sep 18 07:00 PM 2019 C
% Last Change: Wed Sep 18 07:00 PM 2019 C
%
%\documentclass[12pt]{article}
%\usepackage[margin=1.0in]{geometry}
%\usepackage{enumerate}
%\usepackage[spanish]{babel}
%\begin{document}
	\begin{section}{Jueces}
		\begin{itemize}
			\item Título\\
				Le fue dado ya que se refiere a la historia de los líderes que le dio Dios a su pueblo. Tanto el título en hebreo como el título en griego se traducen como ``Jueces''. Narra una época del pueblo en la que tuvieron un gobierno teocrático, sin rey. Antes del gobierno de los jueces el gobierno del pueblo estuvo centralizado en Moisés y Josué. La ley que Dios le dio a Moisés en el Monte sinaí también tenía un caracter civil además de religioso. Aunque la ley civil que se le dio a Moisés ya no es aplicable para nosotros, las leyes religiosas todavía las podemos entender como aplicables.\\
				A pesar de esas leyes claramente establecidas desde el principio, el pueblo de Israel constantemente recibía disciplina de parte de Dios pues los pueblos que habían quedado después de la conquista (sin aniquilar) se rebelaron e incluso llegaron a conquistar al mismo pueblo de Israel.\\
				Cuando Israel se alejaba de Dios se le imposibilitaba el llevar a cabo la conquista, por ello es que cuando el pueblo se arrepentía entonces clamaban otra vez a Dios y cuando Él veía un arrepentimiento sincero es que los libraba de sus opresores. Por ello es que Dios levantó jueces que los guiara cuya palabra en hebreo significa libertadores. El libro narra la historia de 13 de dichos jueces que cuando fueron guiados por el Espíritu de Dios se distinguieron de una manera especial para llevar a cabo la salvación del pueblo de Israel.\\
				Se establece un patrón de conducta del pueblo el cual se convirtió en un ciclo que constantemente se repitió,  se veía al pueblo sirviendo fielmente a Dios, después caía en pecado e idolatría, después Dios permitía que el pueblo fuera sometido por los pueblos que habían conquistado y una vez que el pueblo se daba cuenta del castigo de Dios se arrepentían, por último, Dios al ver su arrepentimiento levantaba un hombre que liberara a Su pueblo y una vez liberados se repetía el ciclo.
			\item Autor y fecha\\
				Por tradición judía y cristiana se le atribuye la autoría del libro a Samuel pero no hay seguridad en ello. Se escribe en tiempo pasado acerca del ``tiempo en el que no había reyes en Israel\ldots''\\
			Como contexto histórico se tiene en Jueces 1:21 que Jerusalén no era parte del pueblo de Israel. 
			\item Tema\\
				Se ve cómo el pueblo de Israel estuvo completamente alejado de Dios. Se manifiesta claramente la naturaleza caída del hombre a pesar de que ya habían visto grandes maravillas de Dios.
			\item Propósito\\
				Mostrar la debilidad de los hombres cuando nos separamos de Dios así como ver Su fidelidad constante.
		\end{itemize}
		\newpage
		\begin{subsection}{Bosquejo}
			\begin{enumerate}
				\item Victoria y obediencia incompleta (Jueces 1-2)\\
					En Jueces 2:1-3 Dios vuelve a reprender al pueblo por no hacer lo que Él les había ordenado.Dios ya no los iba a ayudar para conquistar a los demás pueblos pues era necesario que el verdadero arrepentimiento de Israel reflejara una verdadera obediencia y un verdadero cambio. No basta que nosotros sepamos que estamos pecando, debe de haber fruto en nuestro arrepentimiento pues debe de haber un cambio drástico en nuestra vida con el propósito de mostrar un verdadero arrepentimiento. Ahora ellos debían de depender de sus propias fuerzas y su propia preparación militar.\\
					En Jueces 2:10 se menciona que toda la generación que había entrado a la tierra prometida murió y la generación que siguió se empezó a olvidar del Dios que los había llevado ahí. Ello muestra que la generación que entró a la tierra prometida no les enseñó a sus hijos acerca de lo que había hecho Dios.
				\item La historia de los jueces (Jueces 3-16)\\
					Los jueces levantados por Dios empezaron a obedecer la ley de Dios, se convirieron en patriotas religiosas y Dios les hizo ver el ciclo vicioso en el que se encontraba el pueblo.\\
					Fueron principalmente 13 hombres que Dios ocupó en esta parte de la historia del pueblo de Israel.
					\begin{enumerate}
						\item Otoniel de la tribu de Judá (Jueces 3:9)\\
							Dios castiga al pueblo entregándolos en manos del rey de Siria. El pueblo clamó a Dios, se arrepienten y Dios levanta a Otoniel quien los libra de la esclavitud en la que estuvieron por 8 años, después los gobierna por 40 años.
						\item Aod de la tribu de Benjamín (Jueces 3:12)\\
							Cuarenta años duró la fidelidad de Israel antes de volver a caer en apostasía. Dios envía al rey de Moab quien los conquista y los somete a esclavitud. Nuevamente el pueblo se arrepiente y Dios levanta a Aod, un hombre zurdo. Aod consigue una entrevista con el rey donde lo termina asesinando como se relata en Jueces 3:15-22. Su libertad posterior duró 80 años.
						\item Samgar (Jueces 3:31)\\
							Probablemente era de ascendencia extranjera pues no se relata el nombre de la tribu a la que pertenecía y el origen de su nombre es turco, no hebreo, solamente se menciona el nombre de su padre Anat. Después de esos 80 años nuevamente el pueblo cae en la idolatría, nuevamente se arrepienten y Dios levanta a Samgar para que los libere de los filisteos, mató a 600 filisteos. No se dice explícitamente que fuera un juez ni tampoco detalles de su liderazgo.
						\item Débora de la tribu de Efráin y Barac de la tribu de Neftalí (Jueces 6)\\
							Dios usa a una mujer excepcional como profetiza y le provee a Barac como el líder militar. Veinte años fue oprimido Israel cuando Dios le revela a Débora que ella sería utilizada para liberar al pueblo junto con Barac.\newpage Él no quiso ir a la guerra sin Débora, ella accede pero le profetiza que Dios le daría la gloria por la victoria a una mujer y dicha profecía se cumple cuando en Jueces 4:21 una muejr llamada Jael es quien mata al capitán.\\
						\item Gedeón de la tribu de Manasés (Jueces 6)\\
							Cuarenta años duró la paz en Israel antes de que tuvieran que volver a ser disciplinados.En esta ocasión fueron los amadianitas y los amalecitas quienes invadieron la tierra, se llevaron las cosechas y dejaron destrucción a su paso. El Ángel de Jehová visita a Gedeón y lo anima para que libere a su pueblo. Gedeón entendió que había tratado con Dios y por ello él construye un altar. Gedeón era un hombre pobre pero Dios lo levanta para una muy importante misión.\\
							Lo tratan de matar pero él se levanta junto con varios hombres en contra de sus opresores tal y como se narra en Jueces 6:34.\\
En Jueces 6:37 Gedeón pusó una lana de oveja en el campo con el propósito de pedir señal que en la mañana la lana estuviera mojada pero campo seco. Dios hizo la señal tal y como Gedeón pidió, sin embargo, Gedeón volvió a pedir señal del llamado de Dios pidiéndole que ahora la lana estuviera seca mientras que el suelo estuviera mojado, en ambas ocasiones Dios le concede las pruebas que él pedía. En el capítulo 7 Dios le da una señal extra de que Él era quien iba a ganar la batalla pues Él le reduce el ejército a 300 hombres para librar a Israel.\\
Dios confunde al ejército enemigo para que se ataquen entre ellos y así obtienen la victoria sin ocupar a los 300 hombres. Entonces Israel quiere hacer rey a Gedeón pero él se niega pues Dios es el rey de Israel. Se narra que muere y que tuvo 30 hijos.
\item Abimelec de la tribu de Manasés (Jueces 9)\\
	Cuarenta años después se provoca una guerra civil cuando Abimelec mata a los 70 hermanos que él tenía. Cuando Israel le ofreció a Gedeón el título de rey y él lo negó, dijo que tampoco sus hijos serían rey sobre Israel pero Abimelec, siendo uno de los hijos de Gedeón, se hizo rey en Siquem donde el pueblo lo traiciona. Él muere cuando le tiran una piedra sobre su cabeza.
\item Tola de la tribu de Isacar (jueces 10:1)\\
No se menciona de quién liberó al pueblo, solamente que lo gobernó por 23 años.
\item Jair de Galaad (Jueces 10:3)\\
	No se dice de quien liberó al pueblo pero se menciona que los gobernó por 22 años.
\item Jefté de Galaad\\
	Debido a la apostasía de Israel, Dios permite que los filisteos y los amonitas los conquisten. Dios levanta a Jefté, un hijo ilegítimo. Él huye y se dice que se relacionó con hombres sin escrúpulos formando así cualidaddes de liderazgo ocupadas por Dios para que liberara a su pueblo.\newpage
	Se encuentra una dificultad de interpretación cuando Amón le hizo la guerra a Israel pues Jefté siendo diestro para la contienda derrota a los amonitas y le hace un voto a Dios en Jueces 11:30-31, dicho voto muestra una degeneración de las prácticas cúlticas en Israel pues nunca se narra el sacrificio de un ser humano como mandamiento de Dios. Se ve el suceso trágico de que su única hija es quien lo recibe y por ella es que él se adolece en Jueces 11:35. Le concede un tiempo de 2 meses para llorar su virginidad y pasados los 2 meses hizo con ella de acuerdo al pacto. Jefté ofrece en holocausto a su propia hija. Jefté particpa en una guerra civil en contra de Efraín al cual también derrota. Fueron 6 años en los cuales él gobernó.
\item Ibzán de la tribú de Judá (Jueces 12:8-10)\\
	Juzgó a Israel solamente por 7 años.
\item Elón de la tribu de Zabulón (Jueces 12:11-12)\\
	Juzgó a Israel por 10 años.
\item Abdón de la tribu de Efraín (Jueces 12:13-15)\\
	Juzgó a Israel por 8 años.
\item Sansón de la tribu de Dan (Jueces 13-16)\\
	Dios los entrega en manos de los filisteos una vez más. Sansón fue un hombre al cual se le determinó que se dedicara únicamente a las cosas de Dios y fue elegido para liberar al pueblo de Israel de los filisteos. Pecó en sobremanera, su perdición fueron las mujeres pues hasta tomó esposa de los filisteos. En Jueces 14:6-7 mata a un león con sus manos. En Jueces 15:15-16 mata con una quijada de asno a 1000 filisteos. En el capítulo 16 aparece Dalila del valle de Sorec. Por lo que dice la escritura, Sorec estaba dentro de la tierra de Canaán en un territorio israelita, ella fue tentada poderosamente por los filisteos para traicionar a Sansón por 5500 ciclos de plata, una cantidad bastante abundante. Se ha determinado que los 5500 ciclos de plata equivalen a 15 millones de dólares actuales aproximadamente. Dalila entrega a Sansón pues hizo que él durmiese sobre sus rodillas y llamó a un hombre que le cortara el cabello, como se relata en Jueces 16:19-20. Se celebra que habían derrotado a Sansón y en una petición de Sansón, Dios le regresa la fuerza derrumbando el templo y así muere él junto con todos sus enemigos.
							
					\end{enumerate}
			\end{enumerate}
		\end{subsection}
		\begin{subsection}{Confusión civil y religiosa (Jueces 17-21)}
			Se ve cómo Israel estaba totalmente apartado del conocimiento de Dios y de Su ley. Se narran situaciones creadas por el pueblo que nos muestra la gran confusión que tenían al no poder comprender la voluntad de Dios. Micaía le roba a su propia madre y ella maldice al ladrón sin saber que se trataba de su hijo. Él siente miedo por la maldción de su madre a pesar de no tener miedo a la maldición de Dios, Micaía confiesa a su madre y ella buscando agradar a Dios opina hacer un altar de adoración con lo robado. Micaía consagra a un sacerdote del altar que él había construido. El sacerdote que consagra es su propio hijo que claramente no era de la tribu de Leví. En Jueces 17:6 se muestra que no había rey y que cada uno hacía lo que bien le parecía.\newpage
			En el capítulo 18 se narra que sus imágenes fueron robadas pues el pueblo pensaba que dado que el sacerdote sí era levíta entonces Dios podía prosperar sus planes. Se lo llevan todo y el levita se alegra de obtener un beneficio económico en vez de reprender al pueblo. Lo absurdo de la idolatría de Micaía donde se jacta de los dioses que él mismo hizo. Los danitas se llevaron todo el centro de adoración que tenía Micaía en su casa. En Jueces 18:30 se revela que el levita era Jonatán hjo de Gersón, nieto de Moisés. Ni los mismo descendientes de Moisés pasaron las pruebas de fidelidad a la ley que Dios les había dado. 
			En los capítulos 19-21 se presentan a 2 personajes principlaes, un levita y su concubina. La concubina le había sido infiel al levita, cosa que no iba de acuerdo a la conducta de santidad que debía de tener un sacerdote. Ella debió de ser apedreada pero él la perdona y busca ganarla de nuevo. Cuando ellos llegaron a la ciudad el sol ya se estaba poniendo, entran a la ciudad a buscar hospedaje con sus hermanos israelitas. En Jueces 19:20 un anciano le ofrece hospedaje en su casa y le ofrece un banquete de bienvenida como lo decía la ley donde son interrumpidos por maleantes pues querían violar al levita.\\ \\
			En Jueces 19:15-25 se ve una historia que se asemeja mucho a la de Sodoma. El anciano anfitrión se refiere a los rufianes como hermanos y les ofrece a su propia hija y a la concubina para que hicieran lo que quisieran. El levita compartió la apreciación con su hospedadores y entrega a la concubina para que no le hagan daño, la violaron toda la noche como lo narra Jueces 19:25 y agotó sus fuerzas, cuando regresó a la casa encontró que le habían cerrado la puerta pues el esposo y el anfitrió ya se habían dormido.\\
			El levita se levanta por la mañana y al salir de la casa la presiona para continuar su viaje, al encontrar a la mujer no le expresa compasión ni arrepentimiento solamente le dice que se levanta para que se vayan. Él se da cuenta que está muerta, la carga sobre su asno y la lleva hasta su casa de acuerdo a Jueces 19:28. Una vez que llega a su casa la corta en pedazos y envía los pedazos a cada una de las 12 tribus de Israel para anunciar la atrocidad del crimen y buscando que lo castigaran. El mensaje macabro provocó una  reacción fuerte por parte de las tribus de Israel. Se entabla una batalla fraternal contra la tribu de Benjamín en Jueces 20:41. Después de 25 mil bajas practicamente había desparaecido la tribu de Benjamín. Aniquilaron a prácticamente todos los hombres de la tribu de Benjamín, estaban tan airados que mataron a 40 mil de las demás tribus de Israel y destruyen a todas las ciudades de Benjamín matando a todos. Los israelitas molestos habían jurado no casar a sus hijas con los benjamitas pero puesto que habían matado a todas las mujeres de benjamín, los hombres restantes ya no tenían con quien casarse. Era necesario que se solucionara el problema para que la tribu no se extinguira. Israel recapacita con el propósito de que Benjamín no desaparezca y para compensarlos como un acto de compasión le dan las doncellas de los habitantes de Jabes pero solamente consiguieron 400 mujeres después de aniquilar a los hombres de Jabes. Instruyen a los benjamitas a que tendieran una emboscada a las mujeres que iban a danzar con el propósito que raptaran a una mujer exclusivamente para cada hombre. Jueces 21:25 termina repitiendo que dado que carecían de liderazgo, cada quien hacía lo que quería.
		\end{subsection}
	\end{section}
%\end{document}


