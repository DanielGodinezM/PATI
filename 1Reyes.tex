%        File: 1Reyes.tex
%     Created: Mon Sep 30 06:00 PM 2019 C
% Last Change: Mon Sep 30 06:00 PM 2019 C
%
%\documentclass[12pt]{article}
%\usepackage[margin=1.0in]{geometry}
%\usepackage{enumerate}
%\usepackage[spanish]{babel}

%\begin{document}
\begin{section}{Primer libro de Reyes}
	\begin{itemize}
		\item Título\\
			Inicialmente era un sólo libro al gual que los libros de Samuel y el nombre le fue dado de igual manera al traducirlo al griego y dividirlo.
		\item Autor y fecha\\
			Aunque se han dado nombres de posibles autores, tradicionalmente se considera que el autor es anónimo y que estuvo cautivo en Babilonia. Seguramente los libros fueron escritos poco después del 561 a.C. \\
			Más que narrar la historia de Israel, trata de explicar las razones de Dios para enviar un juicio en contra de Su pueblo. Ellos eran merecedores de dicho juicio por la idolatría que habían mostrado.
		\item Tema\\
			División del reino de Israel.
		\item Propósito\\
			Mostrar a partir de la divisón del reino, que la necedad humana se siguió oponiendo a los planes de Dios.
	\end{itemize}
	\begin{subsection}{Bosquejo}
		\begin{subsubsection}{Reino unido ($1^{o}$ Reyes 1-11)}
			\begin{enumerate}
				\item Salomón el rey sabio ($1^{o}$ Reyes 1-4)\\
					Los crímenes en su propia familia empujaron a David a una impotencia física. Como conlusión de su vida le traen a una mujer joven que lo iba a acompañar en sus últimos días en su cama. Esta hermosa concubina solamente le sirvió como enfermera, ésta práctica de poner a un anciano con una joven virgen y hermosa era común pues se creía que dicho contacto tenía poderes restauradores, es por eso que sus siervos eligieron a esa sunemita, Abisag. No hay base histórica para identificar a esta mujer para conectarla con la sunemita que se menciona en Cantares. Se debe de interpretar que solamente fue una sierva que acompañó a David en su vejez, cuando tenía aproximadamente unos 75 años. Probablemente ya dada su condición de salud, David ya había perdido control sobre su pueblo. \\ \\
					En $1^{o}$ Reyes 1:5 se dice que entonces empezaron los problemas con sus hijos pues querían tener el trono. El cuarto hijo, Adonías, tenía ambición de heredar el trono de su padre. Adonías empezó a proclamar que él sería el futuro rey de Israel. Organizó una serie de actos en los que se adelantaba a nombrarse rey antes de que muriera su padre.\\
					Su primo y general del ejército, Joab, apoyaba a Adonías en su plan para proclamarse rey. Había otro hombre en el reino de David, el sacerdote Abiatar, descendiente de Elí quien recibió a Samuel, que también se unió a Adonías. Abiatar era sacerdote a pesar de que Dios ya le había dicho a Elí que su descendencia no podía ejercer el sacerdocio.\\
					\newpage
					Adonías se aprovecha de la situación para tramar un complot, él mismo trató de convencer a sus hermanos de que él era el heredero legítimo del reino de su padre. \\
					Adonías tenía la ventaja de que era el heredero principal y además era un persona que tenía una gran popularidad entre el pueblo de Israel. Inició una conspiración festejando una fiesta, sin invitar a aquellos que él consideró que se iban a oponer a su plan de autoproclamarse rey. En $1^{o}$ Reyes 1:10 se menciona que no invitó a Natán, profeta y amigo de david, Benaia que David nombró como comandanete en lugar de Joab, Salomón pues era hermano de Adonías y sabía que tenía la preferencia del rey y que tampoco invitó a la guardia personal del rey.\\
					Éstos hombres leales a David, como Natán, ven este acto de Adonías por lo que lo intentan detenerlo y avisarle a David. Natán hace un cuidadoso plan para informar al rey acerca de lo que estaba pasando, su plan era que precisamente Betsabé fuera quien le informara al rey ya que ella tenía más cercanía con él.\\ \\
					Una vez que se enteró David por medio de Betsabé y Natán, aun estando enfermo, David entró en acción. El rey mostró que todavía tenía fuerza mental y voluntad pues David tenía la intención de cumplir con su promesa ya que por medio del pacto ya estaba destinado que Salomón sería quien continuara esa línea de descendencia, David da instrucciones clave para que Salomón fuera el heredero sin que quedara duda. David les instruye a Sadoc, Benaía y a Natán que pusieran a Salomón sobre la cabalgadura real en la mula del rey, todo esto con la presencia de los 3 siendo ellos el profeta de Dios, el líder espiritual y el líder militar. En tiempos de David los mulos eran reservados para gente de cierto rango, en especial, de la familia real. Es la única vez en toda la Escritura que se menciona una mula y no un mulo, esto es, en $1^{o}$ Reyes 1:33.\\
					Recordando la entrada triunfal de Jesús, el hecho de que entrara en un mulo le daba el reconocimineto de rey.\\
					Esta es la primera vez que un hijo de un rey asciende al trono de su padre en la historia de Israel. Por ello es que hubo un desbordante júbilo por parte del pueblo ante la noticia en $1^{o}$ Reyes 1:39.\\ \\
					El ungimiento tenía varios significados pues se podía ungir a los sacerdotes pero también era notorio que se ungiera a un gobernante, es por ello que cuando alguien subía al trono se le llamaba el ungido. Por parte de adonías, se descubre su complot y todos lo abandonan. A Adonías no le queda más que mostrar humildad y sabía que por justicia debía de pagar con su vida. Por ello es que corre al templo y se aferra a los cuernos del altar. Adonías estaba reclamando la protección de Dios, estaba implorando misericordia.\\
					De acuerdo a la ley, no se podía matar a una persona que entrara al templo aunque fuera culpable y Adonías tenía que mostrar una absoluta redención ante el rey. El primer acto de Salomón fue perdonarle la vida a su hermanastro con la condición de que Adonías jamás volviera a intentar ganar autoridad en el reino y fue sujeto a una especie de arrresto domiciliario, sin posibiblidad de alcanzar algún reino, quedó siendo solamente uno más del pueblo de Israel. 
					\newpage 
					Toda esta problemática se sucitó ya que David nunca aclaró quién iba a ser su sucesor y fue como una invitación para que alguien llegara a tomar su trono pero Salomón finalmente queda establecido como el tercer rey de Israel en $1^{o}$ Reyes 2:12.\\
					En $1^{o}$ Reyes 2:10-11 se menciona finalmente la muerte de David y se consilda el reino de Salomón. David le dijo a su hijo que fuera un hombre sabio, prudente y magnánimo y que el futuro y destino de su reinado dependería de su fiel cumplimiento de las promesas de su padre. A diferencia de David que llevó una vida muy difícil, Salomón vivió en la tranquilidad y en los lujos del palacio real. A David le formó el carácter todos los problemas que pasó antes de convertise en rey e incluso yendo al frente del ejército de Israel.\\
					Había muchas cosas que resolver en el reino de David pues ya en su vejez ya no pudo concluir con muchas cosas, entre ellas, el confilcto que había tenido con Joab. David no fue vengativo, fue magnánimo y misericordioso pero había llegado el momento de que se ajustaran las cuentas, David le recordó a Salomón lo que había hecho Joab en $1^{o}$ Reyes 2:5-6.\\ \\
					Igual le recordó de Simei y de la maldición que le había dicho en $2^{o}$ Samuel 16:5-8. David había perdonado sus injurias pero el principio real había sido agraviado. Este hombre debía de ser castigado por haber cometido un delito en contra del estado ya que maldecir al rey era una grave falta. Además, se consideraba que para romper la maldición se debía de aniquilar a aquel que la había dicho. Finalmente muere David en $1^{o}$ Reyes 2:10-11. Cuarenta años reinó David, 7 en Hebrón y 33 en Jerusalén. David fue sepultado en Jersualén, lo cual era de extrañarse pues lo común era que lo sepultaran con sus padres pero David es sepultado en la ciudad que él había conquistado como el fundador de una nueva dinastía. Los reyes futuros también fueron sepultados en Jersualén. El reino de Israel sería eterno bajo la mano sabia de Salomón, el pueblo debía de continuar y ellos iban a alcanzar el momento cumbre de su historia. Salomón también iba a reinar por 40 años, la era más gloriosa de Israel.\\ \\
					Desde el día de sus coronación Salomón mostró que era un hombre de paz, buena voluntad y con la intención de que su pueblo prosperara. El pueblo iba a ser testigo de la actuación del nuevo rey. Fue considerado como un hombre glorioso por su pueblo y aún por otros pueblos, ésto se debe a que Salomón en un principio sólo amaba al Dios de su padre. En $1^{o}$ Crónicas se ve con mayor detalle la muestra del poder de Salomón.\\
					Y Jehová engrandeció en extremo a Salomón, tal y como se menciona en $1^{o}$ Crónicas 29:23-25 y $1^{o}$ Reyes 2:12. Los relatos del libro de Reyes son desde el punto de vista humano mientras que los de Crónicas son desde el punto de vista divino. Salomón heredó un reino completamente establecido pero que aún tenía problemas próximos. Es por ellos que su primer mandato es eliminar toda aquella cosa que pudiera alterar la paz en el reino.\\ \\
					Adonías verdadermante nunca se arrepintió e hizo una petición insólita a la madre de Salomón, hizo el intento de apoderarse del trono de David de una manera psicológica. Adonías le dijo a Betsabé que le pidiera a Salomón la mano de de Abisac para él, ella era la mujer que había acompañado a David en sus últimos días.
					\newpage
					Parece ser una petición normal pero culturalmente se pensaba que desear a la mujer o a cualquier otra cosa del rey era desear la posición del rey y cuando Betsabé le informa ésto a Salomón considera que Adonías lo estaba traicionando de nuevo. Ésto fue lo que provocó que Salomón le diera muerte a su medio hermano y le encarga esta tarea a Benaía por parte de Salomón.\\
Cuando Joab oyó la noticia de que Salomón había mandado la ejecución de Adonías supo que también le tocaba su turno e hizo lo mismo que Adonías pues fue al templo, tomó los cuernos de bronce del altar para implorar misericordia a pesar de que era culpable. En este caso era claro que Joab era culpable y otra vez Benaía es verdugo pero ahora con Joab.\\
En el capítulo 3, se ve que Salomón en el trono empieza la edad de oro. Siguió con la costumbre de hacer alianzas con otros reinos al casarse con otras mujeres. Los reinos cercanos a Israel no se atrevían a negarle las hijas de los reyes a Salomón pues conocían su poder y sabiduría. Los reinos que se aliaban con Salomón adquirían poder y riqueza.\\ \\
Políticamente era aceptable, de acuerdo a las políticas de ese tiempo y decide esposarse con la hija del faraón egipcio pues Salomón sabía que con ello iba a aumentar su poder a pesar de que en la ley de Moisés se dice que no debían mesclarze con mujeres de otros pueblos.\\
Salomón amaba a Jehová, procuró seguir sus estatutos y era muy devoto pues dice que ofrecía muchos holocaustos. Sin embargo, una de las cosas que más le gradaron a Dios fue que cuando Dios le dijo a salomón que le pidiera lo que quisiera y se lo daría, Salomón le pide sabiduría para gobernar en ese reino. Además Dios le agradó tanto que también le dio riqueza y gloria, en adelante a pesar de ser rico era un rey de gran sabiduría. En $1^{o}$ Reyes 3:24-25 se ve un ejemplo de la sabiduría pues Salomón sabía que la verdadera madre del niño no iba a aceptar que mataran a su hijo.\\
Por las múltiples alianzas que él hizo, no le temían pues al contrario eran amigos de él. Compuso 3000 proverbios y su sabiduría era conocida entre todos los pueblos.
\item Salomón y sus construccines ($1^{o}$ Reyes 5-7)\\
	La obra más grande y más significativa que hizo Salomón fue el templo para Jehová. Esta obra tan imponente para su tiempo, finalmente fue destruida por el rey Nabucodonosor. A pesar de que era el lugar en el que se había establecido la adorción a Dios, el templo fue de mayor importancia para el pueblo que para Dios mismo. Después de que fuera destruido el templo, Zorobabel construyó otro templo que también fue destruido por otro hombre que se conoce por la historia que fue Epífanes. Finalmente cuando fue reconstruido por Herodes en el tiempo de Jesús, fue finalmente destruido por los romanos en el 70 d.C.\\
	En el capítulo 7 se narra a detalle otras grandes obras que hizo Salomón con la riqueza y los esclavos que tenía. Una de ellas fue su propoia casa, la casa del bosque del líbano y la casa para la hija del faraón donde todas ellas fueron monumentales.
\item Salomón, su sabiduría y riqueza ($1^{o}$ Reyes 8-10)\\
	Debida a esa estrecha comunión que tenía con Dios, Él siempre apoyaba a Salomón.
	\newpage
	Ahora se narra cómo el arca es llevada al lugar santísimo y se narra una oración que hace Salomón.\\
	En el capítulo 9, nuevamente Dios se le aparece Salomón y le dice que si anduviera como David, Dios afirmaría su trono para siempre pero si se aparta, entonces Dios lo iba a echar de delante de Él y a la casa que Salomón le había construido. Salomón cobraba enormes impuestos y su riqueza fue creciendo aún más y más. En $2^{o}$ Crónicas 9:3-28 se nombra la enorme riqueza que tuvo Salomón.\\
	 En el capítulo 10 se narra la visita que le hizo una reina que venía de Saba, un gobierno que estaba al sur del reino de Salomón, no se dice el nombre de la reina pero Salomón le muestra su sabiduría y sus riquezas de forma que la reina quedó impresionada. La Biblia no dice que Salomón tuviera una relación amorosa con esta reino, sin embargo, tradicionalmtene así se piensa e inclusive se piensa que de ahí surgieron los etiopes que tenien ascendencia judía.
 \item Salomón y su declive ($1^{o}$ Reyes 11)\\
	 Finalmente muestra su naturaleza, Salomón también tuvo una gran debilidad por las mujeres pues además de la hija del faraón, tomó más mujeres extranjeras. Su amor por Jehová fue reemplazado por el de muchas mujeres que llegó a poseer y su lealtaad que en un principio era para un solo Dios fue transferida a todos los dioses de sus mujeres. En $1^{o}$ Reyes 11:3 se menciona que tuvo 700 mujeres reinas y 300 concubinas y que las mujeres desviaron a su corazón. Dios le había advertido que las mujeres extranjeras le desviarían su corazón hacia otros dioses. Esas mujeres, cuando Salomón ya fue viejo, inclinaron su corazón hacia otros dioses. Inclusive Salomón llegó a hacer altares para otros dioses. Eso encendió la ira de Dios y rompió el reino de Salomón, le dijo que solamente le dejaría una pequeña parte del reino por la promesa que le había hecho a David. Dios no rompió su reino mientras estaba Salomón como rey sino hasta que lo sucediera su hijo. Al final del capítulo 11 se narra la muerte de Salomón.\\
	 Cuarenta años reinó Salomón, el mismo teimpo que David y fue enterrado igual en la ciudad de David.
			\end{enumerate}
		\end{subsubsection}
		\begin{subsubsection}{Reino dividido}
			\begin{itemize}
				\item Roboam y Jeroboam ($1^{o}$ Reyes 12-14)\\
					Cuando el reino se dividío, el reino de Efraín era una de las partes más importantes en el norte. A Jeroboam, Dios le manda un profeta que le dice que Dios lo había escogido para darle una parte del reino. Jeroboam huye a Egipto para evitar que Salomón lo matara y regresa hasta que se entera de la muerte de Salomón. El hijo de Salomón es nombrado como sucesor y apenas había sido nombrado rey cuando llega Jeroboam y le advierte que a menos que disminuyeran las demandas y las cargas, solamente con esa condición le serviviría. Salomón había obligado al pueblo a trabajar en exceso y les cobraba impuestos altísimos con el propósito de seguir agrandando la riqueza del reino.\\
					Roboam incialmente hace lo debido, toma consejo de los ancianos que le habían aconsejado a su padre pero también va y toma consejo de sus amigos jóvenes que eran igual de inmaduros que él. 
					\newpage
					Roboam sigue el consejo de sus amigos pues pensaba que era el que más le convenía negando la demanda de Jeroboam en $1^{o}$ Reyes 12:14. Cuando el pueblo de las 10 tribus del norte escuchó eso dijeron que no tenía heredad con el hijo de isaí. Entonces el resto de las tribus de Israel se fueron a sus tiendas y así se divide el reino de Israel. La división del reino fue una consecuencia de que Salomón se separara de los mandamientos de Dios.\\
					Nombran a Jeroboam como su nuevo rey y solamente en el sur queda la tribu de Judá y de Benjamín. Roboam intentó hacer que regresaran las tribus por medio de las armas pero Dios se lo impide. A partir de $1^{o}$ Reyes 12:25 se ve que Jeroboam desaprovechó la oportunidad que Dios le había dado y más aún, Jerobaom pensaba que a pesar de que estaban políticamente divididos, estaban espiritualmente unidos.\\ \\
					Jeroboam empieza a transmitirles una nueva religión con nuevos cultos y dioses. Se fabrican 2 becerros de oro y nombra sacerdotes que no eran de la tribu de Leví. En $2^{o}$ Crónicas 11:14 se dice que cuando los levitas vieron esto tuvieron que dejar esas ciudades para regresarse a Judá y a Jersualén. Ya no tenía caso que los levitas se quedaran en el norte pues la adoración ya no era al Dios de Israel. Jeroboam instituye fiestas sin nada que ver con las fiestas que Dios le había dado a Moisés. Estas nuevas fiestas eran hechos distinguidos de la historia de Israel, sin embargo, en el capítulo 14 se narra la muerte de Jeroboam  y que en el reino del sur los de la tribu de Judá se desviaron y adoraron imágenes en contra de lo que la ley había escrito. Entonces Dios permite que los vecinos, los enemigos de Judá, los invadieran y en el caso de Egipto llegaron hasta Jerusalén y saquearon los tesoros del templo. En este capítulo se menciona la muerte de estos primeros dos reyes del reino dividido.\\
		
				\item Reyes de Israel y de Judá ($1^{o}$ Reyes 15-16)\\
					La sucesión tal y como la había prometido Dios en el norte fue a través de distintas familias. Jeroboam se le ha señalado como aquel quien hizo pecar a Israel, sus sucesores solamente siguieron el ejemplo de Jeroboam. Los reyes de Israel que se narran en estos dos capítulos son Nadab, hijo de Jeroboam por 2 años y Baasa que reinó 24 años, Ela que reinó 2 años, Zimri por 7 años, Omri por 11 años y Tibni por 3 años. Todos ellos procedentes de distntas familias. En el caso de Judá no fue diferente la situación, Abían sucedió a Roboam, Asa por 41 años. Hizo lo recto ante Jehova como David y fue importante pues había más apostasía en el reino del norte y por ello es que algunos se fueron al reino del sur al ver el buen gobierno de Asa.\\
				\item Elías y Acab ($1^{o}$ Reyes 17-22)\\
					Se narran paralelamente la vida de ambos hombres. Elías muy importante en la historia de reyes y Acab rey del norte que se dice que hizo lo malo más que los que habían reinado antes que él. Al reino del norte se le conocía como Israel, Samaria, Efraín, entre otros mientras que al del sur se le llamaba Judá para diferenciarlos.\\
					Acab es el hijo de Omri en el reino del norte. Se dice que se casó con la hija de un sacerdote fenicio con el nombre de Jezabel, esto se menciona en $1^{o}$ Reyes 16:30-31.
					\newpage
					En el reino del norte Elías predice años de sequía sobre Israel, debido esta profecía que había hecho por instrucciones de Dios es que se ve que Acab que reinaba, culpa a Elías de la sequía por lo que Elías huye del reino. Elías es alimentado milagrosamente por Dios a través de unos cuervos en $1^{o}$ Reyes 17:6 y posteriormente por una viuda. De manera milagrosa resucita al hijo de la viuda. Elías todo el tiempo anduvo huyendo pero Dios manda a Elías a que se presente ante Acab que iba a hacer que volviera a llover pero antes Elías lo iba a confrontar con su pecado pues Acab estaba adorando al Dios de Jezabel. Elías le pide a Acab que congregue a todo el pueblo con profetas de Baal y de Asera. Elías confronta al pueblo y les dice que se vuelvan al Dios verdadero. Elías reprende al pueblo en $1^{o}$ Reyes 18:21, sin embargo, el pueblo de Israel se había vuelto sincretista pues adoraban a cualquier Dios que les pusieran en frente. El pueblo no respondió y Dios a través de Elías hace una muestra increíble acerca de Su poder. Elías manda al pueblo a matar a los profetas de Baal en $1^{o}$ Reyes 18:40.\\ \\
					Elías le dice a Acab que ya iba a acabar la sequía en $1^{o}$ Reyes 18:41. Aparece otra vez Jezabel, una mujer malvada y ambiciosa y cuando se entera que Elías había mandado a matar a sus profetas entonces lo busca para matarlo. Elías se sentía en una situación desesperada pues era una persecución constante de una mujer que clamaba venganza. Se esconde en una cueva y ahí Dios habla con Elías, en la misma cueva en la que Dios habló con Moisés. Elías se siente derrotado y permite que las circunstancias le afetcten negativamente, para colmo ve que su propia vida peligra ante la mano de Jezabel. Elías desconoció el poder de su Dios y entonces Dios le da una muestra visible de Su poder en $1^{o}$ Reyes 19:11-13.\\
					Dios puede manifestar su gloria en distintas formas, utiliza los elemtnos de la naturaleza para mostrar su poder. Elías necesitaba aprender que después de la tromenta siempre venía la calma. Dios iba a utiizar a Elías para los propósitos que Él tenía. En $1^{o}$ Reyes 19:15-16 se le dan 3 instrucciones las cuales llegaron a cumplirse como Dios lo había determinado más adelante estas tres instrucciones tuvieron una gran importancia. \\ \\
					Al final del capítulo 19, llama a Eliseo que sería la persona que lo sustituiría. En el capítulo 20 se ven las victorias en Siria de Acab pero en el capítulo 21 aparece otra vez la maldad de Jezabel pues Acab anhelaba la viña de un varón que no se la quería vender y Jezabel lo manda matar para que su esposo se quedara con la viña, cosa que indignó grandemente a Dios. Hubo una gran alianza entre las dos partes de Israel, Acab hace alianza con Josafat que era en ese momento el rey de Judá. Ambos reyes se preparan para una contienda militar para liberar a alguien que estaba en manos de los asirios pero Josafat que venía de la dinastía de David, sí era temeroso de Jehová, no le hizo caso a los profetas de Acab y pidió que trajeran a un verdadero profeta de Dios. Llegó Micaías del norte a quien Acab se quejaba que no le profetizaba lo que él quería. Micaías les profetiza la verdad de Dios pues no iban a salir victoriosos. \\
					Acab se enfurece y manda a que Micaías fuera golpeado pues no le profetizó lo que él quería. Acab iba a morir en esa batalla pues Micaías lo profetizó en $1^{o}$ Reyes 22:28.
					\newpage
					Las palabras del profeta eran las palabras de Dios y así sucedió pues muere Acab en la batalla tal y como lo había profetizado también Elías en $1^{o}$ Reyes 21:19, Nabot había sido la persona a quien había matado para quitarle su viña.
			\end{itemize}

		\end{subsubsection}
	\end{subsection}

\end{section}
%\end{document}


