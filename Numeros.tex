%        File: Numeros.tex
%     Created: Mon Sep 09 06:00 PM 2019 C
% Last Change: Mon Sep 09 06:00 PM 2019 C
%
%\documentclass[a4paper]{article}
%\usepackage{enumerate}
%\usepackage[spanish]{babel}

%\begin{document}
\begin{section}{Números}
	\begin{itemize}

		\item Título\\
Proviene del griego ``arithmos'' de donde viene aritmética, el título le fue dado cuando se formó la Septuaginta y literalmente significa ``Números'' debido a los censos que se tienen después de la salida de Egipto.\\
El título en hebreo es ``bemidbar'' cuya mejor traducción es ``en el desierto''.
		\item Autor y fecha\\
Durante el peregrinar en el desierto Moisés escribió este libro junto con los demás libros del Pentateuco. Como tal es una continuación del libro del Éxodo ya que durante el libro de Levítico el pueblo no se movió del pie del monte Sinaí. Narra las peregrinaciones a través del desierto por diferentes circunstancias.\\
No todos lo hechos ocurridos durante el peregrinaje se narran en el libro, del momento que salieron de monte Sinaí a la entrada a la tierra prometida pasaron 40 años, el libro narra por qué es que el pueblo tardó tanto.\\
El libro termina cuando el pueblo se encuentra del lado oriental del río Jordán ya preparado para entrar a la tierra prometida. Habla de la preparación de una nueva generación quienes iban a conquistar finalmente la tierra prometida.
		\item Tema\\
Habla del fracaso de Israel. Como en el caso de Adán, no se les puede condenar de su fracaso a pesar de su cercanía con Dios ya que cualquier hombre hubiera fallado debido a su naturaleza pecaminosa.\\
		\item Propósito\\
			Mostrar que a pesar del fracaso de Su pueblo, el plan divino tenía que seguir adelante pues Dios es fiel.

	\end{itemize}
	\begin{subsection}{Bosquejo}
		\begin{subsubsection}{El pueblo ordenado (Números 1:1-10:10)}
			Todavía se encuentran al pie del monte Sinaí y el pueblo es distribuido por tribus alrededor del Tabernaculo, cada una contaba con una bandera distinta.\\
		También estaba organizado ya el orden de partida y de marcha dependiendo de cada tribu. El orden estaba perfecramente establecido por Dios, siempre el campamento estaba alrededor del Tabernaculo en el mismo orden. Era costumbre que la tienda del rey se colocaba en el centro del campamento, así se reconocía que Dios era el rey de Su pueblo con el Tabernáculo en el centro.\\
		En el capítulo 5 se dan instrucciones para evitar la contaminación el campamento. Cada una de la reglas que tenía el pueblo tenía el propósito de que el pueblo permaneciera en santidad. Si el pueblo quiere gozar de las bendiciones divinas debe de ser un pueblo santo pues, como se mencionó en Levítico, Dios es Santo.\\
		El concepto de inmundicia también se refiere a contaminación física tales como las distintas enfermedades debido a falta de higiene o aun la misma muerte. El pueblo debe de saber cómo librarse de la contaminación espiritual y física, por ello es que el papel de los sacerdotes erea sumamente importante ya que checaban que el pueblo permanecier en santidad.\\
		Para Números 9, ya había pasado 1 año de la salida de Egipto y el pueblo celebra por primera vez en el desierto del Sinaí la Pascua, antes de salir del monte Sinaí e ir a la conquista de la tierra prometida es cuando celebran la Pascua y les recuerda su identidad como el pueblo que fue redimido por su Dios.\\
		La celebración de la Pascua era sumamente importante, cualquiera que despreciara dicha celebración debía de ser exlcuido del pueblo. Por otro lado, los extranjeros que se unían al pueblo y celebraban la Pascua como si fueran judíos, ellos sí podían pertenecer al pueblo de Dios.\\
		Cuando Moisés edificó el Tabernáculo se apoximó una nube sobre de él y era una señal de la aceptación del tabernáculo por parte del Señor. La nube era más que solamente un recordatorio de la presencia de Dios, era además una guía sobre la cual el pueblo sabía si debía marchar o acampar. El pueblo tenía una teocracia sacerdotal, Jehová es reconocido como el rey mientras que los sacerdotes y los levitas tenían la labor de proteger al pueblo de la ira de Dios y los sacerdotes buscaban que el pueblo siempre andara en santidad. La purificación de los levitas era necesaria antes de que funcionaran como mediadores entre Dios y el pueblo. Estos capítulo duran 50 días aproximadamente, desde el primero del mes segundo hasta que levantan marcha en Números 10:11. Era sumamente importante la organización pues el pueblo era de aproximadamente de 2 millones de personas y cada una con sus pertenencias. El pueblo estaba identificado en 4 ternas de tribus. 

		\end{subsubsection}
		\begin{subsubsection}{El pueblo desordenado (Números 10:11-14-24)}
		El pueblo seguía murmurando y manifestaban su falta de fe en Moisés y en Dios. Al encontrar las primeras dificultades tales como la falta del alimento es que el pueblo se empieza a quejar apenas con 3 días de haber salido del monte Sinaí. Se quejaban de que no tenía a su disposición toda la variedad de alimento que tenían en Egipto, reflejaban su ingratitud hacia Dios.\\
En Éxodo vimos que la respuesta de Dios siempre fue benévola mientras que vemos aquí en los capítulos 11 y 12 que Dios ya empieza a castigar a su pueblo por su murmuracioón, los castiga con fuego y con una plaga. Los propios hermanos de Moisés también empezaron a murmurar, dos líderes importantes del pueblo. Aarón era sumo sacerdote y portavoz mientras que María era líder de las mujeres.\\
El pretexto de Aarón y María era que Moisés se casó por segunda ocasión, el texto no lo narra pero se considera que Séfora ya había muerto. El problema real era que María y Aarón querían tener el mismo reconocimeinto que tenía Moisés para con Dios. Moisés al ser un hombre manso y humilde dependió completamente del plan de Dios y no se defendió de las acusaciones de sus hermanos. Dios permitió que le diera lepra a María, esta enfermedad solamente le fue quitada cuando Moisés le pidió a Dios que se la quitara.\\
El pueblo llega a Cades para Números 13, estaban ya practicamente en la entrada de Canaán. Cades era un oasis muy importante, está en la frontera de la tierra prometida, sin embargo, por falta de fe fue un lugar de tragedia. Moisés mandó a un representante de cada tribu con el propósito de que investigaran la tierra a donde debían de entrar.\\
De los 12 príncipes que enviaron los más importantes eran Caleb de la tribu de Juda y Oseas de la tribu de Efraín. A Oseas, Moisés le habría de cambiar el nombre a Josué. Josué y Caleb representarían a las tribus que serían las más importantes del pueblo de Israel. La edad de los 12 se supone que no era mayor a 40 años, Moisés los instruye para infiltrarse a la ciudad y el tipo de información que era importante que identificaran de la ciudad.\\
Después de 40 días, 10 de los 12 espías afirman que es una tierra muy próspera, la expresión ``fluía leche y miel'' es una forma de identificar la prosperidad del ganado y de la vegetación en la región. Estos 10 desaniman al pueblo de entrar a la tierra pues dicen que los habitantes eran de una gran estatura y niegan que fueran capaces de derrotarlos, los espías les tuvieron miedo por su grandeza, su fuerza y por la fortaleza de sus ciudades.\\
Caleb admite que hay enemigos fuertes en la tierra pero anima al pueblo a confiar en Dios pues Él prometió que les entregía la tierra pero el puebloo convencido por los demás espías no les hicieon caso a Josué y a Caleb y el pueblo incluso trató de matarlos. Ello provocó que Josué y Caleb se rasgaran sus vestiduras.
		\end{subsubsection}
		\begin{subsubsection}{El pueblo castigado (Números 14:11-33-49)}
			Finalmente el pueblo se niega a entrar a la tierra que Dios les había prometido, Dios se queja por su falta de fe. Ceer en Dios es aceptar y confiar en Su Palabra así como obedecerla. La verdadera fe es la semilla que lleva como fruto esa obediencia y es lo que nos va a traer la justicia de Dios. En el colmo de la rebeldía el pueblo de Israel pide mejor morir. El pueblo no puede escapar de las consecuencias de su pecado, en cada instancia la respuesta del pueblo fue retar a Dios en lugar de confiar en Él. Por ello es que Dios no permitió que esa generación entrara a la tierra prometida.\\
			Resulta irónica su petición pues Dios les conecdería lo que pidieron. En Números 14:1-2 el pueblo pide morir y Dios se los conncedió. Sin embargo, antes de morir debían de vagar por 40 años en el desierto, 1 año por cada día que salieron los espías.\\
			Todavía el pueblo en rebeldía inició la conquista por sus propios medios pero Jehová no estaba con ellos y los amalecitas los derrotan fácilmente, el capítulo 14 muestra que por resultado de pecado e incredulidad anduvieron sin sentido por 40 años. El fin de los que se niegan a creer es finalmente la muerte espiritual. No se relata completamente lo que pasó en esos 28 años de los capítulos 13-20 pero los autores bíblicos inspirados solamente relatan aquellos hechos que fueron de gran importante. \\
			En el capítulo 16 se ve otra vez la rebeldía y la reacción de Dios siempre en contra del pecado. \\
			Un levita llamado Coré cuestionó el liderazgo de Moisés y de Aarón, tomó gente  y se levantaron contra Moisés con 250 varones de los hijos de Israel. Moisés afirma que los levitas fueron demasiado lejos al afirmar autoridad para ellos que no se les había sido concedida. Como resultado Dios abrió la boca de la tierra y Coré y su gente fue tragada por la tierra. El pueblo volvió a renegar acusando a Moisés de la muerte de ellos. Dios mató a 14700 personas según Números 16:49.\\
			Dios confirma el sacerdocio de Aarón para evitar más rebeliones. Dios manda que cada tribu ponga una vara en el tabernáculo. Solamente la vara de Aarón floreció mientras que la barra de las otras tribus permanecieron secas, ésto confirmaba que Dios había elegido a los levitas para que sirvieran en el tabernáculo. \\
			En Números 18, Dios confirma las responsabilidades de Moisés y de Aarón.\\
			En el capítulo 19, Dios instruye a Moisés en un ritual de purificación. Las personas debían de ser rociadas con el agua que contenía las cenizas de la vaca sacrificada.\\
			En el capítulo 20 se narra la muerte de Aarón y María, en el capítulo 33 se da un resumen de las jornadas del peregrinaje mientras que en el versículo 38 menciona que Aarón murió a 40 años de la salida, ésto muestra que el pueblo estaba reiniciando su viaje.\\
			La muerte de María y posteriormente, de Moisés, marcaría la muerte de la generación que no habŕia de entrar a la tierra prometida.\\
			Se narra la terminación del castigo del pueblo de Dios, toda la generación incrédula ya había fallecido.\\
			Se menciona que María murió en Cades y que ahí mismo fue sepultada. A pesar de que era una nueva generación nuevamente podemos ver la rebeldía del pueblo de Israel, ahora por falta de agua. Ni Aarón ni Moisés iban a entrar a la tierra prometida debido a su incredulidad. Se menciona que no creyeron en Él y que fueron rebeldes al mandamientto de Dios y esto probablemente se refiere a cuando sacaron agua de la peña en Numeros 20:11.\\
			Parece que esta acción demuestra una falta de fe en la eficacia de la palabra de Dios, la desobediencia tiene como raíz la falta de fe. La desobediencia de Moisés revela su rebeldía. Aquí Moisés y Aarón reciben la misma sentencia por el mismo pecado.\\
			Hubo una negativa de Edom para que Israel pasara por su pueblo y por ello Dios castigó a los edomitas e Israel tuvo que rodear a Edom. Israel tuvo conflictos durante la rodeada de Edom pero con la ayuda de Dios Israel pudo tener victoria sobre los amorreos y los basamitas, éstas victorias fueron las primeras del pueblo de Israel en su camino hacia la tierra prometida. Nuevamente la actitud de queja fue ocasión de castigo para el pueblo y Dios les manda serpientes venenosas, Moisés ora a Dios y hace una serpiente de bronce de manera que si alguin fuera mordido con solo mirarla serían sanados. En $2^o$ Reyes, hablando de Ezequías habla que hizo pedazos la serpiente de bronce que hizo Moisés pues afirmaba que los hijos de Israel le quemaban incienso. Una vez que concluyó la obra de la estatua tuvo que ser destruida pues la naturaleza idólatra del pueblo provocó que la adoraran.\\
			Sigue la historia de Balaam, el pueblo de Israel tenía la necesidad de cruzar por las tierras de Balac y se le fue pedido que maldijera al pueblo de Israel. Balaam conocía a Jehova pero su avaricia le provocó que pretendiera maldecir al pueblo. Se montó en una asna y quiso ir a maldecir al pueblo de Israel pero la asna le habló ya que se negaba a obedecerlo. Entonces Balaam en lugar de maldecir a Israel lo bendijo de manera de Balac se enojó. Dios utiliza a Balaam para bendecir a su pueblo, sin embargo, él sería nefasto para la historia de Israel pues en un futuro él les recomendaría que miraran hacia otros dioses.\\
			En los capítulos 26-30, se narra un nuevo censo del pueblo de Israel para nuevamente intentar conquistar la tierra, en el cap 27 Jehová anuncia a Moisés que él también iba a morir y que tenía a Josué como sucesor.\\
			En los capítulos 28-30 se recuerda cómo es que debían ser las ofrendas.\\
			 En el capítulo 31, todo madianita debía de ser elimindado, unlcuso las mujeres. Se ve el caracter de Dios que es enemigo del pecado y que lo castiga con muerte.\newpage
			 En el capítulo 32 como resultado de las vistorias del pueblo de Isarel, las tribus de Ruben y Gad le pidieron que se quedaran con esas tierras a pesar de que todavía no estaban en la tierra prometida, la condición de Moisés era que siguieran participando en la conquista de la tierra de Canaán donde estarían las demás tribus.
		\end{subsubsection}
		\begin{subsubsection}{Pueblo reordenado}
			En el capítulo 34 se ven las instrucciones finales para entrar a la tierra prometida y en el cap 35 se ven la herencia para los levitas que estaban solamente dedicas al servicio en el tabernáculo y el sacerdocio.\\
			Cada tribu y familia recibiría una porción de la tierra que iba a ser su heredad de forma perpetua, éste es el propósito de la ley del jubileo, se toma en cuenta que habría situaciones en las que las familias vendieran su terreno por necesidad pero que cada 50 años las tierras vendidas debían de ser regresadas a las familias orginales.
			
		\end{subsubsection}
		\begin{subsubsection}{Temas clave}
			\begin{itemize}
				\item Fracaso de Israel\\
					El fracaso del pueblo y su constante ingratitud nos debe de enseñar a que confiemos en lo que Él ha dicho y obedecer sus mandamientos.
				\item El Desierto\\
					Dios muestra que es un lugar donde debían de empezar de nuevo.
			\end{itemize}	

		\end{subsubsection}
	\end{subsection}

\end{section}

%\end{document}


