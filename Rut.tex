%        File: Rut.tex
%     Created: Wed Sep 18 08:00 PM 2019 C
% Last Change: Wed Sep 18 08:00 PM 2019 C
%
%\documentclass[12pt]{article}
%\usepackage[margin=1.0in]{geometry}
%\usepackage{enumerate}
%\usepackage[spanish]{babel}
%\begin{document}
\begin{section}{Rut}
	\begin{itemize}
		\item Título\\
			Es el primer libro de la Biblia que lleva por título el nombre de una mujer y el único que lleva por título el nombre de una mujer extranjera. Esta historia comienza en el momento de una cosecha.
		\item Autor y fecha\\
			La tradición señala a Samuel como el escritor de este libro. La historia se escribe en el último periodo del reino de David pues en el capítulo final se narra la genealogía de David sin mencionar a Salomón.\\
			Como en Rut 1:1 empieza hablando de los días en los que gobernaban los jueces, probablemente era mientras gobernaba Jefté en el 1300 a.C. Nos habla de mujeres piadosas que conocían y seguían la ley de Dios en esos tiempos de idolatría y rebeldía de los jueces. Vemos que una extranjera se vuelve de los ídolos para servir y seguir al Dios verdadero. A diferencia del libro anterior es una historia bella en medio de una de las etapas más oscuras de Israel.
		\item Tema\\
			Redención. Se ilustra el papel redentor de Dios que se extiende más allá del pueblo judío.
		\item Propósito\\
			Mostrar la fidelidad de una extranjera como una coheredera de la gracia salvadora de Dios.
	\end{itemize}
	\begin{subsection}{Bosquejo}
		\begin{subsubsection}{La decisión de Rut (Rut 1)} 
Moab prostituyó a Israel a que alabara a dioses extraños y lo esclavizó por 18 años. Hubieron muchos conflictos entre ellos a pesar de que Moab era hijo de Lot sobrino de Abraham, sin embargo, parece ser que en este periodo había paz entre los pueblos.\\
En Rut 1:4 narra que la familia de Elimelec se establece por 10 años amargos en Moab. Elimelec muere en Moab de causas desconocidad dejando a su esposa y a sus dos hijos. Después los hijos de Elimelec y Noemí se casan con dos mujeres moabitas, ésta es la única vez en la que se ve una desobediencia a la ley de Moisés en este libro, ésta ley que dio Moisés en particular se encuentra en Deuteronomio 23:3-4.\\
Antes de terminar la década en la que vivieron en Moab, mueren también los jóvenes de causas desconocidas dejando a 3 viudas y siendo Rut una de ellas. Ser viuda en el lejano oriente era una tragedia pues se perdía la posición social, las viudas quedaban desprotegidas económicamente y dependían enteramente de la limosna que les daba las sociedad, reflejaban una figura de tristeza y de desolación. \\
Recuperándose de la muerte de sus esposos, realizan planes para hacerle frente a su nueva vida, pues llegan noticias de que la situación económica de Judá había mejorado, como lo relata Rut 1:6.
\newpage
Entonces Noemí toma la decisión de regresar a su tierra y les presenta fuertes argumentos para que ellas se queden en su tierra y no fueran con ella ya que no tenía algo que ofrecerlas a las jóvenes viudas. En Rut 1:8 cabe resaltar que Noemí las manda a ``casa de su madre``, que eran las que se encargaban en los asuntos de matrimonio y de sexo. No las envía a casa de su padre para que las protegiera sino que las manda a casa de su madre para que pudieran conseguir otro esposo. En Rut 1:12-13 Noeamí hace mención de la ley que tiene como propósito perpetuar el nombre del esposo que se encuentra en Deuteronomio 25:5-6 pero Noemí ya era demaisado vieja para volverse a casar. Noemí convence a Orfa que bese a su suegra y se volviera  a su pueblo y a sus dioses, sin embargo, Rut toma la decisión de seguir a su suegra y abandonar el lugar de su nacimiento y los dioses de su nación. Rut la amó con un amor poco usual, demostrado en Rut 1:16-17. Rut siendo viuda se convierte en una verdadera hija de la tribu de Judá y regresa junto con su suegra a Belén.
		\end{subsubsection}
		\begin{subsubsection}{La fidelidad de Rut (Rut 2)}
				Rut toma el compromiso de trabajar para sustentar a las 2. Trabaja en la actividad más humilde que era recogiendo las sobras de la cosecha. Dios la llevó a un campo que era de un pariente cercano de Noemí. Dentro de la ley de Moisés, en Levítico 19:9-10, se especifica lo que aquí se lleva a acabo. Esta ley era para proteger al pobre y al menesteroso que era el caso de Rut y Noemí, aunque era un acto de misericordia significaba un trabajo muy fuerte para el pobre que lo recogía. \\
				Providencialmente Rut conoce en este trabajo al dueño de esta aprte del campo, un hombre de Dios llamado Booz. Cuando Booz le llama por primera vez a Rut le llama ''hija mía``. Booz aparentemente no era casado o era viudo pues en Rut 2:20 Noemí lo reconoce como uno de los redentores de su familia..\\
				Noemí empieza a ver la gracia y la misericordia de Dios en sus vidas el cual puede llevar la redención que ellas estaban buscando. La palabra redención en el hebreo ''gaal`` que significa que un redentor era un miembro que puede ayudar a a recuperar las pérdidas de la familia, por ejemplo, una pérdida humana (podía cobrar venganza). También era redentor si recuperaba la libertad de un pariente esclavo y era quien recobraba el nombre de un familiar mediante la ley del levirato.\\
				Dios es por excelencia un redentor para quien Él ha elegido. Cristo redime a los hombres al pagar un precio por nsootros quienes somos librados de la esclavitud del pecado. 
		\end{subsubsection}
		\begin{subsubsection}{La demanda de Rut (Rut 3)}
			Noemí instruye a Rut sobre lo que tenía que hacer pues ella lo sabía muy bien. El trabajo se hacía al aire libre en un suelo duro y al atardecer aventaban el grano al aire mezclado con la paja para que el aire separara la paja del grano, se hacía hasta muy noche. La persona que hacía el trabajao debía de dormir ahí mismo pues también debía de cuidar el grano de ladrones. Noemí prepara a Rut para que manifestara el deseo de que Booz la redimiera y en Rut 3:3 Noemí la engalana con sus mejores vestidos y perfumes.
			\newpage
			Noemí instruye a Rut a que le destapara los pies una vez que él ya había dormido y se acostara ahí con el propósito de mostrar su deseo de rención, ésto no era una insinuación sino una muestra de sumisión. Booz se dio cuenta después de mucho tiempo que había una mujer recostada sobre el borde de sus pies, relatado en Rut 3:8-10. En Rut 3:10 Booz aprecia las virtudes de Rut pues reconoció que lo hacía por el bien de su suegra, él vió su calidad moral. Ésto era muestra de la sumisión que Rut tenía a la ley de Dios. En Rut 3:12 se menciona que sólo había un problema pues había un pariente más cercano que debía de responder a dicha redención. \\
			Noemí esperó confiada sabiendo que Booz resolvería el problema.
		\end{subsubsection}
		\begin{subsubsection}{La redención de Rut (Rut 4)}
			Booz fue a las puertas de la ciudad ya que estaban abiertas solamente en el día y era un lugar perfecto para llevar a cabo todo tipo de transacciones. En Rut 4:2 Booz encuentra al pariente del cual se desconoce su nombre y Booz toma a 10 varones (ancianos) como testigos. Ésto es muestra de que Booz era conocedor de las leyes y las costumbres del pueblo hebreo y que también las respetaba.\\
			Inicialmente el pariente acepta redimir la tierra pero cuando Booz le menciona que el trato involucra también redimir a Rut entonces él decide ceder su derecho a Booz. El rito de quitarse la sandalía en Rut 4:7-8 provenía precisamente de Deuteronomio 25:8-10. El pariente podía transferir legalmente su derecho a otro y esto se simbolizaba entrgándole su sandalía, representando su renuncia a su derecho.\\
			En Rut 4:11-12 se narra como todos los testigos bendijeron a Booz y él recibe bendición del todos los presentes. En Rut 4:13-16, Rut queda embarazada como una bendición con la que Dios recompensó a Noemí.\\
			En aquél entonces 7 hijos era muestra de la plenitud de Dios, el nombre del hjo fue Obed que sería el padre de Isaí, padre de David, el rey más glorioso de Israel.\\
			Abraham (Génesis 11:26) engendra a Judá (Génesis 29:35), quien engengra a Salmón (Rut 4:20) quien engendra a Booz (Rut 4:21). Por otra parte, Lot (Génesis 11:27) engendra a Moab (Génesis 19:37), quien engendró a Rut (Rut 4:13). De Rut y de Booz nace Obed (Rut 4:17), quien engendró a Isaí (Rut 4:22), quien engendró a David (Rut 4:22). Rut ejemplifica que no se hereda por sangre ni por nacimiento sino a la perosna que vive según la voluntad de Dios siguiendo y odedeciendo la ley por medio de la fe. \\
			La redención es para toda la humanidad, no solamente para el pueblo judío. Rut originaria de un pueblo enemigo fue redimida. Se demuestra que hasta en el NT vemos que las mujeres también son coherederas de la gracia pues antes se pensaba que solamente los hombres podían recibir dicha salvación. Dios en su soberanía protegió a estas dos mujeres. También vemos a una mujer aparentemente indigna por no ser de Israel que está dentro de la genealogía de Jesús. Dios la exaltó mostrando que Él conoce los corazones y los pensamiento de las personas.
		\end{subsubsection}
	\end{subsection}
\end{section}
%\end{document}


