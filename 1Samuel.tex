%        File: 1Samuel.tex
%     Created: Mon Sep 23 07:00 PM 2019 C
% Last Change: Mon Sep 23 07:00 PM 2019 C
%
%\documentclass[12pt]{article}
%\usepackage[margin=1.0in]{geometry}
%\usepackage{enumerate}
%\usepackage[spanish]{babel}
%
%\begin{document}
\begin{section}{Primer libro de Samuel}
	\begin{itemize}
		\item Título\\
			Le es dado el nombre de Samuel a pesar de que en ambos libros se mencionan de manera importante a Saúl y a David. Se piensa tradicionalmente que el escritor es anónimo. Se sigue narrando cronológicamente la historia de Israel y se ve el paso del gobierno teocrático a la monarquía precisamente porque Samuel fue el último antes de la imposición de la monarquía, a Samuel se le considera como el último de los jueces. El libro empieza con el nacimiento de Samuel y a la mitad narra de manera muy breve su muerte.
		\item Autor y fecha\\
			Fue escrito aproximadamente en los tiempos durante el reinado de Salomón. El primer libro no es muy citado pero no debe de ser dudado, en Mateo 12:3-4 hay una alución acerca de lo narrado en el libro y eso se debe de tomar como autentificación.\\
			Así mismo, en Hechos 13:20-22 se hace un comentario acerca de Samuel, Saul y David de eventos que vienen narrados en el libro. Siendo Samuel el último juez debió de haber sido contemporáneo de Sansón y seguramente ellos se conocieron. Se ve con mayor claridad el que Sansón al final de su vida combatió con los filisteos y aún al principio del libro se sigue narrando el conflicto con ellos.
		\item Tema\\
			El pueblo le pide a Samuel que les imponga un rey y aunque Samuel los intentó disuadir de ello, el pueblo de Israel estaba resuelto a obtener un rey así como los demás pueblos.
		\item Propósito\\
			Establecimiento de un trono a través de un hombre y del pacto davídico.
	\end{itemize}
	\begin{subsection}{Bosquejo}
		\begin{subsubsection}{Samuel}
			\begin{enumerate}
				\item El profeta ($1^{o}$ Samuel 1-4)\\
					Samuel fue en primera instancia un profeta. En $1^{o}$ Samuel 1 se empieza la narración con su nacimiento. Su madre Ana era estéril y compartía a su esposo con otra mujer. A diferencia de Ana, Penina la concubina no era estéril y le había dado descendencia a Elcana.\\
 La principal causa de la poligamia que vemos en el A.T. consiste en la gran importancia que le tenían al tener descendencia y por ello incluso las mujeres aceptaban que los esposos tuvieran concubinas con el propósito de conseguir descendencia. El ser estéril era considerado como una maldición, además, era normal que fueran necesarios los hijos (masculinos) para que le ayudaran al padre en sus tareas. Elcana era descendiente de Leví, mientras que Leví fue descendiente de Aarón por lo cual, Elcana tenía todo el derecho de ser sacerdote. Por lo mismo, Samuel tenía el derecho de ser sacerdote.
 \newpage
 A pesar de que en la Biblia se le menciona a Elcana como hijo de Efraín en realidad era descendendiente de Leví pues los levitas tuvieron descendientes en varias tribus. Era efrate pues a pesar de haber nacido en territorio de Efraín era descendiente de Leví.\\
 En $1^{o}$ Samuel 1:3, siguiendo la tradición judía, todos los años subían a adorar en Silo, allí había un sacerdote llamado Elí siendo encargado de los sacrificios y Elí ya era un viejo descendiente de Itamar (hijo menor de abrham). Ana se sentía trastornada pues la concubina la alteraba pero Elí pensaba que estaba ebria cuando la veía hablar a solas. Ana suplicaba con tanta intensidad pues ya no quería ser considerada como deshonra por ser estéril.\\ \\
 En $1^{o}$ Samuel 1:15 Ana le cuenta su pena a Elí y le dice que si Dios le concede un hijo se lo dedicaría a Él. Finalmente tiene un hijo por regalo de Dios y le llama Samuel. En el tiempo en el que Samuel nace se iba imponiendo el pueblo asirio que estaba dominando todo el territorio en asia occidental e incluso Egipto, era un gran imperio. Tal y como Ana había prometido, una vez que Samuel es destetado se lo dedica a Dios, ésto se narra en $1^{o}$ Samuel 1:26-28. Por ello, también a Samuel se le considera nazareo (como Sansón) que en el hebreo quiere decir ``dedicado''. En Números 6:1-6 se ven los detalles de eso y aunque el voto es regularmente temporal, ella dedicó todos los días de Samuel a Dios. La navaja no debía de tocar la cabeza de Samuel como un símbolo de su dedicación a Dios y tampoco debía de rasurarse. Además se abstenían de cualquier tipo de tentación, en especial el vino. Debían de tener una pureza total espiritual y físicamente. Cuando nace Samuel en el periodo de los jueces, el pueblo tambíen tenía conflictos con los amonitas pero Dios levantó a Sansón y a Samuel para que los libraran.\\ \\
 En el capítulo 2, Ana hace una oración que incluso tiene carácter de profecía como se narra en $1^{o}$ Samuel 2:10 pues se da una referencia clara al Mesías que finalmente sabemos que se refería a Jesús. Dentro de la sabiduría que Dios le dio, ella sabía que iba a haber un ungido  que los iba a salvar. La exaltación al nombre de Dios por parte de Ana va más allá de su propio hijo llegando hasta a referirse al salvador. De la palabra ungido es de donde viene la palabra Mesías que es de donde viene ``Cristo'' . Ana es intérprete de esa expectativa del pueblo por tener a un Mesías. La lectura del canto de Ana es parte de la celebración del año nuevo judío. Constituye la lectura de los profetas con la esperaznza del ungido de Jehová que llegaría a liberar a su pueblo.\\
 Tanto en su niñez como en su juventud, Samuel es vestido como sacerdote con el efod. Paralelamente a la vida de santidad que estaba viviendo Samuel se ve la iniquidad de los hijos de Elí. Ellos no estando satisfechos de las porciones de los sacrificios tomaban todo lo que ellos podían, no pedían permiso, ellos mismos se servían de manera irrepetuosa y no seguían las instrucciones de Dios acerca de la manera correcta para hacer el sacrificio.\\
Elí de alguna forma sí trató de controlar la impiedad de sus hijos pero ellos no le hacían caso. Los hijos también adulteraban y su manera de hablar era mala, ellos con su ejemplo hacían pesar al pueblo. 
\newpage
Bíblicamente y de acuerdo a la ley debían de ser condenados a muerte, pero Elí solamente les advertía que Dios les iba a juzgar. Vemos cómo es que Dios vuelve a comunicarse con Su pueblo por medio de Samuel y las primeras palabras que le dice Dios a Samuel son de condenación para Elí, narradas en $1^{o}$ Samuel 3:11-14. Fue entonces que Samuel fue conocido por el pueblo como un profeta de Dios de acuerdo a $1^{o}$ Samuel 3:19-20.\\
En el capítulo 4 se empieza a narrar un acontecimiento que dura 3 capítulos, el pueblo estando en aprietos se lleva el arca a la batalla, narrado en $1^{o}$ Samuel 4:3. Los filisteos los estaban atacando a Israel y llevan el arca esperando que Dios les diera la victoria sin importar su situación espiritual. Su conflicto con los filisteos ya había durado más de 100 años, desde que Samgar mató a los filisteos con la quijada de un buey. Los filisteos tuvieron miedo del arca cuando la ven llegar pero se dan cuenta de que no sirve como un talismán (amuleto) que le diera la victoria a Israel, así es como Israel fue vencido por los filisteos. En su pecado, ellos mismos obligaron a Dios a apartarse de ellos, ese día se cumplió lo que dijo Dios acerca de los hijos de Elí pues ambos murieron. \\
Las noticias llegan a Silo y un benjamita llega a la casa de Elí con los vestidos rasgados diciédole que habían muerto 30000 israelitas y entre ellos estaban sus hijos pero cuando se enteró que capturaron el arca, Elí cae al suelo, se pega en la cabeza y muere.\\
 
				\item Samuel, el juez ($1^{o}$ Samuel 5-8)\\
					Vemos a Dios actuar pues primero hace postrar a Dagón al arca cuando los filisteos la llevan a Asdpd. Siete meses duró el arca en poder de los filisteos donde Dios les manda muchas plagas y mueren muchos de ellos. Por ello los filisteos se ven obligados a regresar el arca a Israel mientras seguían sometiendo a los israelitas. Después de ésto es que Samuel es levantado como líder del pueblo de Israel. \\
					Elí había soportado la idolatría del pueblo pero Samuel los exhorta a dejar su idolatría. En $1^{o}$ Samuel 7:13 dice que la mano de Dios estuvo en contra de los filisteos mientras Samuel lidereó al pueblo. Samuel se equivoca al poner como jueces a sus hijos pues los jueces eran directamente llamados por Dios, ésto lo hace en $1^{o}$ Samuel 8:1-2. Sus hijos no eran buenos y fueron acusados de aceptar sobornos, contratar esposas y ser amantes del dinero. No sabemos si Samuel los reprendió porque no lo dice el texto pero fueron destituidos de su cargo de jueces y es ahí donde el pueblo le pide a Samuel un rey. El problema principal era que Samuel ya era viejo y que sus hijos acababan de ser destituidos, el deseo de tener un rey desagradó a Samuel y a Dios pues Él siempre quizo que Su pueblo fuera diferente y no que fueran igual a los demás pueblos que eran gobernados por reyes. El pueblo pensaba que sus problemas eran de carácter político pero Samuel sabía que sus problemas eran espirituales.\\
					El mensaje del libro queda claro en $1^{o}$ Samuel 15:22 pues siempre la obediencia es lo que Dios ha requerido de su pueblo. En la actualidad, en las iglesias hay gente que parece ser fiel en asistir y en servir pero que no son fieles en obedecer la Palabra de Dios.
					\newpage
					Dios accede y prepara todo para que tengan un rey que fuera precursor del rey mesiánico, sin embargo, era importante que el pueblo obedeciera a Dios por encima del rey. El pueblo en realidad estaba rechazando a aquél al que debían de servir, estaban despreciando y tomando ligeramente Su ley. El pecado perpetuo de Israel era lo que los separaba de Dios. Dios le dijo a Samuel que oyera la voz del pueblo pues habían rechazado a Dios y no a Samuel. Dios le dice a Samuel que les advierta sobre la imperfección de la justicia de su rey pues su rey tomaría a sus hijos como siervos y soldados, los ocuparía como peones, las mujeres servirían en la cocina del rey y se podía apoderar de la tierra que él quisiera pues así funcionaba la monarquía.

			\end{enumerate}
		\end{subsubsection}
		\begin{subsubsection}{Saúl ($1^{o}$ Samuel 9-15)}
			\begin{enumerate}
				\item Saúl el rey ($1^{o}$ Samuel 9-12)\\
			Saúl era un hombre con todo atractivo físico, su nombre literlamente quiere decir pedido o deseado y probablemente era primogénito. Tenía una buena impresión y gracia que adornaba su persona tan impresionante que se destacaba en su pueblo de acuerdo a lo que dice $1^{o}$ Samuel 9:2.\\
			En el capítulo 10 vemos a Samuel ungiendo como rey de Israel a Saúl por instrucciones de Dios, en específico en $1^{o}$ Samuel 10:1-2. Saúl empezó a profetisar después de la unción y en $1^{o}$ Samuel 10:24 es presentado como el escogido de Dios.Por primera vez en la historia del pueblo se les escucha clamar ``Viva el rey''. Las funciones de rey las asume en su totalidad Saúl hasta el capítulo 11 cuando los amonitas están atacando a Jabes de Galaad al sur del lago de Galilea y aunque los amonitas habían sitiado a Jabes requerían de mucho tiempo para poder vencerla y es por ello que los amonitas acceden al trato de Jabes. Amón tenía la seguridad de que tenían bien controlada la ciudad pues no tenían quién les ayudara pero el espíritu de Dios se apodera de Saúl y le da la capacidad de salvar a su pueblo. Saúl tiene coraje por medio de Dios y derrota a los amonitas, en el capítulo 12 procalama que ya hay un reino con ejército y con este hecho se da por terminada la etapa de los jueces para el pueblo. \\
			En el capítulo 12, Samuel se dirige al pueblo diciéndoles que había escuchado la voz del pueblo y que ya era viejo. Les dice que atestigüen delante de Dios y el rey si acaso había agraviado a alguien o si es que había calumniado a alguien pero el pueblo le asegura que eso nunca había pasado. Los exhorta a seguir a Dios con la myor importancia pues de lo contrario la mano de Dios estaría en su contra. Saúl hasta ese momento había obedecido todo lo que le había dicho Dios.
		\item Saúl rechazado por Dios ($1^{o}$ Samuel 13-15)\\
			Saúl se precipita pues no estaba Samuel y ofrece sacrificios a pesar de que él no era la persona que debía de hacer dicha acción. Samuel llega y lo reprende pues lo que hizo desagradó a Dios de acuerdo a $1^{o}$ Samuel 13:13-14 y aparece por primera vez el nombre de Jonatán, hijo de Saúl pues con gran valentía él es el que pelea con los filisteos y los vence demostrando que Dios está con ellos y mostrándole a Saúl que lo único necesario era que creyera.
\newpage
			En el capítulo 15 Saúl vuelve a desobedecer cuando Samuel por instrucciones de Dios lo manda a destruir a los amalecitas, Saúl tenía la orden de destruir todo lo que tuviera vida de los amalectias que incluía a toda la población y hasta los animales. Saúl destruye a la mayor parte de los amalecitas pero al rey le perdona la vida y tampoco mata a los animales sino que los toma como botín. Samuel le dice a Saúl que Dios ya lo había desechado en $1^{o}$ Samuel 15:22-23.\\
			En el $1^{o}$ Samuel 15:35, Samuel nunca más vió a Saúl aunque lloraba por la suerte de Saúl a quien había ungido. Se ve que una simple disculpa por el pecado no es algo que sirva ya que es necesario un verdadero arrepentimiento sobre las acciones que ofenden a Dios.
	\end{enumerate}
		\end{subsubsection}
		\begin{subsubsection}{David ($1^{o}$ Samuel 17-31)}
			\begin{enumerate}
				\item David el ungido (1o Samuel 16)\\
					David es el nombre que más se menciona en toda la Biblia con 1118 veces que se menciona. No hay otro que represente tan típicamente a Cristo como David. Es tan importante que es por ello que al Mesías le llaman hijo de David.\\
					Después de la desobediencia de Saúl, obviamente él se opone al nombramiento de otro rey, el ungimiento de otro rey se considerearía como traición y seguramente era algo que temía Samuel, sin embargo, Saúl siempre respetó a Samuel y hasta le temía. Dios le dice a Samuel que llevara una novilla a sacrififcar y con ese pretexto iría a Belén a llevar a cabo su misión.\\
					Dios manda a Samuel a la casa de Isaí para que ungiera a uno de sus hijos y mientras Samuel buscaba a alguien semejante a Saúl en apariencia física y fortaleza pero Dios solamente mira el corazón del hombre, como lo dijo el Señor en $1^{o}$ Samuel 16:6-7. \\
					Después de ver a los hijos de Isaí y que ninguno fuera escogido, en $1^{o}$ Samuel le pregunta que si ya eran todos sus hijos y fue allí cuando Isaí le menciona al menor que estaba pastoreando a las ovejas. Ve Samuel a un joven rubio y Dios le dice a David que ese joven era a quien debía de ungir y lo unge frente a sus hermanos.\\
					David, bisnieto de Booz y de Rut de la tribu de Judá es a quien ungió Samuel, tal y como lo narra $1^{o}$ Samuel 16:13. La unción era la forma en la que vemos por la forma que Dios se introduce en la persona para que obre como Dios quiera. Es importante recalcar que el Espíritu Santo no tenía permanencia en las personas. \\
					Saúl vivía atormentado por un espíritu ya que no tenía el Espíritu de Dios. A Saúl le envían un joven para que le tocara música para calmarlo y es ahí donde le mandan a David pues también era músico, ésto lo vemos en $1^{o}$ Samuel 16:23. Saúl le toma cariño a David sin saber que era la persona que había de quedarse con su reino, así Saúl lo hace su paje de armas.
				\item David el héroe ($1^{o}$ Samuel 17-31)\\
					Sin conocimiento de la milica, David derrota a un gigantesco soldado de los filisteos. Una vez más Israel estaba a punto de iniciar una confrontación con los filisteos. Ellos ponen al gigante como su campeón de batalla, el gigante se llamaba Goliat que significa ``hombre en medio''.
					\newpage
					Los israelitas debían de esoger a un hombre que pelearía contra él y así el ganador determinaría qué pueblo ganaba. Nadie del pueblo de Israel era capaz de enfrentarle pero llegó David a vencerlo. David le pregunta al ejército de Israel qué se podía hacer para vencer al gigante con el propósito de animarlos y motivarlos a enfrentarse al gigante pues estaba insultando al Dios omnipotente y David estaba molesto. David se ofrece entonces a enfrentarlo al ver que nadie más lo hacía. Se siente indignado por el temor de su pueblo pero él confiaba en Dios.\\
					En este capítulo 17 se narra cómo fue la lucha entre el joven pastor y el poderoso militar enemigo, lo más destacado es la confianza que tuvo David en que él recibiría la ayuda de Dos, la misma confianza que tuvo Josué antes de ir a conquistar la tierra prometida. Tienen una breve plática antes del altercado y David aclara que el que ha sido ofendido es Dios y Él es el que pelearía, no sería una pelea entre dos hombres ni dos ejércitos sino una pelea entre las fuerzas de las tinieblas y de la luz.\\ \\
					Parecería insignificante el instrumento de guerra de David ya que llevaba una honda pero él sabía perfectamente de lo que era capaz. En $1^{o}$ Samuel 17:49 vemos que la piedra que lanzó quedó clavada en la frente de Goliat y lo mata sin tener David espada. La fuerza que acompañó a David era la fuerza de Dios y ésto lo sabía David. Éste suceso le trae un gran reconocimiento a David, él corrió se puso sobre el flisteo y le cortó la cabeza con su espada y cuando vieron ésto los filisteos huyeron.\\
					Por esta hazaña es que David es reconocido como héroe nacional. David era un hombre del campo que defendía a sus ovejas con una honda e incluso mataba a las fieras que las atacaban. Por el reconocimiento del pueblo a David es que Saúl empieza a tener celos de David y por ello es que Saúl sintió odio por David. David nunca lo odió pues Saúl era ungido de Dios y David lo respetaba mucho. Jonatán hizo un pacto de amistad con David en $1^{o}$ Samuel 18:3-5 que David siempre respetó.\\ \\
					En dos ocasiones Saúl trata de matar a David con una lanza y mientras más se engrandecía David más envidia y odio le tenía Saúl. Tanto Jonatán como la hija de Saúl siempre protegieron a David en contra de su padre.\\
					Jonatán confiesa y acepta que David sería el sucesor de Saúl pues Dios lo había determinado así y Jonatán le cede todos sus instrumentos por ello. David termina huyendo, narrado en $1^{o}$ Samuel 20:42. En la persecución de David, él tuvo múltiples oportunidades para matar a Saúl pero nunca lo hizo porque era el ungido de Dios, tal y como lo expresa en $1^{o}$ Samuel 26:11-12.\\ \\
					David confiaba que la justicia sería de parte de Dios. En el capítulo 25 se narra la muerte de Samuel y el conflicto entre David y Nabal, un hombre rico. Nabal menosprecia a David, se niega a proporcionarle ayuda y aparece su esposa Abigail que no permite que David se vengara con su familia. Dios tenía determinado que Nabal moriría días despúes, sin embargo, antes de su muerte en $1^{o}$ Samuel 25:23-25 se narra que Abigaíl inteligentemente y en una muestra de humildad va a David y se le muestra como sierva.
					\newpage
					David admirado por ello la toma como esposa y a pesar de ser un hombre conforme al corazón de Dios tomó múltiples esposas. David huyó a refugiarse de nuevo y ahora los filisteos piensan que David puede ser enemigo de Saúl al igual que ellos. Ellos piensan que David estaría de su lado para luchar contra su enemigo en común. Al llegar con 600 hombres de guerra al territorio de los filisteos se muestra que David ya tenía un pequeño ejército. David ya estaba del lado del ejército enemigo y se narra que incluso estaba en la retaguardia del pueblo pero Dios impide que los ataque.\\
					Del otro lado de la línea de combate se encontraba el pueblo de Israel lidereado por Saúl y al ver el tamaño del ejército filisteo Saúl se atemorizó e intento consultar a Dios en sueños por medio de sacerdotes y por profetas pero Dios no le contesa. Saúl consulta a una adivina, algo prohibido, en su desesperación y le ordena que invocara a Samuel, se creía que el Seol estaba debajo y por ello era que el espíritu debía de subir. Saúl le jura a la mujer por Dios que no le haría daño.\\
					La adivina vio a un ser divino que subía y Saúl lo identifica como Samuel por su manto. El espíritu habla directamente a Saúl.\\
					En el capítulo 31 se narra la pelea entre Israel y los filisteos y al ser derrotado el pueblo de Israel mueren los hijos de Saúl incluyendo a Jonatán. El ejército de Israel retrocede y al fin rodeado por los enemigos es alcanzado por una flecha. Saúl al saber la posible tortura que sufriría le pide a su escudero que él lo matara pero el escudero no lo quiere matar y por ello Saúl se suicida. En $1^{o}$ Samuel 31:4 también se menciona que Saúl se quita la vida con su espada. Los filisteos al encontrar el cuerpo de Saúl le cortan la cabeza, se llevan sus armas y cuelgan su cuerpo en un muro. Los habitantes de Jabes rescatan los restos del rey para darle una honrosa sepultura cuando se enteraron de lo que pasó. Cruzaron el Jordán y marcharon por más de 20 kms. Además se llevan los cuerpos de sus hijos y todos los cuerpos los incineran. Seguramente recordaban con aprecio a Saúl pues la primera acción como rey de Israel fue rescatar a Jabes de los filisteos, esta acción fue un agradecimiento que tenían para el rey. Saúl hubiera sido un gran rey pero su desobediencia le quito el favor de Dios.
			\end{enumerate}
		\end{subsubsection}
	\end{subsection}

\end{section}
%	\end{document}


