%        File: Levitico.tex
%     Created: Wed Sep 04 08:00 PM 2019 C
% Last Change: Wed Sep 04 08:00 PM 2019 C
%
%\documentclass{article}
%\usepackage{enumerate}
%\usepackage[spanish]{babel}
%\begin{document}
\begin{section}{Levítico}

\begin{itemize}
	\item Título\\
	Se le nombró ``Leutikon'' al establecer la Septuaginta que significa ``asuntos de los levitas'' sin embargo en la Biblia hebrea lleva el título que traducido es ``Y él llamó\ldots''
	\item Autor y fecha\\
		Aparecen más de 50 veces que Dios entregó a Moisés los mandamientos por ello no queda duda que Moisés fue quién lo escribió. No se registra ningún movimiento del pueblo de Israel mientras siguen al pie del Monte Sinaí. Dios había llamado a su pueblo a ser santo. Este pueblo debía de santificar a Dios y cumplir con la ley que ya se habían comprometido a cumplir.\\
		Dios enfatiza que Él es santo o que santo es su nombre, un atributo de suma importancia en toda la Biblia. La santidad y todo lo santo deriva de Él. Al mismo tiempo se manifiesta al hombre de esa manera e invita al hombre en que participe en Su santidad. Se puede apreciar el origen divino del libro pues tiene un principio profético pues Dios dice que el pueblo lo iba a desobedecer de manera que traería juicio sobre ellos. Más delante también habla de restauración.
	\item Tema\\
		Santidad y adoración a Dios.
	\item Propósito\\
		Comunión de Dios con su pueblo.

\end{itemize}
\begin{subsection}{Bosquejo}
	\begin{subsubsection}{Sacrificios}
		La santidad que requiere Dios para la adoración es la misma para cualquier persona. Todo servicio y adoración a Dios es exclusivamente para darle la gloria. Provee una descripción detallada de las ofrendas y de los sacrificios que Dios pidió. Se observan 5 distintos tipos de sacrificios:
		\begin{itemize}
			\item El holocausto\\
				No se permitía que tuviera deformidades el animal, debía de ser completamente consumido por el fuego. Era completamente voluntario de parte de una persona que quería expresar su consagración a Dios, el animal debía de ser un animal doméstico, vacuno u ovejuno, ésto lo estableció Dios para diferenciarlo con los distintos sacrficios paganos quienes sacrificaban todo tipo de animales.\\
				El libro de Hebreos dice claramente que estos sacrificios no fueron totales para limpieza. El sistema repetitivo de sacrificios fue sustituido por el sacrificio de Jesús que fue una sola vez y para siempre.\newpage
				El tipo de animal ofrendado también tenía que ver con la capacidad económica del ofrendante pero siempre un macho. En Levítico 1:14 incluso se menciona que también podía ser un ave.
			\item La oblación u ofrenda vegetal (Levítico 2)\\
				Es una ofrenda sin sangre que también le era agradable a Dios. Había 3 variaciones, flor de harina sin cocer, flor de harina cocida o grano tostado de la cosecha. La harina debía de ser mólida finamente, debería de pasar por un tamiz en 2 ocasiones para garantizar que fuera perfectamente molida, esta perfección sustituye la perfección física que debía de tener los animales. En el Nuevo Testamento se aclara que también éstos sacrificios está relacionados con el sacrificio de Cristo en su perfección, el fuego representa la prueba de su sufrimiento mientras que el incienso es la esencia agradable que presentaba la vida santa de Jesús para Dios.
			\item El sacrificio de paz (Levíitico 3)\\
				Simbolizaba la paz y la comunión entre Dios y el ofrendante, representa una reconciliación. Requería que fuera de ganado vacuno, macho o hembra pero perfecto donde las faltas del ofrendante pasaban al animal. En Levítico 3:11 se aclara lo que era una vianda, el sacrificio referenciaba un momento en el que iba a haber paz entre el ofrendante y Dios pues la costumbre era que los pactos eran sellados con una comida, así,  se comía el animal después del sacrificio como un indicio de una vianda entre Dios y el ofrendante.
			\item El sacrificio por el pecado (Levítico 4)\\
				Este sí era obligatorio. Alguien que tuviera pecado en su conciencia tenía por obligación el presentar este sacrificio, es parecido a los demás con la condición de que era obligatorio, incluso el sacerdote lo presentaba con el propósito de entrar limpio a la presencia de Dios.
			\item El sacrificio por la culpa (Levítico 5:14-6:7)\\
				Era un pecado cometido por ignorancia en el cual sí era posible obtener una restitución a la persona que había sido perjudicada por el pecado ``accidental''.


		\end{itemize}
	\end{subsubsection}
	\begin{subsubsection}{Sacerdocio}
		Específicamente recae sobre la descendencia de Aaron, tribu de Leví. Los sacerdotes actuaron como mediadores entre Dios y los hombres, siempre así se mantenía la comunión entre Dios y el pueblo. Posteriormente, la comunicación fue por medio de los profetas. En el capítulo 8 se describe el llamamiento que hace Dios al sacerdocio con detalles del ritual y de la vestimenta. En el capítulo 9 vemos las funciones del sumo sacerdote recién ordenado que incluso hace sacrificios por sí mismo.\newpage
		La manifestación visible de la aceptación de dichos sacrificos era el fuego consumidor pues era una clara evidencia de la relación directa que tenía Dios con su pueblo.\\
		\underline{Dificultades de interpretación}\\
Hermenéuticamente no podemos saber a ciencia cierta por qué es que Dios reconoce el sacrifico de Nadab y Abiú como fuego extraño.\\
Lo que sí se pude interpretar es que el juicio de Dios fue porque había culpa, no se sabe cuál pero existía. Dios debía dejar en claro que sus peticiones eran delicadas y no podía ser condescendiente de forma que manchara su santidad y la de su pueblo.
	\end{subsubsection}
	\begin{subsubsection}{Santidad}
		Dios quería que su pueblo se distinguiera de los demás por sus costumbres. La iglesia está bajo la dispensación cristiana y muchas de estas restricciones ya no son aplicadas a la iglesia tal como le fue confirmado a Pedro en Hechos 10.\\
		En el capítulo 12 se dan restricciones a la maternidad y de la condición de la mujer después del parto, la circuncisión también queda como una señal y como parte de la ley mosaica. Se habla de la lepra que no es el mismo tipo de lepra que conocemos pues eran todo tipo de infecciones de piel. En el capítulo 14 se habla de la recuperación de un enfermo que además de ser sanado físicamente, también era sanado espiritualmente con el perdón de sus pecados. En el capítulo 15 se mencionan las impurezas sexuales para hombres y para mujeres. En el capítulo 17 se menciona la importancia de que todo sacrificio sea en el tabernáculo pues la sangre era sagrada y el significado de la vida.\\~\\
		En toda la escritura se prohibía comer sangre animal y derramar sangre humana. Se relatan las costumbre de los pueblos paganos y Dios hace un consante llamado a la santidad (separación) de su pueblo.\\
		El séptimo día era considerado como una fiesta religiosa, se habla acerca de la Pascua, los panes sin levadura, el pentecostés, las trompetas, la expiación y la fiesta de los tabernáculos.\\
		Las comunidades cristianas siguieron dejando un día para el señor pero distinto al día de reposo del pueblo hebreo y se utilizó el domingo recordando la resurreción de Jesús. El libro finaliza en el capítulo 24 hablando acerca de la blasfemia. Capítulo 25 jubileo y finalmente el capítulo 27 habla sobre los votos con respecto a los siervos, los animales, las cosas o las tierras.
	\end{subsubsection}

\end{subsection}
\end{section}
%\end{document}


