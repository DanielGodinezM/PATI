%        File: Ester.tex
%     Created: Wed Oct 09 08:00 PM 2019 C
% Last Change: Wed Oct 09 08:00 PM 2019 C
%
\documentclass[12pt]{article}
\usepackage[margin=1.0in]{geometry}
\usepackage{enumerate}
\usepackage[spanish]{babel}
\begin{document}
\begin{section}{Ester}
	\begin{itemize}
		\item Título\\
			El propósito es de ver que Dios sigue protegiendo a sue pueblo entodo momento. no podmeos ver ne ets elibro ninguna enseñanza religiosa. No e se puede ver ningua doctrina, tema teológico ni siquiera ver algo del gobierno judaico. sin embargo, es un libro que se le consiedera inspirado y toma un lugar my importante en el pueblo judío.
		\item Autor y fecha\\
			No es identificado. No hay evidencia de que haya sido Mardoqueo pero sínde bío de haber sido alguien que debió de haber conocido las tradiciones judías y que conocía también el pueblo Persa. Ésta libro cierra los acontecimientos históricos en el aT.\\
			Estos acontecimeintso que se narran en el libro ocurrieron en el periodo hsitrico entre Esdras 6 y 7 pues los personajes todavía estaban en cautividad. Josefo lo cosnidera nun libro histórico y sagrado. ester fue incluido como canonc en el concilio de jania en el 90 ad cuando se autorizó todo el canon judío. Los padres de la iglesi primitiva tenían sentimientos encontrados acerca del contenido des este libro. EN el concilio de cartago tambińe se consideró este libro com canon y aun así los comentaristas lo han desechado. tuvieron que pasar 300 años para que alguien escribiera un comentario del libro. Lutero y Calvino no consideraban que este libro fuera importante pero se le sigue considerando como Palabra de Dios.\\
			es un libro atractivo para los judío pues muestra coómo se debe de comportar el judío en terreno extranjero dado que en la ctualidad muchos judio viven en el extranjero. en las escrtituras y en la hsitoria es muy común que los judío vivan en otros países con gobernantes no judíos. El verdadero poder no está en el palacio de los goberantes extranjeros ni en la suert euq epuedan tener, el pdoer para que el pueblo de Dios pueda seguir subsisteiendo depende solamente de Dios.
		\item Tema\\
			Preservacion por parte de Dios hacia su puelbo. todo está en las manos de Dios sin importar las sitacciones por las que estuvieran pasando. No sería a través del poder humando como ellos podía preservar.
		\item Propósito\\
			Dios protege de tood mal a aquellos a quien Él ha elegido. recordadndo todo lo que había hecho en su historia era suficiente para que se humillaran delante de Él pues siempre les había mostrado su amor. Ellos no debería temer porque Dios lo había prometido.
	\end{itemize}
	\begin{subsection}{Bosquejo}
		\begin{subsubsection}{La amenaza para los judíos (1-4)}
			\begin{enumerate}
				\item La reina Vasti (1)\\
					Susa era una de las cuatro ciudades que servía como capital de babilonia. Se narra la historia de vasti, la esposa del rey Xerxes que se conoce como Amestri en la historia universal. Vasti sería la madre de Artajerjes. En medio de la fiesta, ella esJerjes había echo una enomre fiesta en donde había reunido a todos los principales de su gobierno. era una fiesta tan majestuosa que duró 6 meses con el objetivo de planear la guerra contra grecia. Aunque no se mencionan las causas de que vasti lo desobedeciera, se menciona que el rey ya estaba embriagado y deseaba mostrar la belleza de la reina. Si el otivo de la negación de Vasti queda en duda, al explicación de su reacción queda clara, en esaf iesta había mostrado su poder, ahí estaba en juego su honor y el hecho de que la reina le hubiera negado algo al rey estaba dejando en ridículo a sus invitados. Asuero pide consjeo a sus príncipes para saber qué era lo que tenía que hacer para recuperar su honor. Sus príncipes les dijeron que había ofendido al rey y a todos los hombres del reino.\\
					Vasti dejó de ser reina y se les ordenó que todas las muejres debían de obedecera us maridos, las mujeres debían de ser intimidadas para que se sometrian a sus maridos. el libro explica cómo la autroridad del rey era absoluta.
				\item La reina Ester (2)\\
					Pasadas estas cosas se acordaba de su esposa Vasti, vs 2. Sus cortesaons se deciaron a buscarle una sustituta y ahí es cuando aparece la mujer hebrea ester y su primo Mardoqueo, fueron transportados cautivos juntos con la tribu de Benjamín y de judá. ester es seleccionada y empiez a ser entrenada junto con otras candidatas antes de ser presentadas antes el rey, vs 8-9. la expresión que dice ``fue llevada'' implica que fue volntariamente isn embargo, hay un documento judío que lo traduce como que fue llevada a la fuerza. La septuaginta introduce dentro del texto, no bíblibo, una oración de Ester en donde dice Ester que se casó con un incircunciso contra su voluntad.\\
					Lacostumbre era quqe cada una de las candidatas a reina debería tener 6 meses de tratamiento de belleza con aceite de mirra, seguida de ortos seis meses de tratamientos con perfumes y cosméticos.\\
					En el turno de ester para presentarse delante del rey, todo lo que ella eseara de ropas y regales le era provisto. Nunca más podía salir del palacio a menos de que el rey le llamara por su nombre.\\
					Cuando llegó el turno de ester, el rey quedó tan complacido con ella que la coronó como reina y así Ester se convirtiría en reina. Ester no declaró  su nacionalidad a consejo de su primo. Al final del cap se narra que Mardoqueo escucha una conspiración entre 2 hombres que estaban tramando matar al rey, lso 2 conspiradores son atrapados y matados.
				\item La rebelión de Amán (3-4)\\
					Amán era amalecita, los que habían sido malditos por Dios. A Saúl se le abía dado la orden de destruir a todos los amalecitas pero no obedeció. Amán fue primer ministro del reino y llegó a ser muy amado por el rey. en 4:2 dice que cuando Amán pasaba todos tenían que arrodillarse delante de él pero dice la narración que Mardoqueo no se arrodillaba delante de él, Mardoqueo sabía que solamente se inclinaba delante de Dios. Los siervos del rey lo acusan con Amán y Amán se llenó de ira aún sabiendo que todo el pueblo judío era igual. Así como los judío odiaban a los amalecitas, los amalecitas odiaban a os judíos.\\
					EN el vs 7 dice que Amá echó ``pur'' (suerte) era una especia de juego que se hacía con unas piedras, el término asirio ``pur'' se traduce como juego o suerte. Amán perdispone al rey e contra de los judíos diciendo que ellos no se estaban sometiendo a sus leyes. El rey firma un decreto el cual involucra también condenación apra mujeres, niños, ancianos y mujeres.\\
					Esta ley es leía a todos los udíos y hubieron llanto y luto. todo esto llegó a oído de Ester quien tenía comunicación a través de Mardoqueo mientras pensaban qué hacer pues la reina no podía acudiar al rey a menos de que el rey la llamara, si ella se presentaba ante el rey sin ser llamada podía ser ejecutada. La tradición era que a pesar de que la reina y el rey vivían en el mismo palacio no comían ni dormían jntos.
			\end{enumerate}
		\end{subsubsection}
		\begin{subsubsection}{Trinfo para los judío (5-10)}
			\begin{enumerate}
				\item Ester y Mardoqueo son honrados (5-6)\\
					El rey le dice que incluso le daría la mitad el reino si se la pedía. Asuero tenía un gran aprecio por Ester, Ester 5:2-3. Ester hizo un plan para informar al rey acerca de los planes de Amán.Mientras, Amán ya había hecho una orca para mandar ahorcar a Mardoqueo. \\
					Dios estaba movviendo todas las pieas para salvar a su pueblo y antes de que ester actuara, hio que el rey fuera a eleer las cornicas del reino donde s e encontró que una vez un hombre llamado Mardoqueo lo había salvado sin tener recompensa. Lee las cornicas del rino una noche en la que no podía dormir e hizo que le leyeran dichas cronicas.\\
					El rey le preguntó a Amán qué debía de hacer para premiar a un hombre y Amán pensando que se trataba de él le dice que habría de cabalgar con honores en la ciudad y así fue como , sin saberl, Amán había determinado de qué froma Mardoqueo iba a ser honrado. él nunca se imaginó que oba a ser humilaldo de esa forma. Amán humillado pasea a MArdoqueo sobre las calles del reino.\\
					después de hacerle los honores a Mardoqueo, Amán estaba abatido pues ya tenía la orca lista. Mientras, Ester planeaba el banquete privado.
				\item Amán es ahorcado (7)\\
					La reina ruega por su vida y la de su pueblo informándole al rey el plan de destrucción contra su pueblo. Ester no demora a apuntar a Amán como el culpable de tal ley. El rey y Amán se entera de que Ester es judía y Amán se entera que había puesto la vida de la reina en peligro por su mandato y se prendió la ira del rey. Finalmente el rey ordena que su primer ministro Amán fuer aahoracdo en la misma orca que había planeado para Mardoqueo, Ester 7:9.
				\item La defensa judía (8-10)\\
					Mardoqueo fue promovido simplemente por haber salvado la vida del rey. Era costumbre que los bienes de los criminales pasaban a ser de la corona por lo que el rey le regalaba esos bienes a Ester. Ester y mardoqueo hace otro decreto para que contrarreste al anterior, el rey encarga a su segundo, Mardoqueo, para ue se anule el amdanto anterior y ahora la maldición era para quien matara a los judíos. mardoqueo dio la orden de que ese día se conmemorar todos los años y ahora se conoce como la fiesta del Purín. Mardoqueo, nombrado como segundo siempre se preocupó por que hiubiera bienestar para el pueblo judío.
			\end{enumerate}

		\end{subsubsection}
	\end{subsection}
\end{section}
\end{document}


