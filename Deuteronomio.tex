%        File: Deuteronomio.tex
%     Created: Mon Sep 09 08:00 PM 2019 C
% Last Change: Mon Sep 09 08:00 PM 2019 C
%
%\documentclass[a4paper]{article}
%\usepackage{enumerate}
%\usepackage[spanish]{babel}

%\begin{document}
\begin{section}{Deuteronomio}
	\begin{itemize}
		\item Título\\
Viene de la palabra ``Deutero'' que significa ``segundo'' y `` Nomos'' que significa ``ley'' sin embargo no es una segunda ley. Solamente hay una ley que es la del Sinaí que Dios le entregó a Moisés. Una buena traducción sería ``Segunda promulgación de la ley''. En el hebreo se le llama ``ellehhadd barim'' que significa ``estas son las palabras``. Muchos de los escritos de la época así empezaban, ésto nos muestra que Moises era una persona culta y educada en ciencias, sabio.
		\item Autor y fecha\\
			Desde el primer versículo se menciona que son las palabras de Moisés las que se están registrando. Es un discurso de Moisés para la nueva generación antes de que entraran a la tierra prometida. Se sitúa en enero o febrero del 1405 a.C. Moisés da las últimas instrucciones a aquellos que entrarían a tomar la tierra prometida. El tiempo de la historia de Israel practicamente no avanza, los discursos fueron hablados cuando el pueblo estaba en los campos de Moab, el principal propósio del libro es recordarles la relación tan especial que tienen con Dios. Los alienta a obedecer todos los mandamientos de la ley. Probablemente la generación nueva no conocía la ley y Moisés se asegura de que ellos la conocieran exactamente de la misma forma que la conocían sus padres. Les muestra los errores de sus padres con el propósito de que le fueran fieles a Dios y pudieran entrar a la tierra prometida. El pueblo está exactamente en la frontera de la tierra prometida\\

		\item Tema\\
Confirmación de la ley.
		\item Propósito\\
		Informar a la nueva generación de la obra de Dios en Su pueblo así como exhortarlos a permanecer en santidad.
	\end{itemize}
	
	\begin{subsection}{Bosquejo}
		En el primer versículo se anuncia el contenido de todo el libro, 4 discruso de Moisés en los llanos de Moab, previo al inciio de la conqista.
		\begin{subsubsection}{Primero discruso de Moisés}
			Moisés hace un breve resumen del peregrinar del pueblo por 40 años de manera cronológica, están acampando sobre la llanura de Moab. Se le llama ``más allá del Jordán'' a dicha zona mientras que ahora se le conoce como transjordania. Finalmente Israel se prepara para conquistar.\\
			Moisés habla de cuando los liberan de Egipto y la penosa peregrinación por el desierto de 40 años, era necesario que les explicara por qué se habían tardado en llegar pues ya había muerto esa generación. Les resume el fracaso en Cades-barnea, cuando a pesar de que parecía imposible la tarea no confiaron en la ayuda de Dios.\\
			En el capítulo 2, les recuerda las instrucciones de Dios para los pueblos hermanos y los descendientes de Lot. También a esos pueblos Dios le dio un terreno pues los amonitas eran descendiente de Lot.\\
			 En el capítulo 3 les muestra cómo con la ayuda de Dios deerrotaron al rey amorreo y el rey de Basán.\\
			 Les explicó que debían de aprender de esas experiencias anteriores para no adulterar los mandamientos de Dios y que llevaran una vida santa así como no olvidarse de la naturaleza de Dios y de la de ellos.
		 \end{subsubsection}
	\end{subsection}
	\begin{subsubsection}{Segundo discurso}
		Vemos nuevamente que Moisés les repite el decálogo que les dio a su padres en Éxodo 20, ahora en el capítulo 5. Debían entender la relación tan especial que tenían con Dios pues Él los escogió y los comprometía a una gran responsablidad para con Él. En Deuteronomio 5:1 les dice ``Oye Israel\ldots'' como una llamada de atención, les enseñó la ley para que la guardaran y la llevaran a cabo. En los últimos versículos les recuerda las condiciones en las que Dios se las entregó en el monte Sinaí.\\
		En el capítulo 6 Moisés da recomendaciones para el pueblo sobre las responsabilidades que tenían para con Dios.\\
		En los capítulos 8-11 recuerdan bendiciones pasadas y les recuerda lo que les podría pasar si desobedecían, 9 veces les dice ``acuérdate'' con el objetivo de que recordaran la voluntad de Dios para su pueblo. Ésto también debería de ser recordado por la iglesia de Cristo con el propósito de que siempre pensemos en cual es la voluntad de Dios para nuestras vidas.\\
		En los capítulos 10-11 les da más exhortaciones sobre lo que Dios quiere de ellos.\\
		En el capítulo 12 se habla de lo que habrían de hacer una vez que ocuparan la tierra prometida\\
		Durante los capítulos 14-26, les da diversos reglamentos y leyes para que llevaran una vida práctica frente a Dios. hay temas muy variados en dichos capítulo pues habla de leyes dietéticas, fiestas judías, leyes de ofrendas, leyes sobre cómo considerar a la tribu de Leví, del refugio para los asesinos, antes de juzgar que supieran si era culpable, habla sobre la guerra, vida familiar, asesinatos y demás temas de la vida práctica.
	\end{subsubsection}
	\begin{subsubsection}{Tercer discuros}
		Moisés les recuerda del pacto de Dios para con ellos que se hizo en el Sinaí, en el capítulo 27 vemos un ritual de colocar piedras grandes. En ese capítulo se mencionan el monte Jerizim y el monte Ebal, las bendiciones sobre Jerizim y las maldiciones sobre Ebal. Se dijeron 12 maldiciones sobre el pueblo si no cumplían y las bendiciones si sí lo hacían.\\
	En los capítulos 29-30 se habla de la renovación del pacto mosaico. Es el mismo pacto pero ahora se hace con la nueva generación. Tiene las mismas caracterísiticas, solamente se ratifica el momento preciso en el cual comienza la conquista.
	\end{subsubsection}
	\begin{subsubsection}{Cuarto discruso}
		Se encuentra en los capítulos 31-34, en el capítulo 31 Moisés presenta a Josué ante el pueblo como su sucesor con sus responsabilidades para guiar al pueblo y lo anima diciéndole que Dios siempre lo iba a ayudar. Dios le advierte a Josué que el pueblo volvería a inclinarse ante los ídolos. El hombre a través de sus pecados volvería a romper su relación y Dios les advierte que cuando ese momento llegara los abandonaría, posteriormente Moisés escribe su cántico.\\
Termina con Deuteronomio 34:4 con Moisés obedientemente subiendo al monte para contemplar la tierra para la cual había guiado al pueblo, 40 años Dios lo preparó en la corte del faraón, 40 años en la soledad del desierto y 40 años conduciendo al pueblo por el desierto.\\
	\end{subsubsection}
	
\end{section}


%\end{document}


