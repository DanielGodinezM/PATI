%        File: Josue.tex
%     Created: Wed Sep 11 07:00 PM 2019 C
% Last Change: Wed Sep 11 07:00 PM 2019 C

%\documentclass[12pt]{article}
%\usepackage[margin=1.0in]{geometry}
%\usepackage{enumerate}
%\usepackage[spanish]{babel}
%\usepackage[useregional]{datetime2}

%\begin{document}
\begin{section}{Josué}
	\begin{itemize}
		\item Título\\
			Corresponde al personaje principal de los acontecimientos en esta parte de la historia del pueblo de Israel. En hebreo su nombre es \textit{Joshua} que significa ``Jehová salva'' y su traducción al griego corresponde con el nombre de Jesús.
			En un principio, el nombre de Josué era Oseas, sin embargo, Moisés se lo cambió en Números 13:16.\\
			Josué queda nombrado por Dios como el sucesor de Moisés en el liderazgo del pueblo de Israel en Deuteronomio 31.
		\item Autor y fecha\\
El autor no es directamente nombrado pero se asume que Josué lo escribió ya que él era el lider del pueblo y considerando que la mayoría de los relatos son de eventos que vivió Josué de manera directa.
Dado que empieza narrando los preparativos para la entrada a la tierra prometida es congruente suponer que se escribió entre 1405 a.C. y 1385 a.C. 
		\item Tema\\
El muy esperado cumplimiento de la promesa que Dios les había dado de entregarles una tierra.
		\item Propósito\\
			Narrar la conquista de Canaán a pesar de los múltiples pecados del pueblo por medio de la fidelidad constante de Dios para cumplir su promesa.
	\end{itemize}
	\begin{subsection}{Bosquejo}
		\begin{subsubsection}{Entrando en la tierra prometida (Josué 1-5)}
				En los primeros versículos observamos que Dios le garantiza la victoria a Josué y además le ordena en múltiples ocasiones que se esfuerce y que sea valiente.\\
Si Josué quería poseer la tierra prometida, él tenía que ir a cumplir lo que Dios le había pedido. 
En el capítulo 2 Josué manda espías a la ciudad de Jericó, estos 2 investigadores debían de observar en qué condiciones estaba la ciudad.\\\\
Los espías llegaron era una posada situada sobre el muro. La mesonera era Rahab a quien se le llama ramera. Rahab venía de un trasfondo pagano y la prostitución que menciona el pasaje podría referirse a que era una idólatra y no una ramera literal, sin embargo, el hecho de que viviera en un mesón sobre el muro de la ciudad puede ser un indicio de que sí fuera una ramera literal. Es importante aclarar que no se puede asegurar si su prostitución era física o espiritual.\\
Rahab ayudó a los 2 espías y les pidió a cambio que respetaran su vida y la de su familia cuando invadieran la ciudad. Ella les confesó que el pueblo de Jericó estaba aterrado después de escuchar las maravillas que había hecho Dios en su salida de Egipto tal y como Dios les dijo en Deuteronomio 2:25.
\newpage
Los espías hacen el trato con Rahab y esto llegaría a tener una gran trascendencia ya que Rahab se casa posteriormente con Salmón de la tribu de Judá quien fue el tatarabuelo del rey David.\\
Después de esconderlos, ella dejó que los espías escaparan y en el capítulo 3 se narra cómo pasó el ejército israelí por el Jordán para entrar  a la tierra prometida.\\\\
Dios repite el milagro que había hecho con Moisés ahora con Josué separando las aguas del Jordán. En Josué 3:15-17 se narra cómo cruzan en seco en dirección a Jericó marcando así el inicio de su campaña militar. Dios envió dichos eventos milagrosos también para confirmar el llamamiento de Josué para guiar a su pueblo.\\
En el capítulo 4, Dios le ordena a Josué que levantara un monumento. Josué obedece la orden de Dios y además él mismo toma 12 piedras y hace un monumento en el río por donde pasaron los sacerdotes. Después de cruzar, acampan en Gilgal y allí Josué construye el monumento con las 12 piedras que habían llevado los 12 hombres con el propósito de glorificar a Dios y que se conmemorara lo que Él había hecho por el pueblo. Estos dos monumentos establecen que las 12 tribus habían estado juntas vagando por el desierto y así entrarían juntos a la tierra prometida.\\\\
En el capítulo 5, se ve que el miedo que sentían los habitantes de Jericó se basaba en las noticias que recibían de lo que estaba haciendo Dios con Su pueblo.\\
En Josué 5:2-9 Josué circuncida a todos los hombres de Israel por mandato de Dios pues se menciona que ya se había olvidado la señal de la circuncisión, es decir, que la nueva generación no había sido circuncidada pues sus padres que vagaron en el desierto no habían obedecido ese mandato de Dios.\\
En Josué 5:10 se narra que tenían que celebrar la pascua antes de iniciar la conquista. La celebración de la Pascua inauguraba una nueva etapa en la historia de Israel en Canaán y ahora por primera vez se celebra en la tierra prometida. Se enfatiza en el libro una nueva etapa que implicaba una ruptura con la historia de la antigua generación.\\
En Josué 5:11-12 se narra que el pueblo comió del fruto de la tierra y a partir de ese momento ya no se alimentaban del maná del cielo. El pueblo pasó del maná al fruto de la tierra que Dios les estaba dando. El éxodo ya había terminado pues ya estaban en la tierra prometida y así es como se cumple Deuteronomio 6:10 ya que recibieron comida que no sembraron. Dios les dio todo lo neccesario para que pudieran subsistir.\\
Se puede observar una cristofanía en Josué 5:13-14 pues Jesucristo es la representación física de Dios. La instrucción que da Jesucristo es la misma que recibió Moisés en la zarza ardiente, que se quitara sus sandalias pues era tierra santa donde estaba.\\

		\end{subsubsection}
	
		\begin{subsubsection}{Conquistando la tierra prometida (Josué 6-12)}
		En 1887 fueron descubiertas unas tablillas escritas en donde se habla de un ejército que iba arrollando a todo ejército  conquistaba cualquier pueblo. Las conquistas duraron aproximadamente 7 años y fueron llevadas a cabo en tres etapas.
		\begin{enumerate}
			\item En la parte central de la tierra\\
				Israel conquista fácilmente a Jericó. Jericó era un eje político y religioso,un lugar estratégico para comenzar la conquista. Por ser una ciudad importante era un centro de corrupción, violencia y desenfreno. Dios iba a castigar el desenfreno de esa tierra y a la vez iba a darles una lección a Israel para que vieran lo que les ocurría a las personas que se alejaban de Él.\\
				El ejército de Jericó pensó que si estaban bien defendidads las puertas de la ciudad los israelitas no podrían entrar. La táctica militar para entrar a Jericó varía mucho con las técnicas de los hombres para conquistar ciudades. Los secerdotes debían de rodear la ciudad por 7 días, en esos versículos del capítulo 7 se repite mucho el número 7, el número de la perfección con el objetivo de indicar que era Dios quien estaba dirigiendo la batalla. El arca iba en medio de dos batallones, uno en la vanguradia y otro en la retaguardia.\\\\
				Siete días desfilaron sin hablar y al $7^{o}$ día dieron 7 vueltas. Solamente se esuchaba el ruido de las trompeta mientras el pueblo mantenía silencio y a la orden de Jousé en el Josué 6:20 el pueblo gritó con gran estruendo y el muro se derrumbó. Se ha descubierto que el muro tenía 7 metros de ancho y 9 de altura entonces al derrumbarse el muro nada impedía que Israel entrara a tomar la ciudad. Los soldados siguieron las instrucciones de destruir y matar a todo a quien encontraran.\\
				Después de que salvaran a Rahab, a su familia y sacaran sus pertenencias lo quemaron todo. La destrucción de Jericó fue algo verdaderamente portentoso, al cumplirse los 7 días Jehová había cumplido lo que había prometido. Recordaron que así como el arca había abierto el mar Rojo, así abrió las puertas de Jericó. Para concluir con la primera etapa, Josué dice una malción en contra de cualquier persona que intente reconstruir Jericó y se ve en $1^{o}$ Reyes 16:34 que Hiel fue objeto de dicha maldición pue s al tratar de edificar a Jericó, dos de sus hijos perdieron la vida.\\\\
				En el capítulo 7 se ve de nuevo que cuando el hombre deja de obedecer a Dios hay consecuencias. Había instrucciones de que la ciudad sería anatema y por ellos los soldados no podrían tomar botín pero Acán desobedeció y tomó botín e hizo esta desobediencia que se incendiara la ira de Jehová.\\
				La siguiente ciudad a conquistar era Hai, considerablemente más chica que Jericó pero los hombres de Hai los derrotaron por el pecado que confiesa Acán en Josué 7:21. Ésta derrota desanimó al pueblo de Israel y hasta Josué dudó de la fidelidad de Dios. Dios le llama la atención y le menciona que había pecado en el pueblo de Israel. Pronto se descubre el pecado de Acán y él junto con su familia son apedreados y quemados.\\
				En el capítulo 8 una vez más atacaron a Hai y ganaron de forma que quemaron la ciudad.\newpage
				La siguiente ciudad por conquistar era Gabaón, los gabaonitas temerosos por el ejército de Israel narra Josué 9:4 que usaron de su astucia para engañar a los israelitas y salir librados de los combates, les dijeron que eran habitantes de un país lejano y que no representaban un peligro para ellos de acuerdo a Josué 9:4-6. Josué respondió con base en sus sentimientos y no a lo que Dios había establecido pues en Éxodo Dios ya le había dicho que el pueblo de Israel debía de matar a los habitantes de la tierra prometida. En Josué 9:14-15 Josué actuando con base en sus sentimientos olvida el mandamiento de Jehová y hace una alianza con los gabaonitas. Cabe resaltar que no consultaron la voluntad de Jehová en este aspecto.\\
				Pocos días después de la alianza se dieron cuenta que eran vecinos y al llegar a las ciudades de los gabaonitas no los mataron pues estaban comprometidos a cumplir con su alianza. La maldición que dirigió Josué hacia los gabaonitas en Josué 9:23 resultó ser una bendición para ellos pues posteriormente los gabaonitas habrían de unirse del pueblo de Israel y habrían de servir al mismo Dios.
			\item En el sur de la tierra de Canaán \\
				En el Josué 10:1 se nombra por primera vez a la ciudad de Jerusalén y tenía un rey que precisamente fue el rey que encabezó a los demás reyes para oponerse a las fuerzas de Josué y por ello decide atacar a los gabaonitas pero Josué los vence matando a todo sus habitantes.\\
				En la batalla que se narra en el capítulo se narra un suceso sobrenatural pues dice Josué 10:13 que el sol y la luna detuvieron su avance, el sol y la luna se detuvieron. Se han hecho muchas teorías acerca de este acontecimiento tratando de explicar cómo fue que se detuvieron. Se dice que solamente fue un fenómeno de refracción en el que solamente parecía que el sol tardaba en ponerse, sin embargo, hay registros en otras civilizaciones que hubo un día que duró más que los demás y también se menciona en Josué 10:13 que dicho milagro está anotado en el libro de Jaser.
				\begin{itemize}
					\item Dificultades de interpretación\\
				Existe otro reto de interpretación en estos pasajes pues constantemente se dice que Israel derrotó a los ejércitos y mataban a todo lo que tenía vida, así lo había mandado Dios. Es comprensible confundirnos ante el mandanto de aniquilación total solamente para que Israel habitara la tierra pues las instrucciones podrían considerarse brutales, sin embargo, el Dios que adoramos es un Dios verdadero,  que dio esta orden porque debe de haber una razón justa y vemos que los cananeos eran una amenaza militarmente y para la pureza del pueblo de Dios ya que hubieran contaminado al pueblo. \\
				Dios había santificado y adoptado al pueblo y con esta orden los estaba protegiendo. Dios ordenó a todos ellos a muerte debido a que cada uno de los habitantes eran personas rebeldes, idólatras, enemigas de Dios y sabemos que la paga del pecado es muerte. No nos deben de extrañar estas instrucciones cuando vimos que Dios consumió a todo ser viviente en el diluvio, así como la destrucción de Sodoma y Gomorra. \newpage
				Cuando leemos de la justa ejecución de los cananeos debemos de preguntarnos cómo es que Dios no nos ha exterminado a todos y la razón es porque Cristo vino al mundo para morir en nuestro lugar para que pudieramos vivir. Dios sigue aplicando la justicia en cada pecador solamente que el castigo recae sobre la persona de Jesucristo para todo aquel que cree.\\
				La superviviencia dependía de que obedecieran las leyes que Dios les había dado, debían de comprender que todo lo debían de hacer solamente con la autorización de Dios. La guerra correspondía a Dios, Él decidía cuando un pueblo ya había llegado al colmo de la maldad y vemos a Israel como la mano ejecutora del juicio de Dios.\\
				Nosotros que somos criaturas no podemos juzgar a Dios, Él conoce bien qué es lo que tiene que hacer. Conoce el futuro de cada ser humano desde el principio. En cualquier caso, no nos corresponde conocer los designios de Dios.\\
Como conclusión a este problema se puede decir que Dios ha de cumplir sus propósitos independientemente de los propósitos de los hombres. Es importante reconocer que Dios siempre tuvo un propósito redentor para los que ha escogido, su propio Pueblo no reflejó esta forma de su justicia y a pesar del poder que ya habían visto de Dios, ellos siguieron ofendiéndolo por lo que Dios los siguió castigando. Gracias a Dios la muerte que nos correspondía a nosotros es la muerte que tuvo Jesús en la cruz. \\
Ese mismo día Josué fue con otros pueblos y también los derrotó. 
			\end{itemize}
		\item En el norte de la tierra de Canaán\\
			Después de esta gran victoria de Israel, el rey de Hazor envío mensaje a todos os reyes de la zona del norte con el objetivo de que todos juntos juntos pelearan contra Israel, sin embargo, Dios le confirmó a Josué que habrían de matarlos a todos, quemaron todas sus ciudades y mataron todo lo que en ellas tenía vida.\\
			En el capíulo 12 se hace una lista de los 31 reinos que conquistó Josué por medio de Jehová.

	\end{enumerate}
	\end{subsubsection}
	\begin{subsubsection}{Repartiendo la tierra prometida (Josué 13-22)}
		Dos tribus y media ya habían merecido su heredad desde antes de haber entrado ala tierra prometida así que en Josué 14 se narra cómo Josué le da su tierra que le había sido prometida a Caleb. En este capítulo se menciona que la edad de Caleb era de 40 años cuando Moisés lo envió a Cades-Barnea (Josué 14:7) y posiblemente era de la misma edad de Josué. Ya habían pasado 45 años de la entrada a la tierra prometida, por ello menciona Caleb que tiene 85 años.\\
		En ese reparto le tocó  el reparto a Judá, Efraín y a la mitad de la tribu de Manasés, después procede la repartción a Benjamín, Simeón, Zabulón, Isacar, Aser, Neftalí y Dan. Después se asignaron las ciudades de refugio y en Josué 21 se habla de que todas las tribus deberían de tener ciudades epecíficas para que ahí vivieran los levitas pues esa es la herencia que Dios le daba a la tribu de Leví. Las tribus de Judá y Efraín serían representantes en el futuro de la tribu del norte y la del sur.\newpage
		Josué les dijo  que Jehová les dio reposo como lo había prometido a la mitad de la tribu de Manasés y a las tribus de Gad y de Rubén. Josué los devuelve a donde estaban sus familias, sus pertenencias y además les entrega riquezas,ganado y botín como una recompensa material por acompañar al pueblo en la conquista.\\
		Al regresar las tribus de Rubén, Gad, y la mitad de Manasés, construyeron un altar muy grande de forma que se viera de ambos lados del Jordán pero del lado occiddental hubo alboroto pues tenían la orden de destruir a todo aquel que presentara actitudes que peligraran su santidad. La insistencia en que no debía de existir otro altar se basa en que Israel durante este tiempo sólo se habían nombrado dos altares donde habría de adorarse a Dios, uno en Siquem y otro en Silo. Ellos veían este nuevo altar como contrario a los únicos dos altares que se habían construido. Por esto, Rubén, Gad y la mitad de la tribu de Manasés respondieron que no era un altar de adoración sino que era un recordatorio de que más allá del Jordán había parte del pueblo escogido de Dios.	De esta menara fueron convencidos sus acusadores y los dejaron ir en paz.\\

	\end{subsubsection}
	\begin{subsubsection}{Reteniendo la tierra prometida (Josué 23-24)}
		Josué les siguió repitiendo que la única condición para que Dios los siguiera ayudando era que debían de obedecerlo, aparterse y no convivir con los demás habitantes de la tierra aspi como no juntarse con ellas. Desafortunadamente, el pueblo de Israel se contaminó con los pueblos paganos y así también es como se contamina el creyente uniéndose en yugo desigual con incrédulos pues el creyente se termina desviando de lo que conoce de Dios.\\
		En Josué 23:13 se narran las ultimas instrucciones de Josué que fueron para que Israel le sirviera a Dios con integridad. A Josué le debió de haber dolido que la generación que había entrado limpia a la tiera prometida se prostituyera con los demás pueblos. Israel continuaba en la idolatría pero había llegado el momento de tomar una firme  decisión, ya era el momento de que ellos respondieran a lo que Dios había hecho con ellos.\\
		La generación que había muerto en el desierto ya les había enseñado a adorar a otros dioses a pesar de que serían ellos quienes entrarían a la tierra prometida. Ellos se contaminaban con esos dioses y empezaban a adorarlos y a rendirles culto pues bien se ha visto en relatos pasados que al ser humano le encanta la novedad religiosa.\\
		Había dioses que les convenía adorar pues les permitían hacer cosas que Jehová no les permitía, Josué siendo sensible los llama a una revisión de la relación que tenían con Dios, después de verlo actuar tantas veces a su favor.\\
		Josué le llama la atención a su pueblo pues debían de servirle únicamente a Él y confiar en Él. En Josué 24:15 los cuestiona y saca a la luz la idolatría del pueblo. El pueblo al verse confrontado responde confiando en sus propias fuerzas.\\
		En Josué 24:29-33 se narra la muerte de Josué a los 110 años y fue sepultado en el monte de Efraín. Josué lo único que hizo fue apuntar hacía Dios, la atención no era sobre el pueblo sino sobre Dios quien estaba haciendo la obra. Dios les había advertido que la desobediencia podía quitar las bendiciones que ya les había dado.
	\end{subsubsection}
\end{subsection}
\end{section}
%\end{document}


