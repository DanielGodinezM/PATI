%        File: Genesis.tex
%     Created: Mon Aug 26 06:00 PM 2019 C
% Last Change: Mon Aug 26 06:00 PM 2019 C

%\documentclass[a4paper]{article}
%\usepackage{enumerate}
%\usepackage[spanish]{babel}

%\begin{document}

\begin{section}{Génesis}
	\begin{itemize}
		\item Título\\
		Se le otorgó el nombre de Génesis cuando se formó la septuaginta alrededor del 150 a.C. Este nombre le fue otorgado			como la traducción del hebreo \textit{toledot} ya que el contenido principal del libro es acerca de lo orígenes del 			mundo, el universo, la humanidad y entre otras cosas, del pueblo de Israel.

		\item Autor\\
		El libro no nombra al autor de manera explícita pero hay varios pasajes dentro la Biblia que indican que Moisés fue 			quien escribió todo el Pentateuco. Entre dichos pasajes podemos nombrar como ejemplo a:
		\begin{itemize}
			\item $2^o$ Crónicas 35:6
			\item San Lucas 24:25-27
		\end{itemize}
		\item Fecha\\
		Probablemente fue escrito durante el periodo del Éxodo, en particular, durante el tiempo en el que estuvieron vagando 			el desierto, de ser cierto esta suposición, significaría que  Génesis fue escrito entre 1445a.C. y 1405a.c. 				aproximadamente.

		\item Tema\\
		Relata el origen del universo, Dios llevando a cabo su creación y cómo es que Él da origen a la vida, el ser humano y 			en particular, el pueblo de Israel más adelante.


		\item Propósito\\
		Exponer el plan redentor de Dios para el hombre.\\
		Es de suma importancia el tener un correcto entendimiento del libro ya que también en Génesis está el origen de 			prácticamente todas las doctrinas que Dios le reveló al hombre, de esa manera es que lo expone Evis Carballosa en la 			introducción de su comentario de Génesis.\\
%		``Una buena comprensión del libro de Génesis es sumamente importante para entender toda la verdad Bíblica, porque en el primer libro de la Biblia se encuentran prácticamente todas las doctrinas reveladas en la Palabra de Dios. Por está razón, en medio de la confusión y rebeldía de la sociedad contemporánea necesitamos proclamar desde los púlpitos todo el mensaje de Génesis``-Evis Carballosa en la introducción a su comentario de Génesis\\
		Debe considerarse como un libro científicamente preciso. Dios se revela a sí mismo y revela la naturaleza caída del 			hombre.\\
		En él se encuentra la manifestación de la Trinidad, los pactos, la redención, el origen de satanás y los ángeles, la 			promesa, entre otras cosas.
	\end{itemize}
	\begin{subsection}{Bosquejo}
		\begin{subsubsection}{La historia Primitiva(Cap 1-11)}
			\begin{enumerate}
				\item Creación (Cap 1-2)
					\begin{itemize}
\item Génesis 1:1\\
Menciona el inicio de la creación del universo, no se refiere al inicio de la existencia de Dios ya que Él es eterno, Él creó todo lo que existe. Lo que se menciona en este primer capítulo es la verdad. Antes del principio, solamente existía Dios y todo fue creado por Él, a excepción de Él mismo.\newpage
\textit{Ex-nihilo} es una expresión del latín que se ocupa para describir la manera en la que Dios creó el universo, ésto quiere decir ``de la nada''. Dios creó el universo y todo lo que hay en él \textit{ex-nihilo}, es decir, lo creó de la nada. No es posible explicar en términos humanos aquello que está fuera del intelecto humano, la creación del universo fue un evento sobrenatural el cual no podemos explicar con el conocimiento primitivo que tenemos acerca de cómo funciona el universo.\\
Las teorías humanas acerca del inicio del universo han ido cambiando a través de la historia, cada teoría se adapta a los nuevos descubrimientos científicos y conforma va avanzando la tecnología. La verdadera ciencia es objetiva y se lleva a cabo siguiendo el método científico, a las teorías que carecen de una demostración rigurosa no se les puede considerar como verdad absoluta.\\
La forma en que el universo tuvo su origen no puede ser comprobada como verdad por medio del método científico ya que la naturaleza del método científico es empírica. La ciencia moderna ha sido construida por medio de matemática y pensamiento empírico que conlleva varias limitaciones. El hombre natural evalúa la creación utilizando solamente el intelecto humano y lo que se piensa que es verdad cuando hay cosas que el método científico no es capaz de demostrar.\\
Entre las teorías más comunes que se dedican a encontrar una edad aproximada de existencia del planeta Tierra se encuentran:
\begin{itemize}
\item Teoría del uniformismo\\
	Esta teoría consiste en analizar los cortes que se encuentran en las rocas y así determinar que la Tierra debe de tener una edad de miles de millones de años. Sin embargo, hay una gran variedad de fenómenos naturales que pueden alterar la periodicidad en el que se formen dichos cortes, provocando que sus apariciones no sean de manera lineal así como lo asume ésta teoría.
\item Teoría del catastrofismo\\
	Es la principal oposición al uniformiso. Afirma que todo tipo de fenómeno natural violento afecta la formación de los estratos a distintos intervalos de tiempo por lo que ataca el fundamento principal en el cual se sustenta el uniformismo.
	\end{itemize}
Hay un gran peligro de que haya gente que trate de acomodar lo que dice la Biblia con las teoría que va formando la ciencia, entre éstas personas se encuentra un gran número de católicos así como el comentarista de la Biblia Católica Latinoamericana que afirma que el autor del libro de Génesis habla de manera literaria al principio del libro y que a partir del capítulo 11 ya habla sobre el contexto estrictamente histórico.\newpage
Este comentario es claramente incorrecto ya que lo dice porque le es imposible que la creación del universo haya sido realmente como lo narran los primeros capítulos del libro de Génesis. El comentarista está haciendo una interpretación de dichos pasajes como un hombre natural tal y como lo narra $1^a$ Corintios 2:14. Es muy importante que podamos reconocer las limitaciones del intelecto humano así como las inexistentes limitaciones que tiene el intelecto de Dios tal y como lo menciona Pablo en Romanos 11:33-35.


\item Malinterpretación de Génesis 1:2\\
La Teoría del salto afirma que en Génesis 1:1, Dios creó la Tierra completa y a la perfección e incluso con vida, ya habían hombres y ciudades. El sol y la luna ya estaban en existencia.\\
Afirma que en esa época fue cuando exisiteron los dinosaurios y que hubo un cambio catastrófico debido a un juicio divino para el versículo 2. Se apoya en el tiempo en el cual Satanás se rebeló y por ello provocó que la Tierra estuviera desolada.\\
					Posteriormente, afirma que Dios volvió a crear vida. La teoría trata de armonizar los millones 						de años que se afirma que tiene de existencia el universo.\\
					\underline{C. I. Scofield} fue dispensasionalista, premilenialista, un teólogo importante y apoyó esta teoría con Jeremías 4:23-26, Isaías 45:18, Ezequiel 28:12-15 e Isaías 14:9-14. En estos pasajes se habla de la desolación de la Tierra y él afirma que confirman la teoría. Scofield afirma que el día y la noche se refería solamente a un intervalo de tiempo indefinido.\\
					\underline{Henry Morris} dice que Jeremías 4:23 habla de la situación de una Tierra en esas condiciones pero no se refiere a todo el planeta, se refiere a la desolación de Israel después de que los destruyeran.\\
					King James es quién hace la mejor traducción del pasaje que en vez de ``desordenada'' lo menciona como ``sin forma''.\\
					Utilizando una buena hermenéutica no se encuentra un sustento de la teoría del salto.

				\item Génesis 1:3-5\\
				Dios establece los periodos alternados de luz y tinieblas, al igual que con los primeros dos versículos, en ellos podemos encontrar distintas interpretaciones. \\
				No hay indicación clara de por qué es que ya había luz. Hay algunos que interpretan dicha luz como la gloria de Dios pero carece de sustento bíblico. \\
					Se sigue viendo el poder de la Palabra de Dios en la creación a lo largo de estos versículos, se repite constantemente la muestra del poder de Dios por medio de su Palabra.\\
					En Hebreos 11:3 queda claro que Dios no es un ser creado y que en Su eternidad no existía el tiempo.\newpage

				\item Génesis 1:6-10\\
					Se narra cómo es que Dios ordenó a toda la materia para así prepararla para la vida. La ``aguas de arriba'' se refiere a una capa de vapor que envolvía a la Tierra antes del diluvio mientras que las aguas debajo del cielo se separaron de forma que apareciera terreno seco.

					\item Génesis 1:11-13\\
					Después de que Dios creara todo lo que no tenía vida ahora se narra cómo crea la vida y la crea de forma madura, no infante.\\
					Alexander Opalin en 1920 postuló una teoría acerca del origen de la vida llamado ''El origen la 					vida en la Tierra``, en dicho documento él afirma 4 puntos principales:
					\begin{enumerate}
						\item La Tierra primitiva tenía una atmósfera primitiva con  casi nada de oxígeno.
						\item Actuaban chispas de origen eléctrico.
						\item Se acumularon dichas chispas hasta que los océanos primitivos tuvieran una viscosidad de sopa caliente
						\item Dicha viscosidad interactuando con las chispas de la atmósfera dieron como inicio a la vida.						
					\end{enumerate}
					Job 33:4 vuelve a confirmar el hecho de que Dios es el que crea y sustenta la vida, negando retundamente que el inicio de la vida sea solamente por la consecuencia de las condiciones del ecosistema terrestre.\\
					Es importante recalcar que el hombre no es capaz de crear, solamente manipula la materia que ya existe en el universo y utiliza su conocimiento limitado para aprovecharla a su propio beneficio.
	

					\item Gen 1:14-19\\
					Se menciona que la creación de los cuerpos celestiales fue con el propósito de que  marcarían el paso del tiempo y servirían de	señales para el paso del tiempo en la Tierra y el universo.

				\item Gen 1:20-25\\
				Dios termina de preparar su escenario con seres vivientes que vivirían sobre la tierra seca, capaces de desplazarse sobre el suelo y por los cielos.


				\item Gen 1:26-27\\
				Narra la conclusión de la creación con la creación del hombre, la corona de la creación, un ser capaz de razonar y de comunicarse con Él.\\
			La pseudociencia afirma que el hecho de que el hombre tenga razonamiento es consecuencia directa de la evolución, un ejemplo de ello son las conclusiones falsas a las que llega Darwin después de observar las distintas adaptaciones de los seres vivivos.En dicha teoría, Darwin afirma que los microorganismos que fueron sobreviviendo sobre los demás de su misma especie fueron aquellos que tenían diferencias genéticas que les permitían sobrevivir mientras los demás morían. Dichos atributos genésticos de supervivencia fueron pasando de generación en generación de forma que cada una traía carga genética más apta para la sobrevivencia.\newpage
					La teoría evolucionista ha sido apoyada por ateos como una verdad solamente para poder 						negar la Palabra de Dios y a pesar de que carece de demostración rigurosa.\\
La \textbf{Biblia Católica de Estudio} menciona que el autor de Génesis se inspiró en los relatos de la creación de otras religiones y lo escribió de manera meramente poética con el propósito de dar una idea de cómo fue la creación sin embargo, con una correcta interpretación del pasaje podemos entender que la narración del relato es literal.\\
				Una mejor explicación es la que da el pastor John MacArthur:
					''Es evidente que la creación de la raza humana es el asunto central de Génesis. Todo lo demás 						culmina en este aconteciminto, y el texto bíblico dedica más espacio a la decripción de la 						creación del hombre que al de todos los demás aspectos de la creación. De hecho, puesto que 						este que este acto final de creación es tan crucial, todo el capítulo 2 de Génesis se dedica 						a la ampliación de la descripción susodicha. Cabe aclarar que Génesis 2, no es una historia 						diferente, ni un relato alternativo sino la descripción ampliada del mismo día sexto en 						Génesis 1. En Gen 1:26-31 aprendemos las verdades básicas acerca del día sexto``\\
Es innegable la semejanza anatómica del hombre con otros animales, hay una íntima dependencia que tiene el hombre con la tierra y las demás especies, sn embargo, ésto no es debido a que el hombre haya provenido de dichas especies. Es importante recalcar el hecho de que el ser humano no fue creado \textit{ex nihilo}, fue creado de los elementos ya creados en la Tierra.\\
Al hombre se le dio autoridad sobre la naturaleza, ésto lo afirman Génesis 1:28 y Salmos 8:5. Antes de que Dios	creara al hombre hubo un ''Concilio de la Trinidad``, Él decretó cómo es que iba a crear al hombre.\\
				Una \textbf{hipótesis antibíblica} que se tiene acerca de dicho concilio es que Dios utilizó el plural porque estaba en presencia de los ángeles, esto significaría que los ángeles son co-creadores del hombre lo cual es falso.\\
					La singularidad de la creación del hombre con respecto a las demás creaciones es que el resto 						de la creación la hizo solamente por medio de su Palabra pero para crear al hombre Él lo anunció claramente tratándolo como una una creación especial.\\
					Todo lo que fue creado fue para beneficio del hombre, por ello es que se relata con mayor detalle la creación del hombre comparado con los relatos de la creación del resto del universo.
					\begin{itemize}
					\item Características del nuevo ser creado\\
					La teoría de la evolución no explica la entrada de la razón al cuerpo físico del hombre que 						fue perfeccionado. Los evolucionistas crsitianos han dicho que el cuerpo físico del hombre 						fue evolucionando por medio de un animal pero que Dios fue quién le dió el espíritu, esto es claramente es falso dado el relato de Génesis.\\
					Desde el principio, el humano fue colocado en un lugar privilegiado de conocimiento. Somos 						diferentes porque el hombre fue el único creado a imagen y semejanza de Dios, algo distinto a 						lo demás en la creación ya que también nos otorgó caracterísitcas dvivinas como la santidad y el amor que no se ecnuentran en la demás creación.\\
					La imagen física del hombre no puede corresponder a la imagen física de Dios pues Él no	tiene una forma física, Dios es espíritu. La semejanza del ser humano con Dios se refiere a que el humano fue en un principio inocente (libre de culpa), fue hecho a imagen por sus facultades racionales, morales y espirituales que son las características que nos permiten relacionarnos con Dios.\\
Sin embargo, todo ello fue dañado por el pecado. La imagen de Dios en nosotros se desfiguró pero no fue destruida. Así los hombres tienen reservadas ideas acerca de la justicia. Todo ello está en la conciencia del hombre, contaminada por el pecado.\\
En su estado de inocencia el hombre iba a ser inmortal, su cuerpo estaba preparado para ello pero el pecado trajo la muerte (separación).\\
Dios tenía propósitos bien definidos para la cúspide de la creación, se le dió dominio sobre toda la naturaleza. Todo estaba a su gobierno para que Dios fuera glorificado. Dios creó al hombre de esa manera con el propósito de que el hombre le diera la gloria a Él.\\
En el paraíso no había fenómenos naturales, enfermedades, ni animales que le mataran. Existía un clima perfecto y una armonía completa aprovechando las bendiciones de Dios. Además de todos los recursos de los cuales disponía el hombre, Dios le dio una ocupación pues no era bueno que el hombre tuviera tiempo de ocio.
\item La ayuda idónea\\
	Hasta antes de Génesis 1:18, Dios había afirmado que todo lo que había creado era bueno pero en el versículo 18 Dios afirma que no es bueno que el hombre esté solo y que es necesario que le creara una acompañante que le sirviera de ayuda, una ayuda que tuviera distintas cualidades a las de él de forma que al tener ambos el mismo valor y distintas habilidades se pudieran complementar para llevar a cabo la tarea que Dios le había encargado el hombre.\\
	Después de la creación de la mujer a imagen y semejanza de Dios, les dio el mandato de que gobernaran y se multiplicaran hasta llenar la tierra. 
\item El pacto de obras\\
	Dios de manera voluntaria hace una alianza con el hombre y la mujer, una alianza que resaltaba la autoridad divina que Él tenía. Dios les prometió que si le obedecían en Sus mandamientos, no morirían.
\end{itemize}

\end{itemize}
\newpage

\item La caída (Génesis 3 - 5)\\
	En el caítulo 3 se le dada una detallada narración a cómo es que se comete el primer peado en la humanidad. Al igual que con los dos primeros capiítulos del libro, hay gente (algunos que hasta se llaman cristianos) que interpreta éste capítulo como una narración figurativa o un simple mito.
	\begin{itemize}
		\item El tentador\\
			El hecho de que se relate la presencia de una serpiente parlante es un argumento que se utiliza para afirmar que se trata de un mito, sin embargo, se conoce que Satanás es un ser sobrenatural y más aún, es capaz de tomar distintas formas. La aparición que ahora tiene Satanás en el Edén fue una permisión que le dio Dios pues Él siempre es soberano y no existe algo que lo tome por sorpresa.
		\item La tentación\\
			Satanás empieza su labor con el engaño, le habló a Eva con una aparente preocupación por su bienestar. Es por ello que construye la pregunta de forma que Eva se pregunte y dude sobre el mandamiento de Dios.
		\item La caída\\
			Lo primero que hizo Eva después de su charla con la serpiente fue ver el árbol y de esa forma es que sus ojos lo desearon, después tuvo el deseo de adquirir ese conocimiento que le había dicho la serpiente y finalmente tomó del fruto, lo comió y después le dio también a Adán.
		\item Resultados del pecado\\
			El hombre y la mujer desobedecieron de manera consciente el mandamiento de Dios y por ello vieron sus consecuencias inmediatas. 
			\begin{itemize}
				\item Culpa\\
					Después de comer del fruto ambos sintieron una profunda culpa y vergüenza.
				\item Muerte\\
					El lazo estrecho entre el hombre y Dios se rompe, se separó de su creador experimentando una muerte espiritual y una futura muerte física.
				\item Corrupción\\
					Toda la creación y en particular el cuerpo humano fue corrompido por el pecado dando entrada a la muerte.
				\item Expulsión\\
					Al romper el pacto que Dios había hecho con ellos por su propia voluntad es que Dios los castiga expulsándolos del Edén.
			\end{itemize}
		\item El pacto adámico
			\begin{itemize}
				\item Maldición para la serpiente\\
					Dios condenó a la serpiente a ser ahora un animal rastrero deforma que representaría lo despreciable.
				\item Maldición para Satanás\\
					Dios también maldice a la parte espiritual de la serpiente que era Satanás, es maldito como un mentiroso.\newpage
				\item Maldición para la mujer\\
					Fue maldita con dolores de parto desde ese momento en adelante como un recordatorio de su participación e iniciación en la caída del hombre.
				\item Maldición para el hombre\\
Fue maldito con la lucha constante que tendría por conseguir recursos para su sobrevivencia. Maldito a trabajar la tierra ahora maldita para conseguir alimento.
			\end{itemize}
		\item Génesis 4 (El primer homicidio)\\
			Caín y Abel siendo hijos de Adán y Eva ahora tenían distintos trabajos. Caín trabajaba la tierra mientras que Abel pastoreaba ovejas. Seguramente Caín tuvo una mala actitud ante la ofrenda que presentó a Dios pues Él conoce cada uno de nuestros pensamientos. Esto se ve confirmado en el enojo que tiene en contra de su hermano, su ofrenda y en contra de Dios mismo. Es por ese enoj que Caín termina matando a Abel.
		\item Génesis 5\\
			Solamente se incluyen a las generaciones de Adán y es importante resaltar el hecho de que ya existían otras personas e incluso pueblos que los narrados pues Caín temía que le fueran a matar o dañar.
	\end{itemize}

\item El Diluvio (Génesis 6-9)\\
Matthew Henry define a los hijos de Dios como aquellos que creían verdaderamente en Dios mientras que las hijas de los hombres eran hijas de incrédulos.\\
J. MacArthur discrepa y dice que los hijos de Dios son los ángeles caídos que rndaban la Tierra, ángeles que pecaron, que no guardadron su integridad. Éstos ángeles caídos habrían tenido relaciones sexuales sin haberse casado pues los ángeles no se pueden casar.
C. I. Scofield dice que los hijos de Dios se atribuye solamente a los ángeles pero no es así. También niega que los ángeles puedan tener sexo ya que nunca se mencionan ángeles femeninos.\\
Henry Morris va más allá pues traduce la frase como seres creados directamente por Dios i.e. Adán, Eva y ángeles (hijos de Dios). Estos seres espirituales (ángeles caídos) convivieron directamente con los hombres.\\
Los comentaristas del texto hebreo dicen que los hijos de Dios son los gobernantes, mientras que las hijas de los hombres son los plebeyos, sin embargo, ésto no tiene sustento bíblico.\\
La frase ``tomaron para sí mujeres..'' significa que las tomaron por esposas, ésto nos niega que sean ángeles ya que ellos no se casan.A pesar de la increíble dificultad para interpretar éste capítulo no nos debe de preocupar ni desanimar ya que son aspectos vanos que no tienen relevancia doctrinal.\\
No podemos tener una interpretación segura de éste pasaje, recordando lo importante de la Palabra de Dios que menciona Deuteronomio 29:29.\\
El versículo 6 dice que Dios se arrepintió. La palabra ``arrepentir'' en el hebreo ``nacham'', una de las traducciones puede significar arrepentimiento pero de acuerdo al contexto bíblico, una mejor traducción sería ``dolor''.\newpage
En el versículo 8 aparece Noé que halló gracia ante los ojos de Jehová, a pesar de que se refiere a él como varón justo aún así Noé era pecador. Quiza fue de los pocos hombres que buscaban a Dios en su tiempo, él era distinto a los de su generación. Gen 5:24 habla de Enoc que Dios caminó con él, al igual que se refiere con Noé. \\
Dios diseña y ordena la construcción de un arca a Noé y decreta que va a trae un diluvio en Génesis 6:14.\\ 
De acuerdo a la cronología del diluvio incluída en la Biblia de estudio de MacArthur, el diluvio duró 1 año y 10 días.\\
\textbf{Pacto noético}\\
Es narrado al final del diluvio, el pacto que Dios hizo con Noé, eterno,  que no volvería a destruir con agua.
\item Las Naciones (Génesis 10-11)
En el capítulo 11 el hombre empieza a tomar en cuenta la posiblididad de quedarse juntos y no de reproducirse y expandirse tal como Dios se los había ordenado. Deciden ya no continuar con esa ordenanza de Dios ya que habían encontrado recursos para establecer una comunidad.\\
Génesis 11:4, se debe interpretar de manera figurada que la torre iba a ser muy alta  que pareciera que tocara el cielo. Eran edificios que tenía forma de cono truncado, su cúspide era plana ya que la ocupaban como observatorio, así los habitante antiguos de Mesopotamia se dedicaban a estudiar los astros.\\
Dios vió que el hombre estaba decidido a volver a desobedecerlo, así fue que Dios su lenguaje con el propósito de que el hombre se volviera a dispersar. Babel significa confusión aunque muchos lo interpretan como Puerta de Dios, de allí viene el nombre de Babilonia. Dios ocupó la confusión de lenguas para que el hombre cumpliera con su misión y a la vez fue un juicio.

	\end{enumerate}

		\end{subsubsection}
		\begin{subsubsection}{La historia patriarcal (Génesis 12-50)}
			\begin{enumerate}
				\item Abraham (Génesis 12-24)\\
					La depravación del hombre estaba nuevamente en todos los pueblos de la Tierra y se observa cómo Dios disciplina al hombre, sin embargo, por su misma naturaleza pecaminosa el hombre vuelve a desobedecer y Dios vuelve a disciplinar al hombre continuando con el mismo ciclo.\\
Existen 3 religiones que consideran a Abraham como patriarca dada la importancia de lo que hizo, los musulmanes, los judíos y nosotros los crisitanos.\\
	\textbf{El pacto abrahámico}\\
					Enfáticamente se narra en las Escrituras que es Dios quien inicia el proceso de redención. Abram en ningún momento se resistió a obedecer a Dios, por ello se afirma que Abram ya tenía un conocimiento de quién era Dios.
					\begin{itemize}
						\item Promesas que incluye el Pacto Abrahámico
							\begin{itemize}
							\item La promesa de la tierra\\
							Jehová le había dicho a Abraham que se fuera (Génesis 12:1)\newpage
							Las condiciones de Dios a Abraham fueron que tenía que dejar el lugar en el que había nacido, renunciar a su familia lo que implica renunciar a sus privilegios ya que era primogénito y todo por aceptar algo que él desconocía pero lo hizo por fe. Tenía confianza incondicional en el Dios que le había hehco un llamamiento.
					\item La promesa de la descendencia (Génesis 12:2)
				\item La promesa de la redención (Génesis 12:3)
			\end{itemize}
	\end{itemize}
				Aquí se ve claramente la doctrina de la elección. Dios elige por gracia a Abram y no porque haya tenido algún mérito.\\
							Dios, por medio de Abram, inaugura una nueva relación con el resto de la humanidad, la gente sería bendita o maldita dependiendo de la relación que tuvieran con su familia.\\
							Todos los descendientes de Abraham fueron conocidos como hebreos ya que Abraham quiere decir ``Descendiente de Heber''.\\
							En Génesis 15:6 vemos a Abram como un claro ejemplo de la justificaión por fe.\\
							A pesar de que Abraham ya era viejo y sin hijos le creyó a Dios, en los versículos 19-21 se narra el ritual del pacto.\\
En el capítulo 17 vemos que Dios le cambia el nombre a Abraham que significa ``Padre de multidudes''. En cada pacto Dios deja una señal, en este caso, la circuncisión.\\
En Génesis 18:1-2 se narra la aparición de tres varones. En el resto del capítulo se narra que Lot se distancía de Abraham. Dios salva únicamente a Lot de los habitantes de Sodoma y Gomorra como juicio de su perversión.\\
En el capítulo 21 vemos el cumplimiento de la promesa de la descendencia de Abraham, nació Isaac como hijo de Abraham y Sara, el relato parece ser muy sencilllo dada la angustia que precedió a este acontecimiento hasta parece un nacimiento común y corriente. Se demuetra que nada es imposible para Dios ya que Sara era vieja y esteril y Abraham era también viejo.\\
En Génesis 22:2 vemos la gran prueba de fe de Abraham ya que es la acción humana que más se acerca a la dádiva del primogénito que vemos que hace Dios en el Calvario cuando él ofrece a su hijo tan esperado. Isaac era la única posibilidad de descendencia de Abraham, él responde sin poner excusas ni pedir explicaciones, seguramente Isaac ya era lo suficientemente grande para razonar la situación ya que era lo suficientemente fuerte para cargar la leña.\\
Abraham ofrece el animal como holocausto en lugar de su hijo y aprendió que Dios prueba pero Dios provee (Jireh).\newpage
En el capítulo 24, se narra la elección de la esposa para Isaac. Abraham hace jurar por Jehová a un siervo de mucha confianza quien habría de ir a buscar la esposa para Isaac. La promesa era que fuera de su parentela. El siervo observa a las doncellas que tienen los requisito pedidos y en cuanto la encuentra explica el propósito de su búsqueda. Con esto se garantiza una descendencia que entra dentro de los términos del pacto. Abraham es considerado como patriarca ya que mostró su fe en el único Dios verdadero, nos mostró que esa fe es el único medio por el cual podemos ser justificados. Abraham no vio a se descendencia que fuera como la arena pero en capítlo 22 vemos que muere con la confiaza de que esa promesa se cumpliría.
\item Isaac\\
	Nació en la vejez de sus padres, mostró Dios su omnipotencia ya que cumplió a pesar de las circunstancias humanas. Abraham tuvo otros hijos pero Isaac fue el hijo de la promesa, no Ismael. Se narra que Rebeca era estéril y que después de 20 años no podía concebir, eso era considerado como una maldición. Rebeca finalmente concibe gracias a la oración de Isaac y consulta a Jehová dadas las circuntancias tan conflictivas que tuvieron. Dios responde diciendo que había dos hijos en el vientre de Rebeca revelándole que sus 2 hijos estaban en conflicto desde su vientre y que de cada uno de ellos saldría una nación. Aquí Dios establece que el mayor serviría al menor, cosa contraria a lo que se acostumbraba dada la posición de la primogenitura.\\
	Los nombres dados a sus hijos hacen mención a sus atribustos. ``Esau'' se le nombra al velludo y  ``Jacob`` al suplantador por haber naciedo tomado del talón de su hermano.\\
	Isaac estaba a punto de abandonar la tierra de Canaan debido al hambre de la región pero Dios le confirma a Isaac el pacto que tenía con él. Isaac debía de permanecer en la tierra prometida a pesar de la hambruna y debía de confiar en Dios. Dios le dice que siempre estaría con él y toda la bendición que estaba preparada para él, son las mismas bendiciones que Dios le había dado a Abraham.\\
	Isaac se sometió y fue obediente, construyó un altar e invocó el nombre de Jehová.\\
	El primer conflicto que surge en el pacto es que Esaú cuando ya es adulto se casa con dos mujeres de Canaan. Este matrimonio hace peligrar el pacto ya que se había mezlcado con otro pueblo, esos matrimonios estaban en contra de las costumbres patriarcales.\\
	En Génesis 25:27-34 vemos cómo Esaú rechaza su primogenitura. Aparentemente entre los dos hermanos había una constante competencia. Esuáu siendo el hijo preferido de su padre mientras que Jacob era el hijo preferido de su madre.\newpage
	Esaú regresa de un intento de caza sin logro y con mucha hambre, Jacob aprovecha la oportunidad de que le diera la primogenitura a cambio de que le diera de su guiso rojo. Ambos conocían el gran valor que tenía el puesto de la primogenitura, al primogénito se le daba el título de sucesor del padre pero todo ello lo desechó Esaú al momento de vender su primogenitura. La primogenitrua les daba derechos especiales como el doble de los bienes al momento de la muerte de su padre, el derecho a ser el sucesor y que el mesías vendría de la descendencia del primogénito.\\
	Todo ello lo cambió Esaú por un guiso rojo, ésto habla de un caracter inestable que tenía Esaú, las caracterísitcas de la primogenitura eran sumamente importantes para el pueblo en esa época.\\
	Jacob se presetna con su padre ciego y al presentarse como Esaú empieza el engaño. Le presenta el alimento que le había pedido a Esaú. Rebeca escuchó el pedido que Isaac le dijo a Esaú y fue quién le ayudó y le dijo a Jacob para que podía recibir la bendición.\\
	Isaac le pregunta por la rapidez de la comida pero alega que el Dios de Isaac actúa a su favor. La siguiente objeción fue que en realidad fuera Esaú, siendo Isaac ciego no lo podía identificar visualmente  pero utiliza su tacto para confirmar que era velludo como Esaú, Rebeca ya había previsto esta situación al cubrir los brazos de Jacob con vello y vestirlo con la ropa de Esaú, por ello es que Isaac queda satisfecho a pesar de dudar de su voz. Es entonces cuando Isaac come y bendice a Jacob.\\
	Una vez pronunciada la bendición era irrevocable e intransferible, la conexión fue hecha.\\
	En Génesis 35 se narra la muerte y sepultura de Isaac\\
	La edad de Isaac ya era muy avanzada y en el epitafio menciona que había tenido una vida importante. Isaac es sepultado por sus hijos en la sepultura familiar.

\item Jacob\\
	Dios le permite a Jacob que con astucia se apodere de la primogenitura de su hermano. Siendo amenazado de muerte por su hermano al robarle su bendición huye. Dios confirma su pacto con Jacob mientras él va a Aram. Cuando Jacob logra aproximadamente 70 km él tiene su primer encuentro con Dios, se acuesta utilizando una piedra como cabecera para dormir y en un sueño Dios le confirma su pacto. Hay 5 promesas que en ese momento le fueron hechas a Jacob y promesas que podemos creer que Dios las cumple para todos aquellos que hemos de creer, Gpenesis 28:15. 
	\begin{enumerate}
		\item``Yo estoy contigo'', promesa en presente
		\item ``Yo te guardaré'', el cuidado de Dios es una fuente de seguridad, está siempre presente sustentando a toda su creación.
		\item ``Yo te haré volver'', Dios obrará par que pueda volver a la tierra prometida. Formará en Jacob las condiciones que le permitirían reestablecer su relación con su hermano.
		\item ``Yo no te abandonaré'', Él va a estar de nuestro lado para llevar a cabo la tarea, nos dará su dirección.\newpage
		\item ``Yo te lo he dicho'', la promsea venía directamente de Dios mismo. Él velará por el cumplimiento de sus promesas. Sus palarbas no caducan.
	\end{enumerate}
	Así podemos concluir que siempre podemos confiar en las promesas que Dios hizo, Jacob tenía grandes garantías para que pudiera seguir adelante con su tarea. El pacto no se hizo con Jacob por sus atributos.\\
	Dios determina a quién el elige para su propósito epsecial así como Él mismo lo menciona en Malaquías 1:2-3\\
	Jacob continua su viaje hasta llevar con los arameos y llega con Labán.\\
	Jacob, el engañador, iba a ser ahora engañado por su tío trabajando 7 años y le dan por esposa a Lea lo cual lo forza a trabajar otros 7 años por Raquel.\\
Finalmente Dios le ordena que se regrese a su tierra. Jacob forma una familia numerosa con dos esposas, dos concubinas, 12 hijos y una hija.\\
De Raquel solamente tuvo a José y a Benjamín. Se mencionan en la Biblia a otras hijas de las cuales se desconoce quiénes sean sus madres. Se narra la salida de la salida de Jacob con su familia de la tierra de Labán. Dios le ordenó a Jacob que regresara a la tierra prometida, sus esposas deciden separarse de sus familias y seguir a Jacob. La decisión de salir a la tierra prometida es ahora una decisión familiar.\\
El regreso de Jacob a Canaán se narra que no fue sencillo, la riña entre Esaú y Jacob ponía en peligro la supervivencia de su propia familia. De manera inesperada se narra que Jacob lucha con Dios quien al final de la lucha le cambia el nombre de ``Jacob'' por el de ``Israel'', mostrando que aunque Jacob siempre había sido un engañador finalmente lo reconoce como Israle que significa ``el que lucha'', ese sería el nombre del resto de su descendencia.\\
En el antiguo Testamento cuando se habla del Ángel de Jehova se refiera a Dios Hijo. El propósito de este evento fue el de animar a Jacob por la preocupación que tenía del póximo encuentro con su hermano.\\
Jacob se encuentra finalmente con Esaú en el capítulo 33, el reencuentro de parte de Jacob es muy cauteloso y respetuoso pero Esaú al ver a Jacob va y lo besa en muestra de afecto y ambos lloran después de 20 años de separación. Jacob muestra el cambio de su ser, Esaú al igual no era el mismo, había cambiado su odio en afecto.\\
Jacob reinicia su viaje rumbo a Siquem y se pregunta sobre cómo habría de defenderse ante el peligro que se le presentaba con su familia. Dios le dice a Jacob que vaya a Betel y se le  volvió a aparecer, lo bendijo y le dio el mandamiento de que creciera y se multiplicara al igual que le promete la tierra que le había prometido a Abraham.\\
José desde muy joven criaba ovejas y le daba información a su padre acerca de lo que hacían sus hermanos. Jacob mostraba visible y notablemente su amor por José por su conducta fiel y correcta. Tanto José como Benjamín ya eran huérfanos pues Raquel ya había muerto.\newpage
\item José (Génesis 37-50)\\
	Apacentaba las ovejas con sus hermanos, tenía muchos privilegios ya que era el hijo favorito de su padre. Tenía sueños proféticos constantes, ello le daba un puesto prominente dentro de su familia. La interpreteación de estos sueños era un don de Dios. Jacob conocía por sus propias experiencias el poder que tenía los sueños dados por Dios. Los sueños de José hablaban de su prosperidad sobre sus hermanos lo cual le generaba conflictos con ellos ya que se resistían a subordinarse a José.\\
Sus hermanos piensan en matarlo pero deciden venderlo como esclavo por 20 piezas de plata. Estos mercaderes llevan a José a Egipto donde Dios le dio gracia frente a sus amos que lo iban a ascender. Pronto José fue nombrado mayordomo de la casa de Potifar.\\
	José rechazó las invitaciones de la esposa de Potifar pues para él era inconcebible su propuesta. José reconoce que estaría pecando en contra de Dios y aún así es condenado y lo encarcelan.\\
	Después es liberado al interpretar el sueño de faraón y es nombrado en un sitio de honor de Egipto. Por José, Egipto prospera y por medio de sus sueños predice un tiempo de hambruna por lo que guarda alimento para ese tiempo. Se casa con una egipcia y tiene dos hijos, Manasés y Efraín. Por esta hambruna es que José se volvería a acercar con su familia de Canaán pues por su predicción Egipto fue el único pueblo que prosperó durante el tiempo de hambruna y sus hermanos fueron a Egipto en busca de alimento.\\
	José se les revela en Génesis 45:1-3 y después pide que le traigan a su padre.\\
	En Génesis 49:33 se describe la muerte de Jacob. Después de ello los hermanos de Jośe temen por lo que les haría ahora que su padre muriera pero José sigue mostrando el perdón que ya tenía.\\
	José más adelante con fe pide que sus huesos los regresen a Canaán pues confía en el pacto de Dios. Génesis termina con el relato de que José muere, él vivió 110 años y siempre vivió confiendo en la provisión y en las promesas de Dios.

			\end{enumerate}

		\end{subsubsection}
	\end{subsection}
	\begin{subsection}{Temas clave}
 Los pactos son de importancia primaria pues así podeos comprender los actos de Dios en ellos, generalmente la conveción es que son 7 pactos. Un pacto es un convenio que expresa la relación que tiene Dios con Su pueblo.
			 \begin{itemize}

				 \item El pacto edénico (Génesis 1:28)\\
				Señal: El edén (Génesis 2:15)

			\item	El pacto adámico (génesis 3:14)\\
				Señal: La Tierra (Génesis 3:17)

			\item	El pacto Noético (Génesis 9:9)\\
				Señal: El arco iris (Génesis 9:13)

			\item	El pacto Abrahámico (Génesis 15:18)\\
				Señal: La circuncisión (Génesis 17:10)


			\item	El pacto Mosaico(Éxodo 20)\\
				Señal: La Ley (Éxodo 20)

			\item	El pacto Davídico ($2^o$ Samuel 7:12)\\
				Señal: El trono ($2^o$ Samuel 7:13)\\
				Garantizaba que tendrían un reino con un trono eterno, Cristo es el que reinará.


			\item	El nuevo pacto (Jeremías 31:31)\\
				Señal: La sangre (Isaías 53)\\
				La sangre representa la muerte expiatoria de Cristo para salvación de la humanidad.
			

		\end{itemize}
	\end{subsection}
\end{section}

%\end{document}


