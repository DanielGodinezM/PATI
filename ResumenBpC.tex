%\documentclass[12pt]{article}
%\usepackage[margin=1.0in]{geometry}
%\usepackage{enumerate}
%\usepackage[spanish]{babel}
%\usepackage[useregional]{datetime2}
%\begin{document}
\begin{section}{Creación: créalo o no: Génesis 1:1}
	El primer capítulo del libro empieza después de que al autor da una breve introducción acerca del propósito que tiene el libro así como la necesidad que hay de presentar una fuerte defensa en contra de las corrientes del pensamiento que buscan interpretar el inicio del universo con base en hipótesis que carecen de un verdadero fundamento.\\
	El hermano MacArthur deja claro dentro de la introducción que el debate entre lo que dice la Biblia (verdad) y lo que dice la mente humana en realidad es una cuestión acerca de gente que no quiere reconocer la veracidad de la Palabra de Dios sino que busca en estos argumentos tales como la evolución, una forma de poder excusarse de la tremenda realidad de que se tenga a un ser creador tal y como lo narra la Biblia. Dentro de estos argumnetos se hace una aclaración en extremo importante que es que la verdadera ciencia no contridce ni contradecirá lo que dice la Biblia, mas bien, son las hipótesis pseudocientíficas las cuales se utilizan para poder excusar su fe en el ateísmo y en el conjunto de sus dogmas.\\
	\\
Después de haberse centrado principalmente en la evolución, el capítulo uno comienza hablando acerca de la teoría del Big Bang. de dicha teoría no solamente se habla de las extrañas inconsistencias físicas que conlleva el asumir dicha teoría sino que además ello implica que el universo no es más que materia interactuando de manera mecánica sin posibilidad de interacciones y/o espirituales.\\
El gran peligro de la gran aceptación de estas teorías es el hecho de ``sustentar''el hecho de que la humanidad y la vida humana carecen de sentido ya que es consecuencia exclusiva de fenómenos naturales así como el inicio del universo y de la vida. Esa corriente de pensamiento tiene como consecuencia lo radical que se tiene en la actualidad en cuanto a movimienos a favor de la vida animal, que le da un valor semejante a la vida humana así como la poca importancia de la vida humana como para apoyar el aborto.\\
Examinando la creación de manera espontánea a través del azar incluso de manera secularnos permite observar el gran salto de fe que se ocupa para poder crer que el azar fue el responsable del inicio del universo y de todo lo que conocemos, además, lleva por consecuencia el hecho de que con esos argumentos el ser humano no es más que otro ser del reino animal que llegó a desarrollar intelecto y razón sola y exclusivamente por consecuencia de la evolución y de un proceso de miles y miles de años.\\
\\
A lo largo de este capítulo se hace un gran énfasis no solamente en lo que dice Génesis 1:1 sino la gran importancia que tiene  las enormes consecuencias que conlleva. Es por ello que la teoría de la evolución, en un principio, entra en una gran diferencia al proceso de la creación en Génesis, y no solamente eso sino que además son completamente incompatibles entre sí.\\
La teoría de la evolución trae consigo implicaciones acerca de la naturaleza de la vida humana que además de ser profundamente irracionales e improbables, resultan estar fundamentadas en el empirismo y lo qe el hombre cree que puede ser cierto mientras que los primero relatos de Génesis son extremedamante claros acerca de la creación.
\newpage
En ningún otro lado se puede encontrar un relato de la creación tal y como se tiene en Génesis, empezando por los textos que tienen otras religiones acerca de la creación, en ellos se puede encontrar un gran rastro de inconsistencias así como las deidades tan pobres y llenas de mitos que son ``responsables'' de la creación del universo. Por otro lado, se tiene el relato de las teorías científicas modernas acerca del comienzo del universo, una rama de la ciencia en donde se abandona el rigor científico para poder hacer conclusiones naturalistas acerca cómo es que un proceso meramente aleatorio dio a luz un universo ordenado con vida y más aún, vida racional.\\
\\
El capítulo concluye en sus últimas páginas de mnera tajante sobre la importancia que se le debe de tener ala verdad relatada en la Biblia, algo que muchas personas que incluso se hacen llamar ``cristianas'' han abandonado con el propósito de hacer encajar la Biblia con el desarrollo científico actual que incluso ni ha sido demostrado, con el claro peligro de que en un futuro que surja una nueva teoría científica que corrija y/o derrumbe por completo la teoría evolucionista entonces sea necesario que vuelvan a cambiar su postura acerca de la creación ya que el hacer encajar la Biblia con dichas teorías lo único que provoca es tener un fundamento basado fuera de la verdad y dentro de los ``quizas'' y los ``probablemente''. \\
Génesis 1:1 debe de ser entendido tal y como es narrado en la Biblia, todo el relato de Génesis es un relato histórico que escribió Moisés siendo inspirado por Dios con el propósito de que conociéramos cómo fue que la historia comenzó y cuál es el propósito que Dios tiene para Su creación. El darse la libertad de intepretar la creación como una fábula o un mito que solamente ilustra de manera vaga el proceso de creación carece de buenos argumentos ya que el autor de Génesis en ningún lado escribió algo que siquiera sugiriera que el realto no es literal en su totalidad, el querer cambiar los días de la creación en miles de años o en eras geológicas carece buena hermenéutica e incluso carece de carácter científico.\\
\\
Dentro de los principales adeptos y defensores de la evolución se tiene ciertamente a grandes personalidades dentro de la ciencia y la filosofía que lo único que transmiten es su propia fe en el azar con supuesta base ciéntificas y ciertamente han inlfuenciado a un gran número de personas haciéndoles creer que el origen de la vida por medio de la evolución es una verdad científica comprobada, dentro de dichas personalidades se puede encontrar a Carl Sagan, Friedrich Nietzche y Karl Marx. Algo que cabe destacar es que aunque los creyenetes en el azar muestren un carácter objetivo, estrictamente ciéntifico y carente de dogmas o fe, precisamente se sumergen dentro de esta teoría a pesar de la falta de una buena explicación para el inicio de universo, ciertamente les es más fácil creer que el carácter estrictamente mecánico y físico del universo que creer en un Dios creador del universo ya que ello implica que Dios nos creó con propósito y más aún, Dios es Señor sobre toda Su creación, una verdad que simplemente derrumba toda su corriente de pensamiento.\\
\\
En particular, las ideas que tuvo Nietzhe como fuerte combatiente del cristianismo fueron de gran influencia en Alemania ya que de manera ``objetiva'' la evolución nos lleva a pensar en el hecho de que la supervivivencia del más fuerte es la ley que debe de regir nuestra civilización, una idea que influyo ciertamente en el desarrollo del nazismo y además de eso, en la actualidad es lo que ha influido el gran pensamiento del ``relativismo'' ya que si la evolución fue meramente un fenómeno de supervivencia, la moral y la ética carecen de objetividad.

\end{section}
\newpage
\begin{section}{¿Cómo sucedió la creación?}
	Desde el inicio del Génesis la Biblia es muy clara acerca de que la creación fue por medio de Dios y además que Él lo creó todo a partir de nada, solamente bastó que utilizara Su palabra para que todo fuera creado mientras que, por el contrario, la postura evolucionista afirma que todo ser viviente proviene de uno anterior, el cual tuvo modificaciones genéticas que los fueron cambiando en un proceso de miles de millones de años además de que el universo tal cual como lo conocemos así como el resto de los seres vivos, seguimos en constante evolución.\\
	Después, se palantea la gran disyuntiva que existe entre el catastrofismo y el uniformismo. Empezando por el uniformismo se tiene que es una teoría que fue diseñada desde sus fundamentos con la hipótesis de que el relato Bíblico es incorrecto. Con ello en mente es que plantean que los procesos naturales y geológicos por los que pasa la Tierra ocurren con cierta periodicidad constante, ésto les permite observar grandes formaciones de piedra como el Gran Cañón y hacer cálculos bajo ésta hipótesis para determinar el tiempo que tiene que se hicieron dichos sedimentos. Por otro lado se tiene la posición del catastrofismo que menciona que dicha periodicidad no es regular sino que durante la historia de la Tierra han pasado distintas catástrofes que alteran la periodicidad con la que se forman dichos sedimentos sobre las rocas. El catastrofismo parece estar de acuerod con el relato bíblico ya que, por ejemplo, un diluvio universal podría ser capaz de alterar la periodicidad de ocurrencia de dichos sedimentos, tal y como lo mencona el catastrofismo. Ésto además de que se han econtrado numerosos fósiles de distintas especies a través del mundo así como fósiles marinos sobre la superficie de montañas cuya mejor explicación es el relato del diluvio que viene narrado en Génesis.\\
	\\
	A pesar de que la teoría uniformista carece incluso de carácter y rigor científico, hay aún gente que se ha dedicado a tratar de hacer encajar el relato bíblico de Génesis con lo que dice el uniformismo, de forma que alteran el verdadero mensaje del libro y distorsionan la Palabra de Dios.\\
El siguiente aspecto a tratar acerca del relato de Génesis es que se observa que la creación tanto del hombre como del resto del universo fue en un estado pleno de madurez. Dios no creó seres infantes ni huevos que después darían coo resultado animales ni semillas sino una creación madura. Se menciona mucho el debate actual que existe sobre las implicaciones de esto tales como la antomía de Adán y la estructura que tendrían las plantas y animales que fueron creados en un principio pero todos estos debates no pasan de ser especulaciones ya que en la Biblia no se relata ni se detalla, ésto es importante de recalcar ya que hay gente que llega incluso a desarrollar un gran celo por sus teorías a pesar de que no son Palabra de Dios.\\
Posteriormente el hermano MacArthur empieza a hacer una crítica sobre el trabajo que ha realizado Hugh Ross, un cristiano evangélico de Estados Unidos que tiene estudios en Cosmología. MacArthur hace un énfasis en el sumo cuidado de saber si estamos adaptando las teorías científicas a lo que dice la Biblia o si estamos poniendo a las teorías científicas sobre la luz de la Biblia. El trabajo de Ross es precisamente un intento de armonizar la teoría del Big Bang con el relato bíblico con el cual argumenta que los días de lac reación son en realidad miles de años por los cuales tardaron en se creara la Tierra con vida.
\newpage
A pesar de dichas creencias por parte de Ross, él afirma la inerrancia de las Escrituras e incluso ha mencionado que la revelación que tenemos acerca de la creación por medio de la naturaleza es testigo y misma revelación por parte de Dios para que podamos entender cómo fue el proceso de la creación, incluso hasta mencionar que el estudio de la naturaleza y del universo puede considerarse como el libro 67 de la Biblia.\\ 
\\
La gravedad de las conclusiones a las que llega Ross acerca de la revelación por medio de la naturalez es grave ya que es racalcado constantemente en la Biblia su superioridad como Palabra de Dios y el hecho de que es una revelación suficiente para que le conozcamos mientras que la naturaleza no lo es.\\
El argumento principal para refutar la idea de que los días de la creación duraron miles de años es que no hay en la Biblia ni un versículo que siquiera lo mencione además de que desde el tiempo de Moisés hasta nuestros días la interpretación del pasaje siempre ha sido de días literales.\\
Ross incluso llega a mencionar como un argumento a favor de su postura el hecho de que la creación fue algo sumamente perfecto que hubiera requerido de mucho tiempo para que todo tomara su forma y su lugar, sin embargo, el poder de Dios es algo que nosotros no podemos comprender, nada en el texto hace alusión a la evolución o a periodos grandes de tiempo.\\
Debido a la falta de evidencia que proporciona el mismo texto bíblico a favor de sus teoría es que hay gente que descarta este pasaje como un relato y lo interpreta como un mito de creación ilustrativo pero que en realidad no fue como ocurrió, esta postura está también incorrecta por el contexto del libro y dado que todo el libro de Génesis se dedica a relatar las historias del principio de la historia del universo.\\
\\
A través de la historia han surgido distintas teorías sobre el inicio del universo y muchas de ellas han sido moldeadas conforme la ciencia va progresando, es por ello que debemos de permanecer fieles y constantes en la Biblia conociendo su inerrancia pues sabemos que la revelación que Dios nos entregó es verdad y Jesús mismo confirmó la eternidad de Su palabra en Mateo 24:35, cosa que nosotros somos capaces de observar a pesar de los miles de años que tiene la Escritura.\\
Como fieles creyentes tenemos la fe de que así como el universo fue creado algún día llegará a su fin y la Palabra de Dios seguirá vigente en el cielo nuevo y en la tierra nueva.
\end{section}
\newpage
\begin{section}{Luz en el día primero}
	Después de todo el argumento que se tiene sobre la duración de los días en la creación, desde el primer día se queda estipulado la duración que tiene. En Génesis 1:5 se narra que después de haberlas creado, a la luz le llamó Día y a las tinieblas Noche. Éste versículo acaba con la expresión ``\ldots fue la tarde y la mañana un día'', expresión con la que acaba cada día de la creación. Es por ello que podemos concluir que cada uno de los días de la creación tuvo la misma duración que los demás.\\
	A pesar de que la palabra día sí es usada en la Biblia en ocasiones como un periodo largo de tiempo, siempre que se utiliza junto con un número del día se refiere exacta y precisamente a un día con tarde y noche.\\
	\\
Otro argumento que se presenta para la duración no literal de los días en la creación es el hecho de que Agustín de Hipona en sus escritos detalló su creencia de que la creación no había sido en 6 días literales. Sin embargo, la creencia era completamente contraria a los miles de años que se proponen, él pensaba que el fenómeno de la creación era algo sumamente extraordinario que posiblemente nunca llegaremos a comprender y debido al poder de Dios, el creía que toda la creación había sido hecha en un sólo instante. Ésta creencia en parte la heredó debido a filósofos seculares de la época. Además de ello, Agustín incluye un capítulo entero dentro de su texto ``La ciudad de Dios'' con el propósito de hacer una crítica a aquellas personas que alegaban una vejez mayor al universo, diciéndoles incluso que hablan de lo que piensan pero no de lo que saben.\\
A pesar de las creencias incorrecta de Agustín, es importante que sigamos analizando la importancia de los 6 días de la creación. Dios no hizo la creación en 6 días porque necesitara tiempo para descansar ni porque no lo pudiera hacer en tan sólo un instante. Al final del proceso de la creación podemos observar que Dios se tomó los 6 días para mostrarnos un modelo de trabajo y descanso el cual nosotros debíamos de seguir para poder tener una vida saludable y balanceada.\\
Cada uno de los días de la creación tuvo su importancia ya que en cada uno se llevóa a cabo la creación de distintos aspectos de la creación.\\
\\
En especial en el primer día se narra la creación, del tiempo, la materia y la luz. La creación del tiempo es un aspecto sumamente clave ya que Dios no habita en el tiempo, su divinidad opera fuera del tiempo. Sin embargó, el tiempo fue lo primero con lo que decidió someter a Su creación. Para nosotros es sumamente complicado entender a un Dios eterno ya que todo en nuestro univeros está gobernado por el tiempo, todo envejece y muere pero Dios es eterno y se mueve a través del tiempo como le place. Él es soberano sobre el tiempo así como lo es del resto de la creación.\\
Igual durante el primer día es que Dios crea la materia, toda la metria fue creada durante el primer día, sin embargo, no sabemos que frma tenía dicha materia pues la narración de la creación de los planetas y las estrellas ocurre hasta el día 4.\\
Por último pero no menos importante, se menciona que durante el primer día también Dios creó la luz. La luz es quien representa con mayor claridad la gloria de Dios ya que incluso Él mismo se llama luz en $1^{a}$ Juan 1:5.
\newpage
En Génesis 1:2 se mencionan 3 frases en particular para describir el estado de la metria, dice que la tierra:
\begin{enumerate}
	\item Estaba desordenada y vacía.
	\item Las tinieblas estaban sobre la faz del abismo.
	\item Es espíritu de Dios se movía sobre la faz de las aguas.
\end{enumerate}
En la forma en la que se describe a la tierra es como la desolación de un desierto, sin vida y sin forma definida.\\
Dentro de las teorías que envuelven a este versículo sobresale una llamada ``teoría del salto'' la cual afirma que en el versículo 1 la Tierra fue creada de manera perfecta con vida y que llegó a quedar desolada y vacía debido a la posterior caída de Satanás. Después de ello es que la vida pereció y Dios decidió volver a crear vida.\\
\\
La teoría del salto surgió como una forma para poder explicar de mejor manera la aparición de fósiles antiguos y, una vez más, hacer encajar el relato bíblico con las teorías científicas, esta teoría es menos acpeta ya que el tener esta suposición conlleva como resultado grandes problemas teológicos tales como el reconocimiento de Dios de que su creación era buena cuando según esta teoría ya había maldad en ella y también el hecho de que la muerte entró en la creación hasta el pecado inicial con Adán y Eva, entre otros ejemplos.\\
En realidad debe de intepretarse tal y como lo dice el texto afirmando que ese estado que narra el versículo 2 fue el estado inicial de la creación.\\
La frase con la que acaba el versículo 2 indica la vigilancia y cuidado que tenía Dios sobre Su creación. Esta aprte del pasaje recalaca el deseo de Dios de tener actividad directa sobre Su creación, a partir de aquí en adelante, el proceso de creación es narrado como si fuera por un observador en la tierra.\\
\\
A pesar de la gran complicación que conlleva el explicar el comportamiento y la naturaleza de la luz, en el versículo 3 se indica claramente cómo es que Dios creó la luz con Su palabra. La luz tuvo un papel importante ya que además marca la diferencia entre día y noche.\\
Para Génesis 1:4 se dice que vio Dios que la luz que había creado era buena, expresión que se repite a lo largo del relato de la creación pues es producto de un Dios bueno.\\
Con esta expresión junto con el pasó de la noche en Génesis 1:5 marca el final del gran primer día de la creación.
\end{section}
\newpage
\begin{section}{Él demarcó los fundamentos de la Tierra}
Los primeros 3 días de la creación fueron de suma importancia pues en cada uno de ellos, además de la creación, se narra un tipo de separación. Para el día segundo que se estudia en el libro a través de este capítulo, se narra la separación de dos grande sdepósitos de agua, las aguas de los cielos y las de los mares dando como resultado para el día tercero la tierra seca.\\
\\
Una vez más, se narra en este día que Dios cumple su obra solamente con Su palabra. En Génesis 1:8 Dios le llama ``cielo'' a la expansión de arriba y sabemos que se refiere al cielo que se encuentra justamente por enciam de nosotros. En este día segundo es cuando Dios decreta la creación de la atmófera terrestre.\\
\\
En Génesis 1:7 se reitera este acontecimiento con el propósito de enfatizar que sí ocurrió en realidad. Toda posible explicación científica va a quedar incompleta debido a que el poder creativo de Dios va más allá de nuestra comprensión limitada.\\
Algo que ha sido difícil de interpretar ha sido la expresiónque se refier acerca de las aguas sobre la expansión. Regualrmente se piensa que ésto en se refería a una especia de dosel protector de vapor de agua que estuvo sobre la tierra hasta el diluvio, sin embargo, no hay suficientes elemntos en el texto para afirmarlo y existen algunas otras teorías acerca de ello.\\
Dios pronuncia, una vez más, que su creación era buena en Génesis 1:10 después de la separación de las aguas y de la tierra seca en el día tercero. Antes de ello, Génesis 1:8 marca el final del día segundo.\\
\\
Para el amanecer del tercer día la Tierra seguía sin vida y cubierta de agua. La manera espontánea en la que sruge le vegetación sobre la tierra seca para el final del día es otro argumento que se utiliza para explicar que la duración de los días en realidad era de un tiempo mucho mayor. Este argumento sería convincente si el relato fuera acerca de los procesos de la naturaleza perola creación por medio del poder de Dios es definitivamente una obra sobrenatural que excede nuestras ideas de ``sentido común'' y también cabe resaltar que, una vez más, dicha separación la hizo solamente con Su palabra.\\
La tierra seca que surgió al instante en realidad sí era tierra pues ya estaba perparada para sustentar vida vegetal.\\
\\
Dios ahora pronuncia la existencia de la vida vegetal, junto con su semilla y capacidad de dar fruto, mostrando una vez más, una creación madura. En particular, la frase ``\ldots según su género'' refuta claramente la ideología evolucionista pues Dios creó a las distintas especies de plantas y no una sola planta que evolucionaría a las demás.\\
\\
Éste proceso de creación es tan fascinante que resulta irracional el no reconocer la majestad de la inteligencia que lo creó todo.
\newpage
Para Génesis 1:12 Dios concluye otro día diciendo que lo había hehco era bueno. Cabe recalcar que los términos hebreos para mañana y tarde que se mencionan más de cien veces en el Antiguo Testamento siempre se interpretan de manera literal.\\
De todos los días pasados que se habían narrado, éste ofrece un cambio más espectacular pues narra cómo fue que por primera vez la vida surgió en la Tierra después de ser un lugar sumergido en agua.
\end{section}
\newpage
\begin{section}{Lumbreras en los cielos}
	La vastedad, complejidad y belleza de los astros en el cielo solamente revelan la gran gloria de Dios como creador.\\
	Hasta ahora, cada uno de los aspectos de la creación ha sido apuntando hacia la creación del hombre que siempre fue el propósito de Dios en su obra, tanto así que de todo lo creado somo los únicos que somos llamados ``a su imagen''.\\
	La descripción de la creación de las lumbreras ocurre en Génesis 1:14 hacieno un énfasis importante en el hecho de que todas las lumbreras fueron creadas en un instante.\\
	\\
	Es importante recalcar la magnitud que tiene esta creación pues hasta hoy en día nuestro planeta así como el resto de los cuerpos celestes, siguen una trayectoria que les fue impuesta por Dios cuando Él creo a las lumbreras.\\
	\\
	En este día cuarto  Dios creó el sol, la luna y las lumbreras para nosotros que incluso hasta hoy en día siguen siendo de sustento para nosotros, además de ello, esta creación tuvo el propósito de ser un marco de referencia permanente de separación entre el día y la noche, sustituyendo así a la luz que se había creado previamente.\\
	A pesar de ser seres inanimados, se menciona que el sol señorea el día y la luna señorea la noche. Esto quiere dar a entender que la luz que relejan dicta el ritmo de la vida en el planeta y donde cada uno determina la transición entre el día y a noche.\\
	\\
	La creación de dichas lumbreras la hizo Dios con sumo cuidado y detalle para que cumplieran su propósito, un claro ejemplo de eso es el brillo del sol pues a pesar de sus variaciones en caso de erupciones solares, conserva un balance constante de luz y energía que resulta ser perfecto para nuestras condiciones de vida aquí en la Tierra. Si esto no fuera así, es decir, si la cantidad de luz y calor que nos provee el sol fuera más caliente o más fría la vida en la Tierra llegaría a su fin. Por otro lado, la luna también desempeña un papel importante en nuestro planeta al contribuir en la vida con las olas oceánicas  que son causadas por su atracción gravitacional.\\
	Incluso dentro de la comunidad científica ya está descartado que la luna alguna vez haya sido parte de la Tierra pues se han encontrado al menos 3 elementos que no se encentran en la Tierra. Hasta la fecha no existe un consenso científico sobre la formación de la luna.\\
	\\
	Otros de los propósitos de las lumbreras es precisamente que marquen el paso de las estaciones y lo años. Sin embargo, además de eso la luna y el sol son quienes determinan los ciclos climáticos en el planeta, el ángulo en el que orbita la Tierra y el sol alumbra la Tierra es como obtenemos nuestras distintas estaciones en el año que afectan a toda la vida en la Tierra.\\
	\\
	La tercera razón por la que Dios creó las lumbreras fue, en efecto, para alumbrar sobre la Tierra. Esta parte del relato atenta en contra del argumento creacionista progresivo pues los cuerpos celestes no fueron producto de un proceso evolutivo de alguna evolución de la materia a la que llegara después de su vagar por el universo, Dios simplemente ordenó que existieran con madurez y en todo su esplendor.\\
	\\
	Génesis 1:18 concluye el relato de este día con el veredicto usual, Dios declarando que lo que había hehco era bueno. Todas las cosas habían funcionado tal y como Él lo había planeado en su infinita sabiduría, sin defecto ni deficiencia.\\
	\\
	La conclusión de este día conlleva una particularidad que lo destaca de los días pasados y es que por primera vez en la creación, el día había acabado por la luz proveniente de la luna y posteriormente el otro día empieza marcado por la luz del sol.\\
	Con el final de este día la semana de la creación había llegado a su punto medio y la creación empezaba a cumplir su propósito: hacer un despliegue elocuente y admirable de la gloria de Dios.\\
	\\
	C.S. Lewis propone un argumento sumamente interesante en contra de la evolución preguntando por qué sería racional confiar en los pensamientos de la mente humana si éstos surgieron de manera aleatoria y por consecuencia, se generan al azar.\\
	La única explicación razonable para el inicio del universo así como la creación de las estrellas y de nuestro sistema solar es la misma Escritura tal y como lo menciona Pablo en Romanos 1:19-20.
\end{section}
\newpage
\begin{section}{Abundancia de criaturas vivientes}
Cuando comienza el quinto día de la creación, el ambiente terrenal y celestial ya había quedado establecido por lo que los demás días de la creación consistieron en llenar el universo de seres vivientes.\\
La Biblia muestra una clara diferencia entre la vida vegetal y la vida animal pues los árboles y las plantas no se mencionan como seres vivientes. término que es exclusivo para los animales y el ser humano, con ello, se hace una clara diferencia entre la vida consciente e inconsciente.\\
Es por ello que hasta este día es cuando la Biblia se refiere a las creación de ``seres vivientes'' sobre la faz de la Tierra.\\
Este día es especialmente dedicado por Dios para poblar los cielos y los mares de seres vivientes, guardando una paralela con el día segundo cuando se separaron los mares y los cielos.\\
Si la teoría evolutiva tuviera validez, el texto mencionaría que la creación de estos seres vivientes hubiera sido a través de una forma de vida preexistenete, en cambio se tiene que Dios pronuncia su creación de la nada.\\
\\
Para Génesis 1:21 en la expresión en la que habla de ``todo ser viviente'' utiiza un témrino hebreo que significa ``aquello que respira'' para hacer un énfasis de que está hablando de la vida que depende del oxígeno para subsistir.\\
\\
De manera explícita Génesis 1:21 dice ``\ldots y creó Dios''los seres vivientes, éste termino descarta que se refiera a seres que evolucionarios de alguna forma de vida previa en unproceso demorado y lento.\\
Además, el pasaje menciona con énfasis la creación de ``grandes monstruos marinos`` al mismo tiempo que ''todo ser viviente que se mueve``Cabe resaltar del mismo versículo 21 el hecho de que las aves fueron ya creadas con la capacidad de volar, el versículo vuelve a reafirmar el hecho de que fue una creación madura.\\
\\
Ciertamente se tiene que la gran variedad de seres marinos que xisten como el caballito de mar, las grande ballenas, los inteligentes deflines y aún más seres en la profundidad son una gran muestra del poder creativo de Dios el cuál los creó a todos ellos y a aún más que no conocemos en tan sólo un instante.\\
\\
Resulta interesante el notar que Génesis 1:21 hace un énfasis especial en los ''grandes monstruos marinos``, esto puede ser quizá debido a que en los antiguos pueblos, los grandes seres estaban llenos de mitos e incluso les rendían culto como deidades marinas. Con el relato específico acerca de la creación de estos seres es como se demuestra que Dios está por encima de estas ''deidades`` y que Dios era el único Dios capaz de crear semejante universo.\\
\\
Así como hubo una gran vairedad de animales marinos, el versículo 21 menciona la gran variedad de aves que creó Dios, la cual es también sumamente asombrosa y magnífica, tomando en cuenta la variedad de estructura anatómica y colores que tienen.
\newpage
Es interesante incluso observar dentro de la gran variedad de vida en en los distinto tipos de aves que guardan una diferencia genética muy importante. En las aves, hablando también de polillas y de mariposas, las hembras portan el cromosoma XY mientras que los machos están configurados como XX mientras que en el resto de los animales sucede el caso totalmente contrario en su mayoría.\\
\\
La migración de las aves es otra gran muestra de la sabiduría de Dios pues recorren distancias enormes cada año cun una gran precisión sin necesidad de un mapa o una brújula. Hay avez que incluso recorren desde el polo norte hasta el polo sur, además de ese instinto de la migración, tienen un gran institnto para construir nidos que incluso lo hacen con distintas técnicas.\\
\\
Él hizo todas estas criaturas maravillosas por su propia complacencia y de la misma manera Él sigue supervisando de manera continua cada detalle de Su creación de acuerdo a su carácter soberano y amoroso.\\
\\
Cada uno de los seres vivos tiene la característica de que es capaz de subsistir por sí mismo. Hacer cosas como obtener alimento y defenderse de los depredadores y además se reproducen por ellos mismos, cada una de ellas capacidades que Dios le dio.\\
Esta gran capacidad de autosubsistencia es una gran señal sobre el poder que tiene Dios para crear seres, aún con la tecnoloǵia hoy en día, no se es posible el hacer una máquina que sea capaz de de satisfacer sus propias necesidades.\\
\\
En Génesis 1:22 semenciona claramente la procreación derrocando, una vez más, cualquier noción que se tuviera sobre un carácter  evolutivo. La Biblia enseña de manera explícita que Dios completó su creación de las criaturas marinas y de las aves antes de darles la orden de reproducirse.\\
\\
Las expresiones que se muestran en Génesis 1:11, 12, 21, 24, 25 tales como ''según su género``, '' según su especie`` y ''según su naturaleza`` es en realidad una sola expresión intercambiable que refleja que al reproducirse las criaturas vivientes sólo pueden producir criaturas similares a ellas mismas.\\
\\
La genética que ha sido una rama de la ciencia muy reciente ha sido de gran antagonista para la teoría evolucionista pues ayudó a demostrar lo mal que estaba Darwin en su hipótesis sobre las herencias genéticas que cada ser viviente le dejaba a sus descendientes.\\
\\
Aún las mutaciones genéticas no le ayudan a la teoría evolutiva pues las mutaciones solamente pueden remover o modificar aspectos genético mas no añadirlos.\\
\\
El tipo de mutaciones que puede generar las mutaciones genéticas pueden ser las diferentes razas de perros y a dichos cambios se les llama ''microevolución``.\\
\\
Resulta fascinante el entender un poco más de anatomía al entender la enorme complejidad de la vida en la Tierra.\\
\\
Continua el relato de la creación en Génesis 1:23 relatando la tarde y la mañana del día quinto. faltaba aún undía más de la creación antes del día de reposo en el cual se tendría la creación del hombre.

\end{section}
\newpage
\begin{section}{Bestias y animales que se arrastran}
	En el quinto día Dios ya había dado los toques finales llenando el mar y el cielo de vida para hacer una obra similar al sexto día.\\
	Una vez más se narra que la creación se hace con la Palabra de Dios de manera instantánea. Éste día sexto corresponde de manera paralela al día tercero llenando la tierra de criaturas vivientes.\\
	\\
	En este día vemos la creación de toda clase de animales terrestres, desde insectos y gusanos hasta elefantes y jirafas.\\
	La expresión ``animales de la tierra'' es una expresión genérica que se refiere a todo tipo de animales terrestres.\\
	\\
	La expresión ``produzca la tierra\ldots'' resulta muy interesante ya que es un recordatorio de que las criaturas hechas por Dios están compuestas de la materia que ya había creado en la tierra.\\
	A pesar de las ideas o tendencias que podamos tener acerca de ciertos animales, cada uno de ellos fue creado con buenos propósitos y revelan la diversidadcreativa, la sabiduría y la gloria de Dios con la misma claridad con que vemos su majestad despelgada en las estrellas y en el resto de la creación.\\
	\\
	Es probable que dentro de este relato también se esté tomando en cuenta a los dinosaurios quiene muy probablemente sufrieron de la extinción durante el diluvio o después debido al drástico cambio climático. La narración de su aparición junto con los seres humanos la podemos encontrar en Job 40:15 siendo Job el escrito más antiguo de la Biblica, este dinosaurio lleva por nombre behemot que quiere decir ``hierba come como buey`` que era un dinosaurio herbívoro.\\
Es increíble analizar el caso de cada uno de los animales y seres que creó Dios pues cada uno de ellos presenta inteligencia instintiva así como caracterísitcas especiales para su entorno.\\
\\
Génesis 1:25 repite la frase en la Dios dice que su creación era buena y esto nos da seguridad ya que descarta la posibilidad de deformaciones o mutaciones antes de la caída de Adán en pecado.\\
La Biblia enseña que la muerte no existía antes de la caída de Adán sino que la muerte es consecuencia del pecado que además terminó afectando a toda la creación según lo menciona Pablo en romanos 8:20-22.\\
\\
Ésto trae como consecuencia que inguna especia animal era carnívora, ningún animal cazaba ni mataba para alimentarse y esto se ve afirmado por la Escritura en Génesis 1:30. además de que la Biblia menciona que en el reinado milenario todo animal volverá a su condición herbívora como lo profetiza Isaías 11:6-8.\\
Así es como acaba el primer acto de la creación en el día sexto, completando el hábitat terrestre que Dios quiso preparar para Adán, un paraíso en donde todo era bueno y Dios se disponía a coronar su fructífera semana creativa con un ser creado a Su propia imagen.
\end{section}
\newpage
\begin{section}{El hombre a imagen de Dios}
	Todos los días pasados de la creación habían sido solamente un preludio para la creación del hombre que tendría lugar en el día sexto.\\
	la creación de la raza humana fue el objeto central del propósito creativo de Dios desde el principio. La Biblia nos dice que todo lo demás fue creado para la humanidad y que tuviera un ambiente perfecto Adán en el momento que fuera creado.\\
	\\
	Por lo que dice Apocalipsis 6:13-14 y $2^{a}$ Pedro 3:10 podemos analizar que toda la creación en algún momento será descreada con la excpeción de la humanidad pue sDios creó al ser humando para que le glorificara  y disfrutara para siempre. Depués de que todo ya se haya desvanecido, una multitud de seres humanos redimidos habitará con Él para siempre en Su presencia.\\
	El universo entero fue creado para que la gracia,la misericordia y la compasión de Dios fueran despositadas en el hombre.\\
	\\
	La narración bíblica dedica más espacio a la descripción de la creación de Adán que al de todos los demás aspectos de la creación.\\
	De hecho, todo Génesis 2 se dedica a la ampliación de la descripción susodicha. Génesis 2 no es una historia diferente ni un relato alternativo sino que es la descripción ampliada del mismo día sexto de la creacion.\\
	Es importante recalcar que, en efecto, la creación de Adán fue en el mismo día que Dios creó a las demás criaturas terrestres.\\
	\\
	Génesis 2:7 explícitamente relata qye Dios formó al hombre del polvo de la tierra. Además, Génesis 2 tambipen describe cómo es que Eva fue formada por Dios a partir del costado de su esposo, de forma que el hombre y la mujer fueron creados por separados y de forma individual a través de la obra directa de Dios.\\
	El versículo 26 habla acerca de la creación de Adán como una creación personal y única de forma que el hombre tendría una relación íntima con Dios como ningún otro ser en la creación.\\
	\\
	Dios se presenta por primera vez con pronombres personales y lo hace en plural, así dando una introducción a las múltiples relaciones que existen dentro de la deidad. Otra muestra de la Trinidad la vemos en el versículo 2 en donde habla acerca del Espíritu de Dios. La misma verdad también se muestra en los primeros versículos del evangelio de Juan. Y leyendo estos pasajes en armonía podemos ver a las tres personas de la trinidad actuando y participando en la creación.\\
	El hombre fue una creación por completo diferente a las demás ya que estaba dotado de consciencia.\\
	\\
	El hombre fue creado para portar la imagen y la semejanza de su creador.Tanto la imagen como la smejanza son términos paralelos donde uno reitera al otro con el propósito de subrayar la importancia del principio bíblico.\\
	Claramente la semejanza a la que se refiere el texto no es una semejanza física ya que tenemos semajanzas anatómicas con lo demás animales y más aún, Dios no tiene forma física.
	\newpage
	La imagen de Dios tiene que ver con el ser conscientes de nuestra propia existencia, de la moral y de los demás seres así como nuestra capacidad de  entender conceptos abstratcos como la moral, la justicia, la rectitud, santidad, verdad, bondad, entre otros.\\
	Los animales no pueden hacer estas cosas en el mismo sentido en que lo hacen las personas.Por ello es que cuando creó al hombre recalcó que no era buena que estuvier sólo. \\
	Dios nos hizo con el propósito de que tuvieramos realciones personales, en particular con Él.\\
	\\
	Dios diseñó y fromó el cuerpo del hombre para permitirle ejecutar funciones que Dios mismo ejecuta sin necesidad de un cuerpo.\\
	El texto es muy claro narrando que además de que creó a Adán s u imagen y semejanza, también lo hizo así con Eva.\\
	\\
	Dios también creó al ser humano con el propósito de que  de que tuvieran una relación monógama permanente y que se multiplicaran para llenar la tierra.\\
	Eva fue hecha como ayuda idónea para Adán como lo emnciona en los versículos 18,20. Ayuda para cumplir la tarea de procrear.\\
	\\
	También se ve dentro del texto un tercer propósito para la vida del ser humano, que pudiera recibir gozo y bendición de la mano de Dios. En el caso particular de Adán se tiene que todo el disfrute y la bendición en el mundo fueron suyos en un paraíso libre de mal y pecado.\\
	Adán y Eva recibieron total libertad para disfrutar de cualquier cosa que quisieran en el huerto con la excepción importante de que no comieran del árbol del conocimiento del bien y del mal pues ello les traería juicio de Dios.\\
	El último propósito que tiene Dios para la humanidad es que ejerzan dominio sobre el resto de la creación. El rpimer paso de este dominio se ve en Génesis 2:19 cuando Adán tiene la tarea de nombrar a cada uno de los animales de la creación. Así mismo, Adán tuvo la tarea de labrar la tierra del huerto pero con un ambiente perfecto antes de la caída.
\end{section}
\newpage
\begin{section}{El reposo de la creación}
	El capítulo 2 empieza con un recuento del día séptimo que marcó el finl de la semana de la creación. El día séptimo es un día único pues Dios lo bendijo y lo santificó. La estructura del pasaje es sumamente sencilla pero la importancia de su significado se hace evidenete en los distintos verbos utilizados.\\
	Ya todas las cosas habían quedado terminadas y completas. En su sabiduría infinita, Dios lo creó todo para que quedara completo y se mantuviera en funcionamiento.\\
	\\
	La importancia de este día séptimo es tal debido a que sirve para establecer los periodos de trabajo y de reposo que después Dios le daría a Su pueblo.\\
	Isaías 40:28 nos confirma que Dios no se cansa mas bien podemos saber que Dios descansó porque en realidad Él se abstuvo de realizar alguna obra creativa. Este día de reposos solamente es mencionado para Dios y podemos intuir que Adán no debía de reposar en ese día aunque si observamos las posteriores órdenes que tenía el pueblo hebreo acerca del día de reposos podemos observar que Adán cumplía con ada una por vivr en el huerto.\\
	\\
	Acerca del día de reposo para Dios, no se menciona de manera explícita cuando acaba pero debido a la duración semejante a los demás días podemos asumir que fue una simple omisión del texto a epsar de que, en efecto, Dios cesó de su bra creadora de manera permanente.\\
	No existe ninguna razón científica por la cual el calendario esté dividido en semanas de 7 días para juntar un año de 365 días mas que  el hecho de que Dios así lo estableció. El séptimo día nos recuerda y testifica en la actualidad que Dios acabó la obra de la creación.\\
	Es importante recordar que Dios siendo soberano  sobre todas las cosas no fue el autor del pecado. Él no concibió el pecado. Dios creó seres con la capacidad de tomar decisiones morales por las cuales cayeron en condenación.\\
\end{section}
\newpage
\begin{section}{El paraíso perdido}
	Según Romanos 5:12 y $1^{a}$ Corintios 15:22, desde el momento en que Adán pecó, trajo muerte y juicio sobre toda la humanidad ya que cada uno de nosotros hereda el pecado de Adán, el pecado original. romanos 8:22 enseñas que las consecuencias por el pecado de Adán además de sufrirlas el resto de la raza humana también las sufrió el resto de la creación, quedó estropeada y contaminada.\\
	Génesis 3 es el punto de quiebre la historia del declive moral de la humanidad.\\
	\\
	En el siguiente capítulo de Génesis, Génesis 4, se narra el primer homicidio y Génesis 4:19 menciona el pecado de la poligamia así como el versículo 23 habla de otro homicidio de allí en adelante se puede ver el declive de la raza humana hasta que Dios decide destruir a toda la raza humana con la excepeción de una familia.\\
	Jesús mismo habló de la caída del hombre como un hecho histórico en Juan 8:44 al referirse al diablo como el padre de la mentira.\\
	\\
	El hecho de que el texto describa a la serpiente como astuta nos da a entender que no era una serpiente cualquiera. Era un ser que conocía a Dios, con gran inteligencia y sagacidad, era engañoso y hostil.\\
	Satanás era quien había tomado el aspecto físico de la serpiente o de algún modo poseyó el cuerpo de una de ellas.\\
	\\
	Podemos ver que Satanás aparece en Génesis 3:1 por lo que podemos entender que la caída de Satanás debió ocurrir entre el punto final de la creación y los acontecimientos de Génesis 3. Por Ezequiel 28:11-19 es que aprendemos que Satanás era un ángel caído. El pecado por el que Stanás cayó fue precisamente el orgullo que surgió desde su corazón a pesar de que fue creado como un ser bueno que andaba en caminos perfectos.\\
	Además, por Apocalipsis 12:4 sabemos que Satanás cayó junto con una tercera parte de los ángeles pero aunque no sabemos por qué Dios no los condeó en el momento ciertamente el tiempo de su destrucción ya ha sido fijado según Mateo 8:29 y su condenación eterna es absoluta.\\
	Es necesario recordar que Dios permitió a Satanas confrontar a Eva, Él ya lo había planeado todo desde el principio.\\
	\\
	La estrategia que utiliza Satanás para tentar a Eva es la misma clase de tentación que siempre utiliza. Empieza hablándole con mentira mientras se disfraza como portador de la verdad. Lo que hace es plantearle duda a Eva lo cual es la estratagema de toda tentación, que dudemos de la Palabra de Dios. Satanás además torció la Palabra de Dios y le dio una representación falsa de su contenido y siempre enfatizó la atención de Eva sobre la prohibicón que les había dado Dios.\\
	Satanás quiso darle a entender que había una parte del carácter de Dios que era mala y mentirosa.\\
	\\
	La respuesta que le da Eva a Satanás solamente refleja la inocecncia que ella tenía además de que añadió algo a las palabras del mandato, afirmando que ni siquiera podían tocar el árbol cuando Dios no les había dicho eso.\\
	Satanás contestó diciendo que no moriría así además de haber calumniado la bondad de Dios también lo había hecho con su veracidad.\\
	\\
	Después de la respuesta de Satanás, Eva se ve en la disyuntiva de tener que tomar una decisión, si creerle a Dios o a Satanás. El diablo pretendió ofrecerles ser como Dios, algo que el mismo había intentado hacer con su orgullo.\\
	La voluntad de Dios es que seamos como Él en santidad, amor, misericordia,veracidad y sus demás atributos, en cambio lo que Satanás deseaba era Su poder.\\
	\\
	El pecado que  estaba en la mente de Eva ya estaba trabajando en sus emociones por lo que termina consumando el acto debido a que su felicidad personal se había convertido en su meta principal junto con sus motivaciones egoístas.\\
	En primer lugar, Eva fue seducidad por su apetito físico, un apetito ilícito. En segundo lugar, ella fue seducida por su apetito emocional y en tercer lugar fue seducida por su apetito intelectual por la promesa falsa de ser como Dios, siendo así seducida por los deseos de la carne, los deseos de los ojos y la vanagloria de la vida como lo dice $1^{a}$ Juan 2:16-17. El diablo tentó similarmente a Jesús en el desierto con la misma estrategia pero en ese relato vemos que Jesús surge victorioso de dicha tentación.\\
	\\
Después de que ella comió fue con Adán y también le dio del fruto. A pesar de ello, a Adán es a quien constantemente se le nombra como el culpable del pecado original a traves de la Escritura ya que Adán desobedeció e incurrió en transgresión de forma deliberada como lo dice $1^{a}$ Timoteo 2:14. Él tenía la responsabilidad por la caída por ser la cabeza y sus acciones fueron determinantes para el resto de su descendencia.\\
\\
después de haber comido sus mentes fueron abiertas y fueron susceptibles a toda clase de malos pensamientos y además fueron conscientes de su propia culpa, Ellos no recibieron los ''ojos abiertos`` que les había prometido Satanás mas bien recibieron una imitación barata y repulsiva de la iluminación mental.\\
Elpecado destruyó al instante su inocencia y ellos lo supieron, tuvieron vergüenza de su desnudez y ya se había perdido así la pureza sexual que habían tenido hasta entonces.\\
Dios les mostró que era necesario que los culpables de pecado se debían de cubrir y esto lo hizo al sacrificar animales y utilizar su piel para la pareja en el versículo 21. Ésta también fue una muestra de que Dios es el único que puede suministrar una solución apropiada para el pecado y que el derramamiento de sangre es necesario.\\
\\ Dios mismo maldijo la tierra y además se hizo necesario el cultivo arduo del suelo por consecuencia del pecado junto con demás dolencias y aflicciones que ahora plagaban toda la creación.
\end{section}
\newpage
Declaro delante del Señor haber leído en su totalidad el libro ''La batalla por el comienzo`` así como ''El manual bíblico`` en la parte correspondiente de Génesis a Ester.
%\end{document}
