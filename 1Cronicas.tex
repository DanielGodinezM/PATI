%        File: 1Cronicas.tex
%     Created: Mon Oct 07 09:00 PM 2019 C
% Last Change: Mon Oct 07 09:00 PM 2019 C
%
%\documentclass[12pt]{article}
%\usepackage[margin=1.0in]{geometry}
%\usepackage{enumerate}
%\usepackage[spanish]{babel}

%\begin{document}
\begin{section}{Crónicas}
	\begin{itemize}
		\item Título\\
			Quiere decir ``palabras'' o ``hechos de los días'' en hebreo. Es similar a decir ``la historia del pueblo'' pues cuando se tradujo al griego se le puso el nombre de `` Asuntos previamente omitidos''. Jerónimo le dio el término de Crónicas, tradujo el título del hebreo como ``Acontecimientos de los tiempos''. $2^{o}$ Crónicas 33:20 alude al establecimiento del reino de Persia que fue posterior al babilónico.
		\item Autor\\
			En el talmud judío se le atribuye el libro al ecriba Esdras quien llegó a ser un personaje muy importante en la historia de Israel. El libro está redactado desde la perspectiva de un sacerdote pues habla mucho acerca del templo. Como a Esdras se le reconoce como sacerdote es congruente pensar que él escribió los libros de Crónicas y los libros post-exílicos.
		\item Tema\\
			En el primer libro se habla de la vida de David y el tema del segundo libro son los distintos reyes de Judá.
		\item Propósito\\
			Recordar la fidelidad que tuvo Jehová con el pacto que hizo con David.
	\end{itemize}
	\begin{subsection}{Bosquejo del $1^{o}$ Crónicas}
		\begin{enumerate}
			\item Las genealogías (1-10)
			\item El reinado de David (11-29)
		\end{enumerate}
		\begin{subsection}{Bosquejo de $2^{o}$ Crónicas}
			\begin{enumerate}
				\item El reino de Salomón (1-9)
				\item Los reinos de judá (10-36)
			\end{enumerate}
		\end{subsection}
		No es una propia repetición de lo ya narrado, es un punto de vista distinto pues nos habla de la forma en la que espiritualmente el pueblo de Israel se fue alejando de Dios y recuperando después su comunión con él. Solamente trata de la historia de David y los reyes que descendieron de él, se nombran sólo los hechos positivos en la historia de Israel y se observa que tenían la esperanza de volver a la tierra que Dios les había dado.
	\end{subsection}
\end{section}
%\end{document}


