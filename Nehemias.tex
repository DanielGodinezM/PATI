%        File: Nehemias.tex
%     Created: Wed Oct 09 07:00 PM 2019 C
% Last Change: Wed Oct 09 07:00 PM 2019 C
%
\documentclass[12pt]{article}
\usepackage[margin=1.0in]{geometry}
\usepackage{enumerate}
\usepackage[spanish]{babel}
\begin{document}
\begin{section}{Nehemías}
	\begin{itemize}
		\item Título\\
			El protagonista era copero del rey, hizo esfuerzos por gobernar a Jerusalén que él sabía que era un pueblo rebelde. El templo ya estaba reconstruido pero la ciudad seguía desamparada pues la muralla seguía destruida. Seguramente conocían de la ley pues sus padres debieron de haérselas enseñado. Enl nombre dellibro es el del personajes princiapl. Al principio era un solo libro junto con Esdas pero en la traducción a la Septuaginta se les separaron.\\
			En el texto se ve que está narrado en primera persona, parece que escuchamos la narración de Nehemías por lo cua muchos le atribuyen el libro a Nehemías.
		\item Autor y fecha\\
			A pesar de que está escrito en primera persona, la tradición judía y crustiana consiedera que fue el mismo autor que el del libro de esdras pues utilizan vocabularios parecidos. Esdras fue el autro de Esdras. Ya había pasado 90 años de que habían regresado a Jerusalén por lo cual tuvieron tiempo de reconstruir el templo pero no las murallas. A pesar de estar en su ciudad, seguían sujetos al rey persa pues Jerusalén no tenía rey.\\
			Fue escrito alrededor del 424aC.
		\item Tema\\
			Nehemías tomó una actitud valiente ya nimó a su compatriotas a reconstrir el templo y no seguir pasando por un tiempo de vergüenza. Esdras propone un ayuno con el proóposito de perdi la protección de Dios.
	\end{itemize}
	\begin{subsection}{Bosquejo}
		\begin{subsubsection}{Primer regreso de Nehemías (1-7)}
			\begin{enumerate}
				\item Reedificación de los muros (1-3)\\
					Nehemías siendo el copero del rey Artajerjes era un sievo de mucha confianza junto con el cocinero pue seran los cargos de los cuales se podía aprovechar para envenenar al rey. Nehemías quizá era jefe de coperos pues él tenía el acceso diario para ver al rey, cosa que era rara. El rey envía a Nehemías con cartas para los gobernantes que le prestaran ayuda para reconstruir la ciudad. Nehemías nombra al pueblo como Hijos de Israel, la comunidad de Judá era la continuación legítima de las 12 tribus de Israel a pesar de que no habías diso obedientes a Jehová, Dios los seguía llamando sus siervos.\\
					Aproximadamente a mitad del libro es que Nehemías empieza a hablar en primera érsona y es por ello que se haya cnfusión acerca de la autoría del libro, sin embargo, se considera que seguramente Esdras lo transcribió de una pergamino en donde Nehemías escribió sus memorias. Si exisitió ese documento seguramente él las consiguió y las puso en el libro. \\
					 En el cap 2, Nehemías de spués de que consigue el permiso del rey para regresar a Jersualén alienta al pueblo a reconstruir las puertas de la ciudad. No tenían el número de personas necesario para construir una muralla tan grandde, en el vs 12 se dice que lo varnes acompañaron a Nehemías y probalbmente lo guiaron para que vierna las murallas de Jersualén ya destruidas. Nehemías iba en un animal mientras que los demás iban caminando y esto se debe a que Nehemías tenía un buen rango por ser coper del rey.\\
					 Dios había inclinado el corazón de Artajerjes para que se le adorara. La recosntrucción de la muralla participó toda la gente ya que cada familia tenía un pedazo de muralla por construir.
	\item Oposición a la reedificación (4-7)//
Sabían que ellos se podían gobernar a sí mismo y podían ser un  epligropara los demás. En el cap 4 se narra que convocan los ejércitos de Samaria. Tobías era otro hombre poderoso que se oponía a la reconstrucción de la murallla de Jerusalén. Conocían que eran respaldados por el rey por lo que procuraban solamente espantar a Israel sin que pudieran utilizar la fuerza pues sabían que podía  povocar la ira del rey. Nehemías nos e acobardó sino que fue a la ofensiva, él hablaba de que él estaba amparado por el re de los ielos, su confianza no estab en su fortaleza sin o en la fortaleza que él tenía de estar haciendo la obra de Dos, en el cap 4:4-5 surge una oración imprecatoria anónima y el contenido implica que probalblemente proviene de Nehemías pidiendo castigo para Sambalad y Tobías. Utilizaron distintas esttrategias apra intimidarlos, primero con base en burlas, después con amenzasas y ésto provoc+ó que solamente la mitad e la gente trabajara. Sin embargo, dios había determinado que se reedificara el muro, meintras estaban amenazados se dice que Nehemías 4:17-18. De esa forma trabajaron par apoder terminar la obra de Dios.\\
 En el cap 7 se nombra el censoque hace Nehemías.
			\end{enumerate}
		\end{subsubsection}
		\begin{subsubsection}{Avivamiento de Esdras (8-22)}
			\begin{enumerate}
				\item Lectura de la ley (8)\\
					EN el cap 8 se ve que el mismo pueblo es quien le ide a Esdras que les lean la ley. Para ese momento parece que a través del avivamiento que había rpovocado Esdras al rpincipio, el pueblo tenía verdadera habmbre de conocer la let. Solamente Esdras era el que tenía acceso a la ley y dice que la leyó desde que amanecía y la leyó aprox por 6 horas mientras el pueblo estaba atento. Por lo que dice el cap 8 parece ser que los levitas eran quienes también escuchaban la ley y la explicaban al pueblo pues el pueblo era muy numeroso.\\
					El pueblo quería conocer la voluntad de Dios para con ellos.
				\item Arrepentimiento del pueblo (9-10)\\
					Como todo el avivamiento eficas viene el arrepentimiento del pueblo confesando sus pecados, vemos una poderosa oración que hace Esdras. En esa oración empieza a recapitular desde que Dios liberó a su peublo de Egipto y cómo el pueblo responde con idolatría y desobediencia . Narra Esdras cómo Dios disciplinó y volvió a mostrar misericordia permitiéndoles regresar a su tierra. el pueblo recapacita que ésto era por su culpa y prometen servirle a Dios, su corazón estaba tocado por la Pablra de Dios. Redactaron un documento con su arrepentimiento y todo ellos lo firmaron.\\
					Lo que firmaron incluía maldiciones para quiene sno las cumpliera.
				\item Habitantes de Jerusalén (11-12)\\
					Pensaban quizá que fuera del templo, nada tenía que ver Dios pero Jerusalén había sido nombarada ciudad santa y elpueblo, pueblo santa. La lista de los cap 10-12 se compormetireron a serle fiel a Dios y esto destaca un verdadero avivamiento y deseo de obedecer a Dos. se narra la redistribución de los hbaitantes sobre el resto de país y se les dió prioridad a los levitas ara que vivieran en Jerusalén. Se dan los nombre sde los sacerdotes y de s us cargos, se restablecen os servicios en el templo. esdras sabía que Jerusalén debía de ser habitada por gente de linaje judío puro, esta pureza en la sangre era muy difícil de tener \\
					Actualmente los judío aceptan comunidades que se dicen judía pero no acpetan la pureza de esas comunidaeds y que sí pueden ser judío porque conocen las tradiciones pero no saben qué tan puros son. \\
					Nehemías solamente tenía el permiso del rey por un tiempo,ya había permanecido en Jerusalén por 12 años, regresó a Susa y consiguió otro permiso para regresar a Jerusalén (tal vez 10 años depués). Cuando Nehemías regresó vio tristemente que el pueblo ya estaba violando los prncipios que habían firmado, el priero de los conflictos que resolvió fue el de volver a purificar el templo pues el mismo sacerdote se había emparentado con Tobías e incluso lo había metido para que viviera en el templo de Jehová. Nehemías va a al templo y ehca a Tobías a la calles.\\
					 Sabían que los levitas debían de ser sustentaos por el pueblos pero eso ya no lo estaban haciendo por lo ue los levitas tenían que trabajar para sostenerse y descuidadban su labor, además de que se estaban relacionando con mueres gentiles. El vs 24 dice que ni siquiera sus hijos hablaban hebreo pues estaban aprendiendo la lengua extranjera de sus madres. Esdras vio con dolor cómo el pueblo regresa al epcado y se arracna los cabellos  y la barba. Nehemías dice en el vs 25 que les arrancó sus camnos a ellos y los obligó a que regresaran al buen camino. \\
					 Ésto pone en evidencia la naturaleza humana que hasta la fecha nos impide ser totalmente obedientes.
		\end{subsubsection}
	\end{subsection}
\end{section}
\end{document}


