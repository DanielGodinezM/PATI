%        File: Exodo.tex
%     Created: Wed Sep 04 06:00 PM 2019 C
% Last Change: Wed Sep 04 06:00 PM 2019 C
%
%\documentclass[a4paper]{article}
%\usepackage{enumerate}
%\usepackage[spanish]{babel}
%\begin{document}
\begin{section}{Éxodo}
	\begin{itemize}
		\item Título\\
Tomado de la versión de la Septuaginta. El título en la Biblia hebrea es: ``Y estos son los nombres\ldots''. En el idioma latín, para la versión Vulgata, tiene el nombre de Exodus que literalmente significa ``salida''
		\item Autor y fecha\\
El autor claramente es un testigo ocular dada la narración de los eventos. El tipo de vocabulario que se ocupa dentro del libro nos  indica que fue una persona letrada. Hoy en día hay reservas de que Moisés sea verdaderamente el autor del Pentateuco pero por la fuerte relación que exite entre los libros del Pentateuco es muy propio asumir que comparten el mismo autor y por lo que se dice en Éxodo 24:4.\\
En Éxodo y en Números se asegura que fue Moisés quien estaba registrando los sucesos del pueblo hebreo e incluso en Deuteronomio 31 hay un canto cuyo autor se garantiza que es Moisés.\\
En  Marcos 12:26 se dice que Moisés fue quien escribió en particular este libro. La fecha de la salida 1445 a.C. Fue escrito en el 1425 a.C. aproximadamente.
		\item Tema\\
		Redención del pueblo de Israel de la esclavitud que vivió en Egipto. Continuación de Génesis, en el hebreo dice ``y estos son los nombres\ldots'' así confirmando la continuación del texto anterior (Génesis). Habla acerca de cómo una vez que permiten que salga el pueblo de Egipto lo libra de todos los enemigos que tiene que enfrentar hasta que llegan a la tierra prometida.
\item Propósito del libro\\
Narrar la liberación del pueblo de Dios por medio de su misericordio y el posterior establecimiento de Su ley.
	\end{itemize}
	\begin{subsection}{Bosquejo}
		\begin{enumerate}
			\item Israel en Egipto
				\begin{enumerate}
					\item La explosión demográfica (Éxodo 1:1-7)\\
					El relato ya empieza 300 años después de la muerte de Josué. De 70 personas que habían llegado, ya eran 2 millones de personas. El crecimiento del puebo judío fue muy grande y vemos cumplida la promesa que Dios le había hecho a Abraham.\newpage
					\item La esclavitud (Éxodo 1:8-22)\\
					El crecimiento del pueblo de Israel provocó que los egipcios los esclavizaran. Al esclavizar al pueblo que vivía con ellos trataron de disminuir ese crecimiento y creyeron que podían aprovechar esos números para sus propias construcciones. El pueblo de Israel siguió creciendo y por ello el faraón dio la orden de dar muerte a los niños hebreos. A pesar de este mandato de faraón, Dios dispuso que sería del mismo palacio del faraón que vendría la persona que habría de liberar a Su pueblo. 
					\item Moisés el libertador (Éxodo 2:1-11:10)\\
						Moisés en hebreo significa ``hijo'' (Moshéh). Sus padres eran dos miembros de la tribú de Leví que tuvieron dos hijos más, María la mayor y Aarón. Moisés llegó al mundo cuando el pueblo hebreo era víctima del odio y del maltrato de los egipcios. Cuando él nació, su madre lo logró esconder del edicto del faraón. Cuando ya no lo pudieron ocultar, su madre lo colocó en una canasta y lo dejó en las aguas del río Nilo con el propósito de que la princesa egipcia lo encontrara y lo rescatara lo cual sucedió. La hija del faraón lo adopta y autoriza a la misma madre de Moisés para criarlo. A través de este ``incidente'', Dios le dio toda la preparación que Moisés necesitaba para guiar al pueblo de Dios a la tierra prometida. Él fue enseñado en todas las ciencias en las que eran enseñados las personas en la casa del faraón, finalmente es que fue un principé. Josefo narra que Moisés en su juventud inclusive comandó los ejércitos del faraón pero él siempre estuvo consciente de que era hebreo, su madre seguramente fue quien la instruyó en esos ámbitos. \\~\\
						40 años pasó Moiśes en su etapa de preparación en la corte de faraón hasta que por defender a un compatriota suyo, mató a un egipcio y tuvo que huir ya que mostró su verdadera lealtad hacia el pueblo hebreo. Moisés tuvo que huir pues el faraón se enteró del asunto y procuró matarle. El faraón conocía que el acto de Moisés podía conducir al pueblo hebreo hacia una revuelta y por ello Moisés huye a Madián.\\
						Empezaba otra etapa de 40 años en la vida de Moisés donde ya no estaría tan cómodo como en el palacio egipcio. Estando aislado de su pueblo, tomó como esposa a Séfora que le dio a Moisés 2 hijos, Gersón y Eliezer. El llamado de Dios para Moisés fue mientras estaba en el desierto cuando vió una zarza que ardía pero no se consumía, se suele simbolizar como la relación entre Dios y el pueblo. La zarza común y corriente y débil representado a Israel y el fuego representando el poder de Dios que no era para destruir pero para purificar y es por ello que no se consumía.\newpage
						Otro comentario lo hace Charles Windoll en el que dice que Moisés en sus primeros 40 años Moisés se creía una zarza muy valiosa pero Dios le mostró que Él no necesitba eso sino a una zarza común y corriente, fue necesario que Dios le secara de sus pretenciones y abatirlo hasta hacerlo una zarza seca de todo orgullo y debil para que por el poder de Dios se levantara y llevara a cabo Su obra. Moisés incialmente presentó objeciones a la voluntad de Dios pero era Dios mismo quién lo iba a llevar para que realizara dicha obra.  Él sabía que como representante de Dios debía de persuadir a su pueblo. \\~\\
						Los dioses egipcios tenían nombres y el pueblo de israel quería conocer el nombre de su Dios y es por ello que en Éxodo 3:14 Dios le dice ``YO SOY EL QUE SOY'', el tiempo verbal de esta oración es indefinido de forma que puede representar pasado, presente y futuro. Dios le pidió a Moisés que le dijera a faraón que dejara ir al pueblo de Dios, un insignificante pastor se lo tenía que decir al entonces hombre más poderoso del mundo.\\
						Dios ya no iba a obrar en secreto ahora iba a mostrar su poder de manera visible con las 10 plagas que enviaría a farón y a sus hombres\ldots
					\begin{itemize}
						\item Agua se convierte en sangre\\
							Éxodo 7:17
						\item Ranas invaden la tierra\\
							Éxodo 8:2-4
						\item Piojos\\
							Éxodo 8:17
						\item Moscas\\
							Éxodo 8:24
						\item Sobre el ganado\\
							Éxodo 9:3
						\item Úlceras a todo el pueblo\\
							Éxodo 9:9
						\item Lluvia con enormes granizos\\
							Éxodo 9:18
						\item Langostas sobre todos los campos de Egipto\\
							Éxodo 10:4-5
						\item Tinieblas sobre la tierra\\
							Éxodo 10:22
					\end{itemize}
					Para la $10^a$ plaga se tiene la pascua
				\item La Pascua (Éxodo 12:1-13:16)\\
					A partir de ese momento se convirtiría en una fiesta que habrían de recordar. Se salvarían las vidas de los primogénitos de los fieles mientras que los primogénitos de los incrédulos perecerían.\newpage
					Dios había dado instrucciones específicas para que su pueblo saliera de la tierra de Egipto, debían de sacrificar un animal y tomar su sangre para pintar los dos postes y el dintel de sus casas. La palabra Pascua significa ``mostrar misericordia`` en el hebreo original y viene de la misericordia que mostró Dios para con los primogénitos de Israel.
				\end{enumerate}
		\end{enumerate}
	\end{subsection}
	\begin{subsection}{Israel liberado}
		\begin{enumerate}
			\item Salida de Egipto (Éxodo 13:17-14:14)\\
				Moisés tomó la ruta larga porque esa fue la ruta que Dios había designado y fue el camino por el que Él los estaba guiando. No se sabe con exactitud la ubicación del monte Sinaí así como la ubicación de dónde fue que cruzaron el Mar Rojo.
			\item El cruce del Mar Rojo (Éxodo 14:15-15:21)\\
				Pronto faraón volvió a su necedad a perseguir al pueblo hebreo con todo su esplendor militar. Moisés le recuerda al pueblo acerca de la protección divina que tienen, lo único que ellos tenían que hacer era observar la obra de salvación porque Él iba a pelear por ellos. Dios vuelve a mostrar su poder al permitir que se dividiera el mar para que Su pueblo pudiera pasar sobre las aguas. Dios permitió que las aguas volvieran de forma que todo el ejército de Egipto con todo su armamaneto quedara sepultado bajo el mar, después de eso el pueblo de Israel reconoce y teme al Dios que tienen. En el capítulo 15 se muestra un cántico de victoria en forma de poesía hebrea entre Moisés y María su hermana.\\
				\underline{Dificultades de interpretación:}\\
				Mucho se ha comentado de que todo este acontecimiento del éxodo no es literal. Se han tratado de dar distintas posibles interpretaciones de caracter ``científico'' para determinar cómo es que surgieron dichas plagas y cómo es que pasó el cruce.\\
				Una de dichas interpretaciones es la de National Geographic, ellos afirman que todas las plagas surgieron de un fenómeno meteorológico y que todo empezó con la erupción de una volcan en Grecia. Esta erupción provoca  que se conataminara el agua del Nilo y por ello es que el agua se tornó de color rojo, por la falta de oxígeno en el agua los peces comenzaron a morir y por ello las ranas que vivían en el Nilo se salieron y empezaron a invadir Egipto. La carencia de agua limpia produjo la plaga de los piojos y la de las moscas, después plagas bacterianas en los animales y los hombres. Ésto le causó los sarpullidos a la gente. El granizo que calló se considera un granizo volcánico (piedra, no hielo) que procedió de la erupción del volcán Santorini. Cuando la erupción llegó a la estratosfera se mezcló con el aire y cayó como granizo. La nubre de cenizas llegó hasta Egipto una altura de 200m y provocó una oscuridad total en Egipto (9a plaga).\newpage
				En realidad no sabemos cómo fue que sucedió pero creemos que todo ocurrió tal y como lo narra. Muchos arqueólogos incluso sostienen que el cruce del mar rojo no ocurrió.\\

		\end{enumerate}


	\end{subsection}
	\begin{subsection}{Israel en Sinaí}
		\begin{enumerate}
			\item Viaje y Murmuración (Éxodo 15:22-18:27)\\
Se narra en Éxodo 17:6 cómo es que Dios les suministra agua en una peña en Horeb después de la murmuración del pueblo hacia Moisés por la falta de agua y de comida.\\
En Éxodo 17:8 se narra el primer conflicto que tiene Israel cuando sufren un ataque de Amalec. En el capítulo 18, vemos los consejos de Jetro, su suegro,  a Moisés.
\item Acampando en el Sinaí (Éxodo 19:1-19:35)\\
	El pueblo de Israel ya había sido testigo del gran poder de Dios pero fue necesario que llegaran al pie de esa montaña para que Dios revelara su poder en gran esplendor. Ahí contemplaron la grandeza de Dios.\\
	Dios les habló en ese lugar directamente y les llamó su especial tesoro, gente santa, el pueblo que Dios escogió para que de ellos viniera la salvación para toda la humanidad pero Dios les pedía santidad y obediencia. Israel estaba consciente de su debilidad y del pecado que vivía en ellos pero en Éxodo 19 mostraron su arrogancia al decir que ellos iban a hacer TODO lo que Dios les dijera. El pueblo por primera vez tuvo temor y miedo de Dios, por ello en Éxodo 20:19 le dijeron a Moisés que él hablara con Dios y luego les transmitiera el mensaje ya que consideraban el escuchar a Dios como una tarea muy temeraria.
	\item La ley (Éxodo 20:1-24:18)\\
	Dios da oralmente a Moisés el decálogo (esencia de la ley) y en Éxodo 24:3-8, Moisés se lo comunica al pueblo\\
	Después vuelve a llamar a Moisés para entregarle la ley escrita. Cuando Moiśes asciende al monte se menciona que una nube cubre el monte y al $7^o$ día le es llamado en la nube, Moisés permanece ahí 40 días y 40 noches.
\item La idolatría y el tabernáculo (Éxodo 25:1-40:38)\\
	Dentro de las instrucciones que dio Dios, estaba la orden de la construcción de un tabernáculo pero en el capítulo 32 el pueblo vuelve a ofender a Dios. Le dijeron a Aarón que les hiciera a otros dioses pues Moisés ya se había tardado en bajar del monte. Cuando Moisés llegó al campamento y vio el becerro de oro y las danzas arrojó las tablas de la ley y las quebró. Del capítulo 32-39 se narra la construcción del tabernáculo conforme Dios se los ordenó. Moisés hizo de acuerdo a todo lo que Dios le mandó, el tabernaculo entró en funcionamiento hasta el capítulo 40 que dice que una nube cubrió el tabernáculo de reunión y la gloria de Jehová lo llenó.
		\end{enumerate}
	\end{subsection}
\end{section}


%\end{document}


