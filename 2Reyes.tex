%        File: 2Reyes.tex
%     Created: Mon Oct 07 05:00 PM 2019 C
% Last Change: Mon Oct 07 05:00 PM 2019 C
%
%\documentclass[12pt]{article}
%\usepackage[margin=1.0in]{geometry}
%\usepackage{enumerate}
%\usepackage[spanish]{babel}
%\begin{document}
\begin{section}{Segundo libro de Reyes}
	\begin{itemize}
		\item Tanto el título como el autor y la fecha de escritura del libro son análogos al libro pasado ya que en realidad eran un mismo libro hasta que se dividió en la traducción al griego.\\
			No se conoce quién fue el autor de ambos libros pero se calcula que se escrbió durante el exilio del pueblo por el 561 a.C.
		\item Tema\\
			El exilio, la infidelidad del pueblo de Dios en los dos reinos se incrementó cuando se prostituyeron los reyes y Dios aplicó su justicia ya profetizada enviando a ambos reinos al exilio. Dios que en amor les había entregado la tierra prometida, los iba a expulsar por su desobediencia.
		\item Propósito\\
			Mostrar la disciplina de Dios cuando el hombre peca. El propósito de que Dios mandara al exilio a Su propio pueblo fue para disciplinarlos y purificarlos para llevarlos a una vida en comunión con Él y en santidad.
	\end{itemize}
	\begin{subsection}{Bosquejo}
		\begin{subsubsection}{El ministerio de Eliseo ($2^{o}$ Reyes 1-13)}
			\begin{enumerate}
				\item Sus milagros ($2^{o}$ Reyes 1-7)\\
					Desde $1^{o}$ Reyes 19:19 se ve el llamamiento de Eliseo pero vamos a ver el poder de Dios en Eliseo hasta este libro, llegó a tener tanta fama como Elías. En el capítulo 1 se narra la última parte del ministerio de Elías profetizando la muerte de Ocozías, hijo de Acab quien solamente reina por 2 años. Tras su muerte lo sustituye su hermano Joram.\\
					En el capítulo 2 Elías es llevado al cielo en medio de un torbellino y se menciona que Elías no murió físicamente al igual que Enoc que fue arrebatado. Este hecho fue visto por Eliseo y le pidió que le diera una capacidad más allá de lo normal para llevar a cabo la tarea de profeta de Dios. \\
					Eliseo hace un primer milagro abriendo las aguas del jordan en el capítulo 2, se postra la gente ante Eiseo y le reconocen como el sucesor de Elías, además limpia las aguas del manatial que estaban contaminadas.\\
					Reprueba una alianza entre Joram, el rey de Israel y Josafat el rey de Judá. Ambos consultan a Eliseo antes de luchar contra los moabitas y Eliseo conociendo la idolatría del norte desecha al reino del norte. Eliseo lo reprende diciéndole que fuera con los profetas de sus padres pero sólo por Josafat es que atiende a su llamado de profetizar. Eliseo les profetiza que iba a haber lluvia y que iban a tener una victoria militar.\\
					En el capítulo 4 se narra que hace provisión de aceite para una viuda para que la vendiera y que satisfaciera sus necesidades. Otro milagro en $2^{o}$ Reyes 4 es cuando resucita al hijo de la sunemita y después purifica la comida envenenada.\\
					En el capítulo 5 se menciona al comandante Naamán que tenía una enfermedad en la piel y Eliseo le dijo al rey de Israel que se lo mandara a Samaria, él lo podía sanar.
					\newpage
					El hombre esperaba que Eliseo le diera una honrada recepción, cuando llega, Eliseo le indica que se lavara en el río Jordán, narrado en $2^{o}$ Reyes 5:14. Ésto enfurece a Naamán pero al final sus oficiales lo hacen entrar en razón y obedece lo que le había dicho el profeta. Era un hombre poderoso pero noble pues siendo un hombre distinguido pudo haber hecho llamar a Eliseo a su casa, después de lavarse como se le había dicho, la piel del hombre es sanada. Naamán se da cuenta de que de verdad había sido el Dios de Israel quien lo había sanado y no fue una acción simple del profeta, por ello es que reconoce la unicidad de Dios. Además se comenta en este pasaje que Naamán pidió a Eliseo comprensión para cuando tuviera que acompañar a su rey, quien acostumbraba hacer adoración al dios Rimón, que Dios lo perdonara por inclinarse cuando el rey se apoyara sobre su brazo ya que su lealtad política tenía que ser para el rey. El rey le iba a pedir que lo acompañara a ritos profanos y Naamán sabía que ésto podía ofender al Dios de Israel por lo que Eliseo lo mandó en paz.\\
					Naamán vivía en medio de un pueblo pagano pero ahora Naamán conocía a Dios pues había visto Su obra en su propia persona. Lo que le tocaba a Eliseo era pronunciar con claridad el mensaje de salvación a un hombre idólatra, y Dios le daría el entendimiento a éste hombre ahora humilde para conseervar su fe pura viviendo en medio de un pueblo pagano.\\
					Nosotros en la actualidad vivimos en medio de un pueblo pagano pues vemos que la myoría adora imágenes y esculturas, así como le creen más a los hombres que a Dios. En muchas ocasiones ni siquiera sabemos cómo actuar ante estas situaciones. El conocer quiénes somos y a quién adoramos nos confirma el saber que lo que hacemos no tiene algo malo. Nosotros sabemos a quien adorar y lo que pase alrededor de nosotros no tiene valor.\\
					A partir de $2^{o}$ Reyes 5:20 se narra que un siervo de Eliseo, Giezi, va a cobrarle a Naamán con engaños y le saca dinero y vestidos para él. Giezi pudo haber llegado a ser quizá sustituto de Eliseo pero su codicia lo hizo fallar y por consecuencia de su pecado la lepra de Naamán se le pega a Giezi.\\
					Lo que debemos de buscar siempre es fortaleza en Dios para evadir estas tentaciones que siempre se presentan y así evitar que caigamos en ellas.\\
					En el capítulo 6 pasan dos milagros más pues Eliseo hace flotar la cabeza de un hacha sobre el agua del río Jordan y despúes cega al ejército arameo para evitar que lo atraparan.\\
					En el capítulo 7 Dios hace huir al ejército enemigo haciéndoles crees que numeroso ejército se acercaba a su campamento y por la prisa dejaron todas sus provisiones con las que el pueblo de Israel pudo salir a comer.
				\item Reyes de Israel y de Judá (2 Reyes 8-13)\\
En el capítulo 8 se narra que Eliseo llora y le predice a Hazael que él sería el instrumento por el cual juzgaría Dios a Israel de forma sanguinaria. Hazael se consideraba un don nadie por lo que no comprende cómo habría de ser posbile que él cometiera esas terribles cosas pero Eliseo le profetiza que él sería rey de Aram.\\
Los reyes de Israel siguieron andando en los pasos de Jeroboam, algunos hicieron algún reconocimiento al Dios de Israel pero todos siguieron adorando a otros dioses. 
\newpage
Ocozías, siguiendo los pasos de sus padres, fue un rey irrelevante, lo sustituyó Joram que hizo una alianza con el reino del sur, terminó asesinado por Jehú quien después se convirtió en el siguiente rey del reino del norte y después reinó su hijo Joacaz.\\
En el capítulo 9 se ve la aplicaión de la justicia de Dios sobre la casa de Acab pues Eliseo manda a un profeta para cumplir la instrucción que Dios le había dado a Elías pues Eliseo tiene miedo de cumplir esta instrucción. Jehú es ungido por un profeta que mandó Eliseo y es nombrado como rey de Israel. Jehú era comandanete del ejército del rey Joram por lo que esta unción puede ser considerada como traición y por ello es que Eliseo mandó al profeta a hacerlo, las instrucciones de Eliseo se encuentran en $2^{o}$ Reyes 9:2-3. Este rey tendría que hacer justicia por los profetas que habían sido asesinados por la casa de Acab, por ello, Jehú tenía la misión de exterminar a toda la descendencia de Acab y Jezabel.\\
Jehú inicia de inmediato las instrucciones pues rapidamente mata a Joram, después se va en contra del rey de Judá Ocozías que estaba de visita pues también era miembro de la familia de Acab. Una vez que Jehú mando a esos dos descendientes de Acab se va con toda su furia en contra de Jezabel y la mata de una forma sumamente violenta en $2^{o}$ Reyes 9:32-33 pues mandó que la arrojaran por la ventana y se narra que aún su sangre salpicó la pared cuando cayó. Posteriormente los caballos en los que venían Jehú y su ejército pasaron por encima de Jezabel. Se tenía la profecía de $1^{o}$ Reyes 21:23 y finalmente se cumple en manos de Jehú.\\ \\
Cuando fueron para sepultar a Jezabel no hayaron de ella más que el cráneo, los pies y las manos, narrado en $2^{o}$ reyes 9:35-36.\\
En el capítulo 10 se narra la masacre que hizo Jehú con todos los descendiente de esta pareja, manda a degollar los 70 hijos de Acab y manda a matar a los 42 parientes de Ocozías. En $2^{o}$ Reyes 10:8 se narra que Jehú manda apilar las cabezas de los hijos del rey pues quería mostrar la justicia de Dios.\\
En el capítulo 11 se narra que Atalía fue reina de Judá, esposa de Joram, hija de Acab. Su matrimonio selló una alianza entre los dos reinos. Ella se distinguió por el fomento a la idolatría, en especial, el culto a Baal. Atalía al ver que su hiijo había sido asesinado, manda a matar a toda la familia real. Atalía es nombrada reina de Judá y así es como una descendiente de Acab termina como reina de Judá, un trono que se había dicho que iba a ser exclusivamente de la familia de David. Atalía gobernó por 6 años y es destronada por Joiada quien preservó la vida de Joás para que siguiera reinando la casa de David.\\
En el capítulo 13 se narra la muerte de Eliseo, en específico en $2^{o}$ Reyes 13:20. Eliseo no solamente tuvo poderes para restaruar vidas mientras vivía pues aún muerto sucedió otro milagro en el lugar de su sepultura. Un año después de su muerte se narra que unos israelitas estaban eterrando un cuerpo pero que al ser sorprendidos por saqueadores moabitas quitan la piedra de la tumba de Eliseo y arrojan el cadaver del soldado muerto pero cuando el cadaver toca los restos de Eliseo revive y es levantado.
\newpage
			\end{enumerate}
		\end{subsubsection}
		\begin{subsubsection}{Antecedentes y cautiverio de israel (2 reyes 14-17)}
			\begin{enumerate}
				\item Últimos reyes desobedientes (14)\\
				Dos reyes pasan sobre el trono de manera muy breve pues fueron asesinados de manera muy violenta y su asesino Manahem es a quien le toca gobernar mientras los conquista Asiria.
			\item Consquitados por Asiria (15-17)\\
				Su rey llamado Pul había convertido a Manahem como un tributario pues le había dado impuestos. Muere Manahem y Peka mata al hijo de Manahem, Pekaía. Oseas mata a Peka quien llega a ser el último rey del reino del norte. Viendo que el rey de Asirir lo obliga a pagar tributo, se niega y forma alianza con Egipto.\\
				Salmanasar, rey de los asirios, sitia el reino del norte y toma como prisionero a Oseas en $2^{o}$ Reyes 17:6. En el capítulo 17 se menciona el por qué del juicio de Dios para Israel y enumera sus pecados contra Dios desde que los sacó de Egipto. Edificaron lugares altos en todas sus ciudades, levantaron estatuas e imagenes de Asera, quemaron incienso en todos los lugares altos tal y como lo hacían los pueblos paganos, todas estas razones se mencionan en $2^{o}$ Reyes 17:7-11. Con esto desaparecieron prácticamente 10 de las 11 tribus de Israel. Se pobló Israel en $2^{o}$ Reyes 17:34 de gente que era extranjera.
			\end{enumerate}
		\end{subsubsection}
		\begin{subsubsection}{Antecedentes y cautiverio de Judá (18-25)}
			\begin{enumerate}
				\item Reyes obedientes y desobedientes (18-23)\\
					A este periodo de la historia se le conoce como el reino solitario pues Judá luchó por sobrevivir sólo ya que Israel, su único posible aliado, ya estaba en cautividad. En $2^{o}$ Reyes 18:13 se narra que se tomaron las ciudades fortificadas de Judá después de que Ezequías, rey de Judá, se había rebelado y negado a pagarles tributo. Ezequías le habló a Senaquerib confesando que había cometido un error al no pagar los tributos y le pide que dejara a sus ciudades con la condición de que le pagaría cuanto tributo quisiera. A pesar de eso, Senaquerib manda a un ejército a sitiar Jersualén. Rabsaces (general de los ejército asirios) trató de intimidar a Jerusalén diciéndoles que venía de parte de Jehová en $2^{o}$ Reyes 18:30-31. De esta forma, blasfemando el nombre de Dios, se manifiesta la arrogancia de Senaquerib pues ae presenta como un rival de Jehová. A pesar de ello, Ezequías sabía que solamente las armas espirituales eran las que lo podían librar de sus enemeigos, el poder de Jehová, esperaba Ezequías, que atendiera a las oraciones para no ser invadidos.\\
					A Isaías le informan de las blasfemias y responde diciendo que Dios le había dicho que no debían de temerle, volvería a su tierra y ahí moriría. El rey Ezequías oró a Jehová reconociendo Su unicidad y el poder de Dios. Dios escuchó las oraciones del rey y del pueblo, contesta por medio de Isaías en $2^{o}$ Reyes 19:20 y en $2^{o}$ Reyes 19:32-33 le dice que el rey no entraría a la ciudad y que por el mismo camino que llegó regresaría. Dios actuaría conforme a sus actos portentosos del pasado, Jerusalén permanecería inviolable. Aquella misma noche se cumple la profecía según $2^{o}$ Reyes 19:35.\\
					Finalmente en el capítulo 20, se ve la misericordia que tuvo Dios con Ezequías que por su fe Dios le ayudó para que Jerusalén no fuera conquistada, más adelante, Dios sana a Ezequías de una enfermedad de muerte y le añade aún 15 años más de existencia.
					\newpage
					Se narra su muerte como uno de los mejores reyes del reino del sur que hizo lo recto ante los ojos de Jehová.\\
					De los capítulos 21-23 se dice que reinaron sobre el trono 8 descendientes de David, Joaquín y al final Sedequías, el último rey del reino del sur. De estos 8 reyes, 6 cayeron en los erroes de los reyes de Israel, solamente 2 le fueron fieles a Jehová. En el capítulo 22 se narra el reinado de Josías que es considerado el mejor rey de Judá en la época del reino dividido pues hizo lo recto ante los ojos de Dios. Dentro de varias reformas que hizo, decidió reparar el templo y se encontró un ejemplar del libro de la ley que se le había dado a Jehová, el libro le es llevado y rompe en llanto al conocer la desobediencia del pueblo en el libro, esto se narra en $2^{o}$ Reyes 22:8-9.
				\item Conquistado por Babilonia (24-25)\\
En el año 612 a.C. los caldeos derrotan al imperio de Asiria. El comandante de los caldeos marcha a conquistar Egipto y Nabucodonosor II se queda como comandante de los ejércitos caldeos. Nabucodonosr se enfrentó contra el ejército egipcio y después de la batalla toda la región del imperio asirio queda en mano de los caldeos. Nace el imperio babilónico. Siendo Joacim el rey de Judá llegó Nabucodonosor y los hizo su tributario. El hijo de Joacim, Joaquín  ya siendo rey se le ocurre rebelarse en contra de los babilonios, sitian la ciudad y se lo llevan cautivo. Hubieron tres deportaciones, en la primera se llevarona Joaquín y al profeta Daniel junto con otra parte del pueblo, gente importante y noble. Nabucodonosor pone a Sedequías que después al rebelarse provoca a ira a Nabucodonosor, regresa el ejército y llega a Jersualén para asediarla. \\ \\
A pesar del ejército, tomaron 2 años para conquistarla. Todo iba bien hasta que tuvieron el problema de que ya no había comida en la ciudad y el hambre empezó a apretar a la ciudad y en la moral de la ciudad. La conquista se hace cuando logra abrir una brecha el ejército caldeo, entran y destruyen la ciudad así como queman el templo de Jehová. Nabucodonosor estaba muy enojado por la rebeldía y por ello destruyen los muros, queman  el templo y saquean la ciudad. Sedequías con cobardía escapa junto con las mejores tropas de noche y abandonan al pueblo. El ejército se da cuenta y lo persigue hasta alcanzarlo en las llanuras de Jericó, Sedequías paga por su rebeldía y frente a él matan a sus hijos y le sacan los ojos en $2^{o}$ Reyes 25:7.\\
 Posteriormente quedan pocos en Jerusalén pero esa gente se rebela contra los caldeos y al no tener forma de defenderse se van a Egipto. Israel regresa a Egipto, escogieron la muerte antes de la vida y cumplen la profecía de Deuteronomio 30:15-20 pues el pueblo regresa a donde habían sido esclavos. A pesar de que Dios había enviado a un gran mnúmero de profetas, ellos nunca le hicieron caso, Dios fue muy paciente y les dio muchas oportunidades para arrepentirse\\
 Es peligroso para un pueblo desobediente dar por sentado que está dentro de la gracia de Dios y es indispensable que todos aquellos que somos del pueblo de Dios le obedezcamos. El reino del sur fue deportado a Babilonia pero Dios iba a permitir que un remanente regresara a Jerusalén a reconstruir.
			\end{enumerate}
		\end{subsubsection}
		\end{subsection}
\end{section}
%\end{document}


